\documentclass[12pt]{article}

\usepackage{bm}
\usepackage{amsmath}
\usepackage{amsfonts}
\usepackage{amssymb}
\usepackage{graphicx}
\usepackage{colortbl}
\usepackage{xr}
\usepackage{hyperref}
\usepackage{longtable}
\usepackage{multirow}
\usepackage{xfrac}
\usepackage{tabularx}
\usepackage{float}
\usepackage{siunitx}
\usepackage{booktabs}

%\usepackage{refcheck}

\hypersetup{
    bookmarks=true,         % show bookmarks bar?
      colorlinks=true,       % false: boxed links; true: colored links
    linkcolor=red,          % color of internal links (change box color with linkbordercolor)
    citecolor=green,        % color of links to bibliography
    filecolor=magenta,      % color of file links
    urlcolor=cyan           % color of external links
}
\newif\ifcomments\commentstrue

\ifcomments
\newcommand{\authornote}[3]{\textcolor{#1}{[#3 ---#2]}}
\newcommand{\todo}[1]{\textcolor{red}{[TODO: #1]}}
\else
\newcommand{\authornote}[3]{}
\newcommand{\todo}[1]{}
\fi

\newcommand{\wss}[1]{\authornote{blue}{SS}{#1}}
\newcommand{\bmac}[1]{\authornote{red}{BM}{#1}}

\newcommand{\NN}[1]{{\color{red}#1}}
\newcommand{\WSS}[1]{{\color{blue}#1}}

\newcommand{\colZwidth}{1.0\textwidth}
\newcommand{\blt}{- } %used for bullets in a list
\newcommand{\colAwidth}{0.13\textwidth}
\newcommand{\colBwidth}{0.82\textwidth}
\newcommand{\colCwidth}{0.1\textwidth}
\newcommand{\colDwidth}{0.05\textwidth}
\newcommand{\colEwidth}{0.8\textwidth}
\newcommand{\colFwidth}{0.17\textwidth}
\newcommand{\colGwidth}{0.5\textwidth}
\newcommand{\colHwidth}{0.28\textwidth}
\newcounter{defnum} %Definition Number
\newcommand{\dthedefnum}{GD\thedefnum}
\newcommand{\dref}[1]{GD\ref{#1}}
\newcounter{datadefnum} %Datadefinition Number
\newcommand{\ddthedatadefnum}{DD\thedatadefnum}
\newcommand{\ddref}[1]{DD\ref{#1}}
\newcounter{theorynum} %Theory Number
\newcommand{\tthetheorynum}{T\thetheorynum}
\newcommand{\tref}[1]{T\ref{#1}}
\newcounter{tablenum} %Table Number
\newcommand{\tbthetablenum}{T\thetablenum}
\newcommand{\tbref}[1]{TB\ref{#1}}
\newcounter{assumpnum} %Assumption Number
\newcommand{\atheassumpnum}{P\theassumpnum}
\newcommand{\aref}[1]{A\ref{#1}}
\newcounter{goalnum} %Goal Number
\newcommand{\gthegoalnum}{P\thegoalnum}
\newcommand{\gsref}[1]{GS\ref{#1}}
\newcounter{instnum} %Instance Number
\newcommand{\itheinstnum}{IM\theinstnum}
\newcommand{\iref}[1]{IM\ref{#1}}
\newcounter{reqnum} %Requirement Number
\newcommand{\rthereqnum}{P\thereqnum}
\newcommand{\rref}[1]{R\ref{#1}}
\newcounter{lcnum} %Likely change number
\newcommand{\lthelcnum}{LC\thelcnum}
\newcommand{\lcref}[1]{LC\ref{#1}}

\newcommand{\tclad}{T_\text{CL}}
\newcommand{\degree}{\ensuremath{^\circ}}
\newcommand{\progname}{SWHS}


\usepackage{fullpage}

\begin{document}

\title{Verification and Validation Plan for Solar Water Heating Systems Incorporating 
Phase Change Material} 
\author{Maya Grab and Brooks MacLachlan}
\date{\today}
	
\maketitle

\tableofcontents

%%%%%%%%%%%%%%%%%%%%%%%%
%
%	1.) General Information 
%
%%%%%%%%%%%%%%%%%%%%%%%%

\section{General Information}
The following section provides an overview of the Verification and Validation (V\&V) Plan 
for a solar water heating system incorporating phase change material simulator.
 This section explains the purpose of this document, the scope of the system,
  common definitions, acronyms and abbreviations that are used in the document,
   and an overview of the following sections.

%1.1 Purpose
\subsection{Purpose}
The main purpose of this document is to describe the verification and validation 
process that will be used to test a simulation for solar water heating systems incorporating PCM.
This document is indented to be used as a reference for all future testing and will
be used to increase confidence in the software implementation.  

This document will be used as a starting point for the verification and validation report. The 
test cases presented within this document will be executed and the output will be analyzed to 
determine if the software is implemented correctly.  


%1.2 Scope
\subsection{Scope}

\bmac{No content... should this be here?}


%1.3  Definitions, Acronyms, and abbreviations 
\subsection{Acronyms, Abbreviations, and Symbols }

\renewcommand{\arraystretch}{1.2}
\begin{tabular}{l l} 
  \toprule		
  \textbf{symbol} & \textbf{description}\\
  \midrule 
  QA		&Quality assurance\\
  SRS		&Software requirements specification\\
  V\&V		& Verification and validation\\
  V\&VP 	& Verification and validation plan\\
  V\&VR 	& Verification and validation report\\
  PCM		& Phase change material\\
  SWHS		& Solar Water Heating System\\
  $\epsilon$& $10^{-2}$\\
  \bottomrule
\end{tabular}\\

\bmac{The value for epsilon in the actual Matlab scripts is different.}

%1.4 Overview of Document
\subsection{Overview of Document }
The following sections provide more detail about the V\&V of a solar water heating
 simulator. Information about the testing process is provided, and the software specifications
that were discussed in the SRS document are stated.  The evaluation process that will be followed during 
testing is outlined, and test cases for both the system testing and unit testing are provided.

%%%%%%%%%%%%%%%%%%%%%%%%
%
%	2.) Plan
%
%%%%%%%%%%%%%%%%%%%%%%%%

\section{Plan}
This section provides a description of the software that is being tested, the team that will
perform the testing, the milestones for the testing phase, and the budget allocated to the testing. 

%2.1 Software Description
\subsection{Software Description}
The software being tested is a simulator for a SWHS
incorporating PCM. Given the physical parameters of the system,
 including dimensions, properties of the water and PCM, and relevant physical constants,
  the simulator calculates the changes in temperature and energy of the water and PCM 
  over time.

%2.2 Test Team
\subsection{Test Team} 
The team that will execute the test cases, write and review the V\&VR consists of:

\begin{itemize}
 \item Maya Grab 
 \item Dr.\ Spencer Smith
 \item Thulasi Jegatheesan 
\end{itemize}  

%2.3 Milestones
\subsection{Milestones}

%2.3.1 Location
\subsubsection{Location}
The location where the testing will be performed is Hamilton, Ontario. The institution that
will be performing the testing is McMaster University. 


%2.3.1 Dates and Deadlines
\subsubsection{Dates and Deadlines}
Test Case:
~\newline
The creation of the test cases for both system testing and unit testing is 
scheduled to begin June $1^\text{st}$ 2015.
The deadline for the creation of the test cases is June 15th 2015. 
~\newline
~\newline
Test Case Implementation:
~\newline
Implementing code for the automation of the unit testing is scheduled to begin 
June 15th 2015. The implementation period
is expected to last approximately two weeks and has a deadline of June 30th 2015.
~\newline
~\newline
Verification and Validation Report:
~\newline
The writing of the V\&VR is scheduled to begin July 1st 2015 and end on July 15th 2015. 

%2.4 Budget
\subsection{Budget}
The budget for the testing of this system is being funded by McMaster University and NSERC.

%%%%%%%%%%%%%%%%%%%%%%%%
%
%	3.) Software Specification
%
%%%%%%%%%%%%%%%%%%%%%%%%

\section{ Software Specification}
This section provides the functional requirements, the business tasks that the
software is expected to complete, and the nonfunctional requirements, the
qualities that the software is expected to exhibit.

%3.1 Functional Requirements
\subsection{Functional Requirements}

\noindent
\begin{itemize}
\item Input the physical constants, properties and initial temperatures of water
 and PCM, and dimensions of the tank  
\item Verify that the inputs satisfy the required physical constraints 
\item Compute the calculated values required to solve the governing differential equations
\item Calculate the temperatures and energy of water and PCM over time.
\end{itemize} 

%3.2 Nonfunctional Requirements
\subsection{Nonfunctional Requirements}
Priority nonfunctional requirements are correctness, understandability, reliability, and maintainability. 


%%%%%%%%%%%%%%%%%%%%%%%%
%
%	4.) Evaluation
%
%%%%%%%%%%%%%%%%%%%%%%%%

\section{Evaluation}
This section first presents the methods and constraints that are to be used during
the evaluation process. This is followed by how the data obtained by the testing will be 
evaluated, which includes: how the data will be recorded, how to move from one test
to the next, and how to determine if the test was successful. 

%4.1 Methods and Constraints
\subsection{ Methods and Constraints} 

%4.1.1 Methodology
\subsubsection{Methodology} 
The testing of the SWHS will be fully automated with the exception of one testing 
method, where a change to the ODE solver algorithm is required.

% 4.1.2 Extent of Testing
\subsubsection{Extent of Testing}
The extent of testing that will be employed is extensive. The unit test
 cases below provide complete code coverage and will increase confidence in the
  verification of the software. The system test cases increase confidence in the 
  validation of the system.

% 4.1.3 Test Tools
\subsubsection{Test Tools}
A unit testing framework will be used to implement the unit test cases and run them automatically.

The following equation will be implemented in a script in order to compare the outputs of different implementation:
$$\Delta_{\text{relative}} = \frac{\text{True} - \text{Calculated}}
 {\text{True}} $$

% 4.1.4 Testing Constraints
\subsubsection{ Testing Constraints}
There are currently no anticipated limitations on the testing.

% 4.2  Data Evaluation
\subsection{ Data Evaluation}

% 4.2.3 Testing Criteria
\subsubsection{ Testing Criteria}
Test criteria are divided into two categories:
\begin{enumerate}
\item \textbf{Numerical}: testing results will be compared to expected results within an allowable margin of error $\epsilon$.

\item \textbf{Error catching}: testing will ensure that faulty and unrecommended
 inputs are caught in exceptions and warnings. 
\end{enumerate}

% 4.2.4 Test Data Reduction
\subsubsection{ Testing Data Reduction}
The results of the test data will be evaluated on a PASS/FAIL basis. If the actual 
results match the expected
results the test will be considered a PASS, otherwise the test is considered a FAIL. 


%%%%%%%%%%%%%%%%%%%%%%%%
%
%	5.) System Test Description
%
%%%%%%%%%%%%%%%%%%%%%%%%

\section{System Test Description}


%5.x Test identifier 
\subsection{Faulty Input}

% 5.x.2 Input
\subsubsection{ Input}
The input will be based on the Data Constraints on Input table provided 
in the appendix of this document (borrowed from Input Variables table in the SRS document). Each test will correspond to one entry from 
the physical constraints column, altering a specific input variable to a non-permissible
 value. The list of inputs is in order with the entries in the table, though note there are several cases tested for each constraint described.

\begin{center}
	\begin{longtable}{ | r | p{4cm} | p{4cm} | p{4cm} | p{4cm} |}
	\caption{Faulty Input Cases} \\ \hline \label{TblFaultyInput} 
	No. & Input & Expected Outcome & MsgID \\ \hline
	01 & $L = -2$ & error: Tank length must be $> 0$ & input:L \\ \hline
	02 & $L = 0$ & error: Tank length must be $> 0$ & input:L\\ \hline
	03 & $D = -2$ & error: Tank diameter must be $> 0$ & input:diam\\ \hline
	04 & $D = 0$ & error: Tank diameter must be $> 0$ & input:diam \\ \hline
	05 & $V_P = -0.05$ &error: PCM volume must be $> 0$ & input:Vp\\ \hline
	06 & $V_P = 0$ &error: PCM volume must be $> 0$ & input:Vp\\ \hline
	\multirow{3}{*}{07} & $L = 0.5$ & \multirow{3}{*}{\parbox{4cm}{error: Tank volume must 
	be $>$ PCM volume}} & \multirow{3}{*}{input:VpVt} \\
	& $D = 0.5$ & & \\
	& $Vp = 0.5$ & & \\ \hline
	\multirow{6}{*}{08} & $V_P$ & \multirow{6}{*}{\parbox{4cm}{error: Tank volume must be $>$ PCM volume}} & \multirow{6}{*}{input:VpVt} \\ 
	& $=0.199974938771605$ & & \\
	& (tank volume) & & \\
	& $A_P$ & & \\
	& $=2.208137511613965$ & & \\
	& (tank surface area) & & \\ \hline
	09 & $A_P = -1.5$ &error: PCM area must be $> 0$ & input:Ap\\ \hline
	10 & $A_P = 0$ &error: PCM area must be $> 0$ & input:Ap \\ \hline
	11 & $\rho_P = -1000$ &error: rho\_p must be $> 0$ & input:rho\_{p} \\ \hline
	12 & $\rho_P = 0$ &error: rho\_p must be $> 0$ & input:rho\_{p} \\ \hline
	13 & $T^P_{\text{melt}} = -10$ & error: Tmelt must be $> 0$ and $<$ Tc & 
	input:Tmelt \\ \hline
	14 & $T^P_{\text{melt}} = 0$ & error: Tmelt must be $> 0$ and $<$ Tc 
	&input:Tmelt \\ \hline
	\multirow{2}{*}{15} & $T^P_{\text{melt}} = 45$ & \multirow{2}{*}{\parbox{4cm}{error: 
	Tmelt must be $> 0$ and $<$ Tc}} & input:Tmelt \\
	& $T_C = 40$ & & \\ \hline
	16 & $C^S_P = -1000$ & error: C\_ps must be $> 0$ &input:C\_{ps} \\ \hline
	17 & $C^S_P = 0$ & error: C\_ps must be $> 0$ &input:C\_{ps} \\ \hline
	18 & $C^L_P = -1000$ & error: C\_pl must be $> 0$ &input:C\_{pl} \\ \hline
	19 & $C^L_P = 0$ & error: C\_pl must be $> 0$ &input:C\_{pl} \\ \hline
	20 & $H_f = -200000$& error: Hf must be $> 0$ & input:Hf \\ \hline
	21 & $H_f = 0$ & error: Hf must be $> 0$ & input:Hf \\ \hline
	22 & $A_C = -0.12$ & error: Ac must be $> 0$ &input:Ac \\ \hline
	23 & $A_C = 0$ & error: Ac must be $> 0$ &input:Ac \\ \hline
	24 & $T_C = -50$ & error: Tmelt must
	 be $> 0$ and $<$ Tc &input:Tmelt \\ \hline
	25 & $T_C = 0$ & error: Tmelt must
	 be $> 0$ and $<$ Tc &input:Tmelt \\ \hline
	26 & $T_C = 100$ & error: Tc must be $> 0 \text{ and } < 100$ &input:Tc \\ \hline
	27 & $T_C = 110$ & error: Tc must be $> 0 \text{ and } < 100$ &input:Tc \\ \hline
	28 & $\rho_W = -1000$ & error: rho\_w must be $> 0$ &input:rho\_{w} \\ \hline
	29 & $\rho_W = 0$ & error: rho\_w must be $> 0$ &input:rho\_{w} \\ \hline
	30 & $C_W = -4000$ & error: C\_w must be $> 0$ &input:C\_{w} \\ \hline
	31 & $C_W = 0$ & error: C\_w must be $> 0$ &input:C\_{w} \\ \hline
	32 & $h_C = -1000$ & error: hc must be $> 0$ &input:hc \\ \hline
	33 & $h_C = 0$ & error: hc must be $> 0$ &input:hc \\ \hline
	34 & $h_P = -1000$ & error: hp must be $> 0$ &input:hp \\ \hline
	35 & $h_P = 0$ & error: hp must be $> 0$ &input:hp \\ \hline
	36 & $T_{\text{init}} = -5$ & error: Tinit must be $> 0$ and $< 100$ &input:Tinit \\ \hline
	37 & $T_{\text{init}} = 0$ & error: Tinit must be $> 0$ and $< 100$ &input:Tinit \\ \hline
	38 & $T_{\text{init}} = 100$ & error: Tc must be $>$ Tinit & input:TcTinit \\ \hline
	39 & $T_{\text{init}} = 110$ & error: Tc must be $>$ Tinit & input:TcTinit \\ \hline
	40 & $T_{\text{init}} = 45$ &error: Tinit must be $<$ Tmelt & input:TinitTmelt \\ \hline
	41 & $T_{\text{init}} = 50$ &error: Tc must be $>$ Tinit & input:TcTinit \\ \hline
	42 & $T_{\text{init}} = 60$ &error: Tc must be $>$ Tinit & input:TcTinit \\ \hline
	43 & $t_{\text{final}} = 0$ &error: tfinal must be $> 0$ &input:tfinal \\ \hline
	44 & $t_{\text{final}} = -50000$ &error: tfinal must be $> 0$ &input:tfinal \\ \hline
	\end{longtable}
\end{center}

% 5.x.4 Test  Procedures
\subsubsection{Preparation and Procedure}
Input files are to be generated for each test case and be stored in an appropriate
 directory. The automated test should include a setup procedure that would add 
 the directory to the Matlab path. Each test case runs main.m on an input file
  and checks for the appropriate MsgId. 

\subsection{Unrecommended Input}

\subsubsection{Input}
The input will be based on the Data Constraints on Input table provided 
in the appendix of this document (borrowed from Input Variables table in the SRS document). Each test will correspond to one entry from 
the physical constraints column, altering a specific input variable to a non-advisable
 value. The list of inputs is in order with the entries in the table, though note there are several cases tested for each constraint described.
 
\begin{center}
	\begin{longtable}{ | r | p{3cm} | p{5cm} | p{4cm} |}
	\caption{Unrecommended Input Cases} \\ \hline \label{TblUnrecommendedInput} 
	No. & Input & Expected Outcome & MsgID \\ \hline
	\multirow{2}{*}{01} & $L = 0.01$ & \multirow{2}{*}{\parbox{5cm}{It is recommended that $0.1 <=$ L $<= 50$}}& inputwarn:L\\ 
	& $V_P = 0.001$ & & \\ \hline
	02 & $L = 55$ & It is recommended that $0.1 <=$ L $<= 50$ & intputwarn:L\\ \hline
	03 & $L = 30$ & It is recommended that 
	$0.002 <=$ D/L $<= 200$ & inputwarn:diam \\ \hline
	04 & $D = 400$ & It is recommended that $0.002 <=$ D/L $<= 200$ & inputwarn:diam\\ \hline
	\multirow{3}{*}{05} & $L = 15$ & \multirow{3}{*}{\parbox{5cm}{It is recommended that Vp be $>= 0.0001\%$ of Vt}} & \multirow{3}{*}{inputwarn:VpVt}\\ 
	& $D = 4.12$ & & \\
	& $V_P = 0.0006$ & & \\ \hline
	06 & $A_P = 110$ & It is recommended that Vp $<=$ Ap $<=$ 2*Vp/0.001 & inputwarn:VpAp \\ \hline
	07 & $\rho_P = 450$ & It is recommended that $500 < \text{rho\_{p}} < 20000$ &inputwarn:rho\_{p} \\ \hline
	08 & $\rho_P = 20005$ & It is recommended that $500 < \text{rho\_{p}} < 20000$ & inputwarn:rho\_{p}\\ \hline
	09 & $C^S_P = 90$ & It is recommended that $100 < \text{C\_{ps}} < 4000$ & inputwarn:C\_{ps}\\ \hline
	10 & $C^S_P = 5000$ & It is recommended that $100 < \text{C\_{ps}} < 4000$ & intwarn:C\_{ps}\\ \hline
	11 & $C^L_P = 90$ & It is recommended that $100 < \text{C\_{pl}} < 5000$ & intwarn:C\_{lp} \\ \hline
	12 & $C^L_P = 5005$ & It is recommended that $100 < \text{C\_{pl}} < 5000$ & intwarn:C\_{pl} \\ \hline
	13 & $H_f = \text{min}$ & & \\ \hline
	14 & $H_f = \text{max}$ & & \\ \hline
	15 & $A_C = 0.7$ & It is recommended that Ac $<=$ pi*D/2 & intwarn:Ac\\ \hline
	16 & $\rho_W = 900$ & It is recommended that $950 < \text{rho\_{w}} <= 1000$ & intwarn:rho\_{w} \\ \hline
	17 & $\rho_W = 1010$ & It is recommended that $950 < \text{rho\_{w}} <= 1000$ & intwarn:rho\_{w} \\ \hline
	18 & $C_W = 4160$ & It is recommended that $4170 < \text{C\_{w}} < 4210$ & intwarn:C\_{w} \\ \hline
	19 & $C_W = 4220$ & It is recommended that $4170 < \text{C\_{w}} < 4210$ & intwarn:C\_{w} \\ \hline
	20 & $t_{\text{final}} = 86500$ & It is recommended that $0 < \text{tfinal} < 86400$ & intwarn:tfinal \\ \hline
	\end{longtable}
\end{center}

\subsubsection{Preparation and Procedure}
Input files are to be generated for each test case and be stored in an appropriate
 directory. The automated test should include a setup procedure that would add 
 the directory to the Matlab path. Each test case runs main.m on an input file
  and checks for the appropriate MsgId. 

\subsection{Closed Form Solution for Latent Heating}
\subsubsection{Means of Control}
A closed form solution for the temperature of water is derived using Maple18 for
a case of latent heating. The solution vector generated over time will be 
compared to the latent heating stage from the output generated by Standard Input Variables.
\subsubsection{Input}
The file will be based on the Standard input table (Table \ref{TblInputVar}) and its output. Since the closed
 form solution is only correct for latent heating, the initial temperature will
  be set at the melting temperature of the PCM, and the time span will be the 
  amount of time required for full melting.

\subsubsection{Expected Output}
The delta between the standard out vector and the closed form solution must be 
$< \epsilon$

\subsubsection{Preparation and Procedure}
The closed form solution found:
$$ T_P(t) = T_{\text{init}}$$
\\
$$T_W(t) = \frac{(T_{\text{init}}-T_{\text{C}})e^{-\frac{(\eta+1)t}{\tau_{\text{W}}}}+T_{\text{init}}\eta+T_{\text{C}}}{\eta+1}
$$



\subsection{Alternative Algorithm}
\subsubsection{Means of Control}
The program will be manually compared to an implementation substituting the ode45
 algorithm with ode23. 

\subsubsection{Input}
The input will be the standard working input, run once through the ode45 implementation
 and once through the ode23 implementation. 

\subsubsection{Expected Output}
The delta between the output vectors must be 
$< \epsilon$.

\subsubsection{Preparation and Procedure}
The ode45 function will be manually substituted with ode23 algorithm in main.m in order to compare the implementations. Both functions will be run on the input file and results will be compared.

\subsection{Comparison to Original Implementation}
\subsubsection{Means of Control}
Valid output from the current Matlab implementation will be compared to output
 from the original Fortran implementation.
 
\subsubsection{Input}
The standard input file (outlined in Table \ref{TblInputVar}) for Matlab and the corresponding file in Fortran format will be run through their respective implementations. 
 \\ The standard input file will be modified to five different variations, as described in the table below.
 
 \begin{center}
	\begin{longtable}{ | r | p{4cm} | p{4cm} |}
	\caption{Faulty Input Cases} \\ \hline \label{TblOrigImplement} 
	No. & Purpose & $\Delta$ Input \\ \hline
	01 & & Standard \\ \hline
	\multirow{3}{*}{02} & \multirow{3}{*}{\parbox{4cm}{Set coil temperature just above melting temperature of PCM.}} & $T_{\text{C}} = 44.21\degree{C}$ \\
	& & $T_{\text{melt}} = 44.2\degree{C}$ \\
	& & \\ \hline
	03 & Set $t$ at exactly the initial melting time according to No.1 (Fortran
	 implementation). & $t = 20570\text{s}$ \\ \hline
	04 & Set $t$ just above the initial melting time. & $t = 20580\text{s}$ \\ \hline
	\end{longtable}
\end{center}
 
\bmac{Text says 5 cases, but table only shows 4}
 
\subsubsection{Expected Output}
The delta between the output vectors should be
 $< \epsilon$.

\subsubsection{Procedure}


\subsubsection{Preparation}

\bmac{Should these sections be blank?}

%%%%%%%%%%%%%%%%%%%%%%%%
%
%	6.) Unit Test Description 
%
%%%%%%%%%%%%%%%%%%%%%%%%
\section{Temperature Modules}


\subsection{Module Information}
This testing suite will test the three temperature modules handling the governing 
differential equations for temperature of water and temperature of PCM.
\begin{itemize}
\item Temperature1.m handles the case where $T_{\text{P}} < T_{\text{melt}}$.
\item Temperature2.m handles the case where $T_{\text{P}} = T_{\text{melt}}$.
\item Temperature3.m handles the case where $T_{\text{P}} > T_{\text{melt}}$.
\end{itemize}

\subsubsection{Module Inputs}
The Temperature modules take as input:
\begin{itemize}
\item a time $t$;
\item a temperature vector $T$ where $T(1) = T_{\text{W}} \text{ and } 
T(2) = T_{\text{P}}$;
\item an input parameters structure containing the typical simulator parameters
\end{itemize}

\subsubsection{Module Outputs}
The Temperature modules output a vector dTdt where:
\begin{itemize}
\item dTdt(1) = $\frac{dT_{\text{W}}}{dt}$
\item dTdt(2) = $\frac{dT_{\text{P}}}{dt}$
\item dTdt(3) = $\frac{dQ_{\text{P}}}{dt}$ (temperature2.m only)
\end{itemize}

\subsubsection{Related Modules}
The module load\_params.m was used in testing in order to load the input parameters
 structure into each function.

\subsection{Test Data}


\subsubsection{Inputs}
 \begin{center}
	\begin{longtable}{ | r | c |}
	\caption{Temperature Tests Input} \\ \hline \label{TblTempIn} 
	Test & Input \\ \hline
	temperature 1 & \shortstack{\\ $t = 100$ \\ $T = [40.7, 40.5]$ \\params} \\ \hline
	temperature 2a & \shortstack{\\ $t = 3000$ \\ $T = [44.2, 44.2]$ \\ params} \\ \hline
	temperature 2b & \shortstack{\\ $t = 4000$ \\ $T = [45, 44.2]$ \\ params} \\ \hline
	temperature 3 & \shortstack{\\ $t = 25000$ \\ $T = [47, 46.5]$ \\ params} \\ \hline	
		\end{longtable}
\end{center}


\subsubsection{Expected Outputs}
 \begin{center}
	\begin{longtable}{ | r | c |}
	\caption{Temperature Tests Expected Outputs} \\ \hline \label{TblTempOut} 
	Test & Input \\ \hline
	temperature 1 & [0.00139536, 0.002708315] \\ \hline
	temperature 2a & [0.001108642, 0, 0] \\ \hline
	temperature 2b & [-0.000573435, 0, 960] \\ \hline
	temperature 3 & [-0.00038229, 0.005249596] \\ \hline	
		\end{longtable}
\end{center}

\section{Energy Modules}

\subsection{Module Information}
This testing suite will test the three energy modules handling the governing 
equations for energy of water and PCM.
\begin{itemize}
\item energy1.m handles the case where $T_{\text{P}} < T_{\text{melt}}$.
\item energy2.m handles the case where $T_{\text{P}} = T_{\text{melt}}$.
\item energy3.m handles the case where $T_{\text{P}} > T_{\text{melt}}$.
\end{itemize}

\subsubsection{Module Inputs}
The Energy modules take as input:
\begin{itemize}
\item a temperature matrix $T$ where $T(:,1) = T_{\text{W}} \text{, }
T(;,2) = T_{\text{P}} \text{ and (energy2 only) } T(:,3) = Q_{\text{P}}$
\item an input parameters structure containing the typical simulator parameters
\end{itemize}

\subsubsection{Module Outputs}
The Energy modules output two vectors: $E_{\text{W}} \text{ and } E_{\text{P}}$.

\subsection{Test Data}

\subsubsection{Inputs}
 \begin{center}
	\begin{longtable}{ | r | c |}
	\caption{Energy Tests Input} \\ \hline \label{TblEnergyIn} 
	Test & Input \\ \hline
	energy 1 & \shortstack{\\ $T = [40:44; 40:44]'$ \\params} \\ \hline
	energy 2 & \shortstack{\\$T = [44.2:0.1:44.6; 44.2, 44.2, 44.2, 44.2, 44.2; 
	372000:51000:576000]'$ \\ params} \\ \hline
	energy 3 & \shortstack{\\ $T = [45:49; 45:49]'$ \\ params} \\ \hline	
		\end{longtable}
\end{center}

\bmac{In these tables throughout the document, values are expressed using Matlab's syntax. Is that okay, or do we want this document to be completely independent from the choice of software?"}

\subsubsection{Expected Outputs}
 \begin{center}
	\begin{longtable}{ | r | c | c |}
	\caption{Energy Tests Output} \\ \hline \label{TblEnergyOut} 
	Test & $E_{\text{W}}$ & $E_{\text{P}}$ \\ \hline
	energy 1 & \shortstack{\\ 0\\ 627795.0938\\ 1255590.188\\ 1883385.281\\ 2511180.375} &
	 \shortstack{\\ 0\\ 88616\\ 177232\\ 265848\\ 354464} \\ \hline
	energy 2 & \shortstack{\\ 2636739.394\\ 2699518.903\\ 2762298.413\\ 2825077.922\\ 2887857.431} &
	\shortstack{\\ 744187.2\\ 795187.2\\ 846187.2\\ 897187.2\\ 948187.2} \\ \hline
	energy 3 & \shortstack{\\ 3138975.469\\ 3766770.563\\ 4394565.657\\ 5022360.75\\ 5650155.844 } &
	\shortstack{\\ 11117682.8\\ 11231977.3\\ 11346271.8\\ 11460566.3\\ 11574860.8} \\ \hline	
		\end{longtable}
\end{center}


\section{Event Modules}

\subsection{Module Information}
This testing suite will test the two Event modules, which handle the switch 
between the cases $T_{\text{P}} < T_{\text{melt}} \text{, } T_{\text{P}} = T_{\text{melt}}
\text{ and } T_{\text{P}} > T_{\text{melt}}$.

\subsubsection{Module Inputs}
The Event modules take as input:
\begin{itemize}
\item a time $t$;
\item a temperature vector $T$ where $T(1) = T_{\text{W}} \text{, } 
T(2) = T_{\text{P}} \text{ and (event 2 only) } T(3) = Q_{\text{P}}$;
\item an input parameters structure containing the typical simulator parameters
\end{itemize}

\subsubsection{Module Outputs}
The Event modules output three values: [value, isterminal, direction]. 

\subsubsection{Related Modules}
The module load\_params.m was used in testing in order to load the input parameters 
structure into each function.

\subsection{Test Data}

\subsubsection{Inputs}
 \begin{center}
	\begin{longtable}{ | r | c |}
	\caption{Event Tests Input} \\ \hline \label{TblEventIn} 
	Test & Input \\ \hline
	event 1a & \shortstack{\\ $t = 100$ \\ $T = [41, 40.9]$ \\params} \\ \hline
	event 1b & \shortstack{\\ $t = 3000$ \\ $T = [44.2, 44.2]$ \\ params} \\ \hline
	event 2a & \shortstack{\\ $t = 3000$ \\ $T = [44.2, 44.2, 0]$ \\ params} \\ \hline
	event 2b & \shortstack{\\ $t = 4000$ \\ $T = [45, 44.2, 600000]$ \\ params} \\ \hline
	event 2c & \shortstack{\\ $t = 20570$ \\ $T = [44.7, 44.2, 10654060]$ \\ params} \\ \hline	
		\end{longtable}
\end{center}

\subsubsection{Expected Outputs}
 \begin{center}
	\begin{longtable}{ | r | c |}
	\caption{Event Tests Expected Output} \\ \hline \label{TblEventOut} 
	Test & Output \\ \hline
	event 1a & [3.3, 1, 0] \\ \hline
	event 1b & [0, 1, 0] \\ \hline
	event 2a & [-1, 1, 0] \\ \hline
	event 2b & [-0.943683441, 1, 0] \\ \hline
	event 2c & [0, 1, 0] \\ \hline	
		\end{longtable}
\end{center}

\section{Output Verification Module}

\subsection{Module Information}
This testing suite will test the Output Verification module, which verifies that the energy outputs obey the law of conservation of energy.

\subsubsection{Module Inputs}
The Output Verification module takes as input:
\begin{itemize}
	\item an input parameters structure containing the typical simulator parameters;
	\item a vector of times, $t$;
	\item a matrix with at least two columns of temperature values, $T$ where $T(:,1) = T_{\text{W}}$, and $T(:,2) = T_{\text{P}}$;
	\item a vector of values for energy in the water, $Ew$;
	\item a vector of values for energy in the PCM, $Ep$;
\end{itemize}

\subsubsection{Module Outputs}
The Output Verification Module has no outputs.

\subsubsection{Related Modules}
The module load\_params.m was used in testing in order to load the input parameters 
structure into each function.

\subsection{Test Data}

\subsubsection{Input}

\begin{center}
	\begin{longtable}{ | r | c |}
		\caption{Output Verification Tests Input} \\ \hline \label{TblOutVerifyIn} 
		Test No. & Input \\ \hline
		1 &  \shortstack{\\ params \\ t = [0;10;20;30] \\ T = [40 40; 42 41.9; 44 43.8; 46 45.7;] \\ Ew = [0;1000;2000;19800] \\ Ep = [0;1000;2000;5400]}\\ \hline
		2 &  \shortstack{\\ params \\ t = [0;10;20;30] \\ T = [40 40; 42 41.9; 44 43.8; 46 45.7;] \\ Ew = [0;1000;2000;19800] \\ Ep = [0;1000;2000;3000]}\\ \hline
		3 &  \shortstack{\\ params \\ t = [0;10;20;30] \\ T = [40 40; 42 41.9; 44 43.8; 46 45.7;] \\ Ew = [0;1000;2000;3000] \\ Ep = [0;1000;2000;5400]}\\ \hline
		4 &  \shortstack{\\ params \\ t = [0;10;20;30] \\ T = [40 40; 42 41.9; 44 43.8; 46 45.7;] \\ Ew = [0;1000;2000;3000] \\ Ep = [0;1000;2000;3000]}\\ \hline
	\end{longtable}
\end{center}

\subsubsection{Expected Output}

\begin{center}
	\begin{longtable}{ | r | p{9cm} | c |}
		\caption{Output Verification Tests Expected Output} \\ \hline \label{TblOutVerifyOut} 
		Test No. & Expected Warning & MsgID \\ \hline
		1 &  None & N/A \\ \hline
		2 &  'There is greater than 0.00001 relative error between the Ep output and the expected output based on the law of conservation of energy' & 'output:Ep' \\ \hline
		3 &  'There is greater than 0.00001 relative error between the Ew output and the expected output based on the law of conservation of energy' & 'output:Ew'\\ \hline
		4 &  'There is greater than 0.00001 relative error between the Ew output and the expected output based on the law of conservation of energy' and 'There is greater than 0.00001 relative error between the Ep output and the expected output based on the law of conservation of energy' & 'output:Ew' and 'output:Ep'\\ \hline
	\end{longtable}
\end{center}

\section{Appendix}

\begin{table}
\renewcommand{\arraystretch}{1.2}
\caption{Standard Input Variables} \label{TblInputVar}
~\newline
\centering
\noindent \begin{tabular}{l l} 
  \toprule
  \textbf{Var} & \textbf{Typical Value}\\
  \midrule
  $L$	& 1.5 \si[per-mode=symbol]	{\metre}
  \\
  $D$	& 0.412 \si[per-mode=symbol] {\metre}	
  \\
  $V_P$ & 0.05 \si[per-mode=symbol] {\cubic\metre}	
  \\
  $A_P$ & 1.2 \si[per-mode=symbol] {\square\metre}	
  \\
  $\rho_P$ & 1007 \si[per-mode=symbol] {\kilogram\per\cubic\metre}
  \\
  $T_\text{melt}^{P}$ &	44.2 \si[per-mode=symbol] {\celsius} 
  \\
  $C_P^S$ & 1760 \si[per-mode=symbol] {\joule\per\kilo\gram\per\celsius}
  \\
  $C_P^L$ & 2270 \si[per-mode=symbol] {\joule\per\kilo\gram\per\celsius} 
  \\
  $H_f$ & 211600 \si[per-mode=symbol] {\joule\per\kilo\gram} 
  \\
  $A_C$ & 0.12 \si[per-mode=symbol] {\square\metre}
  \\
  $T_C$	& 50 \si[per-mode=symbol] {\celsius}
  \\
  $\rho_W$ & 1000 \si[per-mode=symbol] {\kilo\gram\per\cubic\metre} 
  \\
  $C_W$ & 4186 \si[per-mode=symbol] {\joule\per\kilo\gram\per\celsius}
  \\
  $h_C$ & 1000 \si[per-mode=symbol] {\watt\per\square\metre\per\celsius}
  \\
  $h_P$ & 1000 \si[per-mode=symbol] {\watt\per\square\metre\per\celsius} 
  \\
  $T_\text{init}$ & 40 \si[per-mode=symbol] {\celsius} 
  \\
  $t_\text{final}$ & 50000 \si[per-mode=symbol] {\second} 
  \\
  AbsTol & $10^{-10}$
  \\
  RelTol & $10^{-10}$
  \\
  \bottomrule
\end{tabular}
\end{table}

\newpage

\begin{table*}
\caption{Data Constraints on Input} \label{TblInputConstraints}
\renewcommand{\arraystretch}{1.2}
\noindent \begin{longtable}{l l} 
  \toprule
  \textbf{Var} & \textbf{Physical Constraints} \\
  \midrule 
  $L$	& $L > 0$		
  \\
  $D$	& $D > 0$		
  \\
  $V_P$ & $V_P > 0$ (*)	
  \\
   & $V_P < \pi (D/2)^2 L$
  \\
  $A_P$ & $A_P > 0$ (*)	
  \\
  $\rho_P$ & $\rho_P > 0$	
  \\
  $T_\text{melt}^{P}$ 	& $0 < T_\text{melt}^{P} < T_C$
  \\
  $C_P^S$ & $C_P^S > 0$ 
  \\
  $C_P^L$ & $C_P^L > 0$ 
  \\
  $H_f$ & $H_f > 0$ 
  \\
  $A_C$ & $A_C > 0$ (*)	
  \\
  $T_C$ & $T_C > 0$ (+)	
  \\
  $\rho_W$ & $\rho_W > 0$	
  \\
  $C_W$ & $C_W > 0$	
  \\
  $h_C$ & $h_C > 0$	
  \\
  $h_P$ & $h_P > 0$	
  \\
  $T_\text{init}$ & $0 < T_\text{init} < T_\text{melt} $ (+) 
  \\
  $t_\text{final}$ & $t_\text{final} > 0$ 
  \\
  \bottomrule
\end{longtable}
\end{table*}

\noindent \begin{description}
\item[(*)] These quantities cannot be equal to zero, or there will be a divide by
  zero in the model.
\item[(+)] These quantities cannot be zero, or there would be freezing.
\end{description}

\end{document}
