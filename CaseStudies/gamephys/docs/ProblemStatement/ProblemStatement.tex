\documentclass [12pt] {article}

\usepackage{tabularx} 
\usepackage{booktabs} 
\title{CAS 741: Problem Statement\\Game Physics} 
\author{Oluwaseun Owojaiye and 400223745}

\date{September 18, 2018}

%% Comments

\usepackage{color}

\newif\ifcomments\commentstrue

\ifcomments
\newcommand{\authornote}[3]{\textcolor{#1}{[#3 ---#2]}}
\newcommand{\todo}[1]{\textcolor{red}{[TODO: #1]}}
\else
\newcommand{\authornote}[3]{}
\newcommand{\todo}[1]{}
\fi

\newcommand{\wss}[1]{\authornote{blue}{SS}{#1}}
\newcommand{\an}[1]{\authornote{magenta}{Author}{#1}}


\begin{document}

\maketitle

\begin{table}[hp] \caption{Revision History} \label{TblRevisionHistory}
\begin{tabularx}{\textwidth}{llX} 
\toprule 
\textbf{Date} & \textbf{Developer(s)} & \textbf{Change}\\ 
\midrule September 14, 2018 & Oluwaseun Owojaiye & Initial draft\\ 
September 14, 2018 & Oluwaseun Owojaiye & Added problem description\\
\bottomrule 
\end{tabularx} 
\end{table}

Some aspects of physics are very useful in video game development and most of
the popular games on the market today would never be successful without a good
physics engine. Game physics libraries are used to simulate physical phenomena
in objects e.g collision between two objects in a game. It can be an arduous
task in game development to create a game physics library from scratch, and if
not included in game products, might result in sub-standard video games. If
there are more affordable/free open source game physics library on the market,
game developers would be able to include them in their products, producing high
quality video games. Chipmunk2D (https://chipmunk-physics.net/) is an example of
a physics library widely used in 2D physics games, similar to this project.

The purpose of this software is to create a fast, easy to use and portable game
physics library for simulation of 2D rigid bodies. This library will be used to
simulate how two-dimensional rigid bodies interact with one another when they
are acted upon by a force as well as predict the time history of each objects'
position and orientation.

\end{document}