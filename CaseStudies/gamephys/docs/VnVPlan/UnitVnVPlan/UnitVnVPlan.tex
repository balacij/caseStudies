\documentclass[12pt, titlepage]{article}

\usepackage{booktabs}
\usepackage{tabularx}
\usepackage{hyperref}
\hypersetup{
    colorlinks,
    citecolor=black,
    filecolor=black,
    linkcolor=red,
    urlcolor=blue
}
\usepackage[round]{natbib}

\newcommand{\progname}{Tamias2D}

%% Comments

\usepackage{color}

\newif\ifcomments\commentstrue

\ifcomments
\newcommand{\authornote}[3]{\textcolor{#1}{[#3 ---#2]}}
\newcommand{\todo}[1]{\textcolor{red}{[TODO: #1]}}
\else
\newcommand{\authornote}[3]{}
\newcommand{\todo}[1]{}
\fi

\newcommand{\wss}[1]{\authornote{blue}{SS}{#1}}
\newcommand{\an}[1]{\authornote{magenta}{Author}{#1}}


\begin{document}

\title{Tamias2D: Unit Verification and Validation Plan} 
\author{Oluwaseun Owojaiye}
\date{\today}
	
\maketitle

\pagenumbering{roman}

\section{Revision History}

\begin{tabularx}{\textwidth}{p{3cm}p{2cm}X}
\toprule {\bf Date} & {\bf Version} & {\bf Notes}\\
\midrule
Nov. 29, 2018 & 1.0 & Initial Draft\\
Date 2 & 1.1 & Notes\\
\bottomrule
\end{tabularx}

~\newpage

\section{Symbols, Abbreviations and Acronyms}
See MIS documentation at \url{https://github.com/smiths/caseStudies/blob/gamephy_MIS/CaseStudies/gamephys/docs/Design/MIS/GamePhysicsMIS.pdf} \\
\renewcommand{\arraystretch}{1.2}
\begin{tabular}{l l} 
  \toprule		
  \textbf{symbol} & \textbf{description}\\
  \midrule 
  T & Test\\
  \bottomrule
\end{tabular}\\

%\wss{symbols, abbreviations or acronyms -- you can reference the SRS tables if needed}

\newpage

\tableofcontents

\listoftables

\listoffigures

\newpage

\pagenumbering{arabic}

This document describes the unit verification and validation (V\&V) plan for \progname, a 2D Physics Game Library. It is intended to be a refinement of the tool's system (V\&V) plan by providing test cases based on the modules in the library's module interface specification (MIS) document. The MIS, along with the full documentation of \progname{}, can be found at: \url{https://github.com/smiths/caseStudies/tree/master/CaseStudies/gamephys}.The unit V\&V plan starts by providing general information about the library Section \ref{GeneralInfo}. Then, Section \ref{Plan} provides additional details 

about the plan, which include information about the V\&V team, automated 

testing and verification tools and non-testing based verification. This is 

followed by the unit test description in Section \ref{UnitTestDescription}, 

which consists of tests for the library's functional and nonfunctional 

requirements, categorized based on the modules in the MIS, and traceability 

between the test cases and modules.
This document ... \wss{provide an introductory blurb and roadmap of the
  unit V\&V plan}

\section{General Information}

\subsection{Purpose}

\subsection{Scope}

\subsection{Overview of Document}

\section{Plan}
	
\subsection{Verification and Validation Team}

Member(s) of the verification and validation team will include myself, Olu.

\subsection{Automated Testing and Verification Tools}
PyTest framework will be used for automated unit
testing. A script with all the testcases covering all \progname{} modules  will be created by me. 


\subsection{Non-Testing Based Verification}
Not applicable for \progname

%\wss{List any approaches like code inspection, code walkthrough, symbolic
%  execution etc.  Enter not applicable if that is the case.}

\section{Unit Test Description}
The test cases discussed in this section are based on the Module Interface Specification(MIS) which can be found at \url{https://github.com/smiths/caseStudies/blob/gamephy_MIS/CaseStudies/gamephys/docs/Design/MIS/GamePhysicsMIS.pdf}. Each test case listed covers for all applicable functions that need to be tested in each module to ensure that \progname{} functions as intended based on the requirements of the software.
	
\subsection{Tests for Functional Requirements}

\subsubsection{Body Module}
		
\paragraph{}
Body module is responsible for storing the physical properties of an object such as mass, position, rotation properties, velocity, e.t.c and provides operations on rigid bodies such as setting the mass, moment, applying force e.t.c.
\begin{enumerate}
	

\item{UTC1} {: Validate apply\_force1 \\}

Type: Automatic.
					
Initial State: Initial force on a body is Vec2(0, 0)
					
Input: b.apply\_force(Vec2(10, 6))
					
Output: New total force applied to body is Vec2(10, 6)
					
How test will be performed: PyTest 
					
\item{UTC2}{: Validate apply\_force2\\}

Type: Automatic
					
Initial State: Vec2(10, 6)
					
Input: b.apply\_force (Vec2(-20, 2))
					
Output: New total force is Vec2(-10, 8)

\item{UTC3}{: Validate apply\_force3\\}

Type: Automatic

Initial State: Vec2(-2, -1)

Input: b.apply\_force (Vec2(-5, -8))

Output: New total force is Vec2(-7, -8)
					
How test will be performed: PyTest

\item{UTC4}{: Validate apply\_torque\\}

Type: Automatic

Initial State: 0

Input: apply\_torque(100)

Output: self.torque=100

How test will be performed: PyTest

\item{UTC5}{: Validate set\_space\\}

Type: Automatic

Initial State: 

Input: body.set\_space(space)

Output: Expected result of(space is body.space): True

How test will be performed: PyTest\\

{-Not sure how to unit test set\_space, each body has a reference to space, and it accesses the gravity in space.\\
-update: updates parameters after a cycle\\}
\end{enumerate}

\subsubsection{Vector Module}

\paragraph{}
Vector module provides operations such as addition, scalar and vector multiplication, dot and cross products e.t.c.
\begin{enumerate}
	
	\item{UTC6}{: Validate vector addition \\}
	
	Type: Automatic.
	
	Initial State: 
	
	Input: Vec2(2,5) + Vec2(1,1)
	
	Output: Vec2(3,6)
	
	How test will be performed: PyTest 
	
	\item{UTC7}{: Validate vector subtraction\\}
	
	Type: Automatic
	
	Initial State: 
	
	Input: Vec2(2,5) - Vec2(1,1)
	
	Output: New total force is Vec2(-1, 4)
	
	\item{UTC8}{: Validate scalar multiplication\\}
	
	Type: Automatic
	
	Initial State: 
	
	Input: apply force Vec(2, 4) * 2
	
	Output: New total force is Vec2(4, 8)
	
	How test will be performed: PyTest
	
	\item{UTC9}{: Validate scalar division\\}
	
	Type: Automatic
	
	Initial State: 
	
	Input: apply force Vec(2, 4) / 2
	
	Output: New total force is Vec2(1, 2)
	
	How test will be performed: PyTest
	
	\item{UTC10}{: Validate vector magnitude\\}
	
	Type: Automatic
	
	Initial State: 
	
	Input: Velocity = Vec(2, 0)
	
	Output: magnitude = V.mag() = 2.0
	
	How test will be performed: PyTest\\
	Ref: \url{https://www.omnicalculator.com/math/distance}
	
	\item{UTC11}{: Validate vector dot product\\}
	
	Type: Automatic
	
	Initial State: 
	
	Input: Vec2.dot(Vec2(1,2), Vec2(2,1))
	
	Output: magnitude = 4.0
	
	How test will be performed: PyTest
	
\end{enumerate}

\subsubsection{Shape Module}

\paragraph{}
Shape module is responsible for storing the surface properties of an object such as restitutionand provides operations on shapes, such as setting the coefficient of restitution. e.t.c.
\begin{enumerate}
	
	
	\item{UTC12} {: Validate set\_angle \\}
	
	Type: Automatic.
	
	Initial State: 
	
	Input: shape.set\_angle(45)
	
	Output: self.angle = 45
	
	How test will be performed: PyTest 
	
	\item{UTC13}{: Validate set\_position\\}
	
	Type: Automatic
	
	Initial State: 
	
	Input: shape.set\_pos(Vec2(2,2))
	
	Output: Vec2(2,2)
	
	\item{UTC14}{: Validate get\_vertices\\}
	
	Type: Automatic
	
	Initial State: 
	
	Input: shape.get\_verts(Box(Vec2(0,0), Vec2(100,100), 0)) 
	
	Output: 4
	
	How test will be performed: PyTest
	

\end{enumerate}
\subsubsection{Space Module}

\paragraph{}
The space module is responsible for all the rigid bodies and shape interaction. It is the container for simulation.
\begin{enumerate}
	
	
	\item{UTC15} {: Validate set\_gravity \\}
	
	Type: Automatic.
	
	Initial State: 
	
	Input: space.set\_gravity(Vec2(0, 9.8))
	
	Output: (Vec2(0, 9.8)
	
	How test will be performed: PyTest 
	
	\item{UTC16}{: Validate add\_body\\}
	
	Type: Automatic
	
	Initial State: 
	
	Input: space.add\_body(b)
	
	Output: len(self.bodies) = 1
	

	\end{enumerate}

\subsection{Tests for Nonfunctional Requirements}

Non-Functional testing will not be required for unit testing. This will be covered in section 5.2 of the System Verification and Validation document located at: 

\url{https://github.com/smiths/caseStudies/blob/gamephy_SysVnVPlan/CaseStudies/gamephys/docs/VnVPlan/SystVnVPlan/SystVnVPlan.pdf.pdf}

\subsection{Traceability Between Test Cases and Requirements}


\begin{table} [h!]
	
	\centering
	
	\begin{tabular}{|c|c|}
		
		\hline	
		
		\textbf{Test Case ID} & \textbf{Module}\\
		
		\hline 
		
	
		UTC1-UTC5& Body Module\\ \hline
		
		UTC6 - UTC11& Vector  Module\\ \hline
		
		UTC12 - UTC14&  Shape Module\\ \hline
		
		UTC15 - & Module\\ \hline
		
		T5& Module\\ \hline
		
	\end{tabular}
	
	\caption{Traceability Between Test Cases and Modules}
	
	\label{Table:Traceability} 
	
\end{table}

~\newpage
\bibliographystyle{plainnat}

\bibliography {../../../refs/References}

~\newpage



\section{Appendix}



This section provides additional content related to this document.



\subsection{Symbolic Parameters}



There are no symbolic parameters used in this document.



\end{document}
