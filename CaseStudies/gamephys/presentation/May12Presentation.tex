\documentclass{beamer}

\mode<presentation>
{
%  \usetheme{Hannover}
\usetheme[width=0.7in]{Berkeley}
% or ...

  \setbeamercovered{transparent}
  \usecolortheme{crane}
  % or whatever (possibly just delete it)
}

\usepackage{longtable}
\usepackage{qtree}

\usepackage[english]{babel}
% or whatever

\usepackage[latin1]{inputenc}
% or whatever

\usepackage{utopia}
\usepackage[T1]{fontenc}
% Or whatever. Note that the encoding and the font should match. If T1
% does not look nice, try deleting the line with the fontenc.

\usepackage{hhline}
\usepackage{multirow}
\usepackage{multicol}
\usepackage{array}
%\usepackage{supertabular}
\usepackage{amsmath}
\usepackage{amsfonts}
\usepackage{totpages}
\usepackage{hyperref}
\usepackage{booktabs}

\usepackage{bm}

\usepackage{listings}
\newcommand{\blt}{- } %used for bullets in a list

\newcounter{datadefnum} %Datadefinition Number
\newcommand{\ddthedatadefnum}{DD\thedatadefnum}
\newcommand{\ddref}[1]{DD\ref{#1}}

\newcommand{\colAwidth}{0.2\textwidth}
\newcommand{\colBwidth}{0.73\textwidth}

\renewcommand{\arraystretch}{.9} %so that tables with equations do not look crowded

%\pgfdeclareimage[height=0.7cm]{logo}{McMasterLogo}
%\title[\pgfuseimage{logo}]  % (optional, use only with long paper titles)
\title{Chipmunk2D - Game Physics Engine}

%\subtitle
%{Include Only If Paper Has a Subtitle}

\author[Slide \thepage~of \pageref{TotPages}] % (optional, use only with lots of
                                              % authors)
{Luthfi Mawarid}
% - Give the names in the same order as the appear in the paper.
% - Use the \inst{?} command only if the authors have different
%   affiliation.

\institute[McMaster University] % (optional, but mostly needed)
{
  Computing and Software Department\\
  Faculty of Engineering\\
  McMaster University
}
% - Use the \inst command only if there are several affiliations.
% - Keep it simple, no one is interested in your street address.

\date[May 12, 2016] % (optional, should be abbreviation of conference name)
{May.\ 12, 2016}
% - Either use conference name or its abbreviation.
% - Not really informative to the audience, more for people (including
%   yourself) who are reading the slides online

\subject{computational science and engineering, software engineering, software
  quality, literate programming, software requirements specification, document
  driven design}
% This is only inserted into the PDF information catalog. Can be left
% out. 

% If you have a file called "university-logo-filename.xxx", where xxx
% is a graphic format that can be processed by latex or pdflatex,
% resp., then you can add a logo as follows:

%\pgfdeclareimage[height=0.5cm]{Mac-logo}{McMasterLogo}
%\logo{\pgfuseimage{Mac-logo}}

% Delete this, if you do not want the table of contents to pop up at
% the beginning of each subsection:
\AtBeginSubsection[]
{
  \begin{frame}<beamer>
    \frametitle{Outline}
    \tableofcontents[currentsection,currentsubsection]
  \end{frame}
}

% If you wish to uncover everything in a step-wise fashion, uncomment
% the following command: 

%\beamerdefaultoverlayspecification{<+->}

\beamertemplatenavigationsymbolsempty 

% have SRS and LP open during the presentation

\begin{document}

%%%%%%%%%%%%%%%%%%%%%%%%%%%%%%%%%%%%%%
\begin{frame}

\titlepage

\end{frame}

%%%%%%%%%%%%%%%%%%%%%%%%%%%%%%%%%%%%%%

\begin{frame}

\frametitle{Overview}
\tableofcontents
% You might wish to add the option [pausesections]

% make like a story - the phases - reason for, why works, advantages
% changing the history a bit to make a more rational narrative

\end{frame}

%%%%%%%%%%%%%%%%%%%%%%%%%%%%%%%%%%%%%%

\section[Purpose]{Purpose of the Software}

%%%%%%%%%%%%%%%%%%%%%%%%%%%%%%%%%%%%%%

\begin{frame}

\frametitle{Purpose of the Software}

\begin{itemize}
	\item Free and reliable open-source physics library for game developers
	\item Allows developers to implement realistic physics in their video games
	\item Leads to production of higher quality games at little extra time and cost
	\item Fast, portable and lightweight
	\begin{itemize}
		\item High reusability - bindings and ports are available in several other languages (e.g. Python, Java, Objective-C, C++, etc.)
		\item Fast and lightweight - efficient and provides physics modelling functionalities without significantly increasing the resource cost of the games
	\end{itemize}
\end{itemize}

\end{frame}

%%%%%%%%%%%%%%%%%%%%%%%%%%%%%%%%%%%%%%

\section[Inputs]{Software Inputs}

%%%%%%%%%%%%%%%%%%%%%%%%%%%%%%%%%%%%%%

\begin{frame}

\frametitle{Software Inputs}

Software inputs mainly involve the initial physical properties of the rigid body (or bodies) being simulated, such as:

\begin{itemize}
	\item Initial kinematic properties of the rigid bodies (such as initial mass, velocity, position, orientation, angular velocity, etc.)
	\item Surface and material properties of the bodies (such as friction, elasticity, coefficient of restitution, etc.)
	\item Properties of constraints (such as the length of a pin joint, positions of where the bodies are held by the constraints, the error reduction parameter, etc.)
\end{itemize}

The software will validate and ensure that the inputs satisfy physical constraints.

\end{frame}

%%%%%%%%%%%%%%%%%%%%%%%%%%%%%%%%%%%%%%

\begin{frame}
	
	\frametitle{Examples of Input Variables}
	
	\begin{center}
	\begin{tabular}{||c | c | c||}
		\hline 
		Var & Quantity & Unit \\
		\hline\hline 
		\textit{d} & displacement & $m$ \\
		\textit{v} & velocity & $m/s$ \\
		\textit{a} & acceleration & $m/s^2$ \\
		\textit{$C_R$} & coefficient of restitution & unitless \\
		\textit{m} & mass & $kg$ \\
		\textit{L} & length & $m$ \\
		\textit{V} & volume & $m^3$ \\
		\textit{k} & force constant & $N/m$ \\
		\textit{$\rho$} & density & $kg/m^3$ \\
		\textit{$\phi$} & orientation & $rad$ \\
		\textit{$\omega$} & angular velocity & $rad/s$ \\
		\textit{$\alpha$} & angular acceleration & $rad/s^2$ \\
		\textit{$\zeta$} & damping coefficient & unitless \\
		\hline 
	\end{tabular}
	\end{center}
	
\end{frame}

%%%%%%%%%%%%%%%%%%%%%%%%%%%%%%%%%%%%%%

\section[Outputs]{Software Outputs}

%%%%%%%%%%%%%%%%%%%%%%%%%%%%%%%%%%%%%%

\begin{frame}

\frametitle{Software Outputs}

\begin{itemize}
	\item From the input data, Chipmunk will determine the kinematics of rigid bodies undergoing collision/acted upon by a force over a period of time
	\item Outputs information such as velocity, force, angular velocity, etc. (see examples below)
\end{itemize}

\begin{center}
	\begin{tabular}{|| c | c | c ||}
		\hline
		Var & Quantity & Unit \\
		\hline\hline
		\textit{p} & position & $m$ \\ %I'm not so sure about the units
		\textit{v} & velocity & $m/s$ \\
		\textit{F} & force & $N$ \\
		\textit{$\phi$} & orientation & $rad$ \\
		\textit{$\omega$} & angular velocity & $rad/s$ \\
		\hline
	\end{tabular}
\end{center}

\end{frame}

%%%%%%%%%%%%%%%%%%%%%%%%%%%%%%%%%%%%%%

\section[Design]{System Architecture and Design}

%%%%%%%%%%%%%%%%%%%%%%%%%%%%%%%%%%%%%%

\begin{frame}

\frametitle{System Architecture and Design}

\begin{itemize}
	\item Modularization - the library is divided (decomposed) into \textit{modules} based on information hiding and design for change
	\item Each module has a specific role - usually contains a data structure implementation and a collection of related functions
	\item Module hierarchy - 3 top-level modules:
		\begin{itemize}
			\item Hardware-hiding
			\item Behavior-hiding
			\item Software decision
		\end{itemize}
\end{itemize}

\end{frame}

%%%%%%%%%%%%%%%%%%%%%%%%%%%%%%%%%%%%%%

\begin{frame}
	
\frametitle{System Architecture and Design}

Modules are divided based on their 'secrets' (underlying algorithms and data structures in the implementation):

\begin{itemize}
	\item Abstraction - users don't need to know implementation details
	\item Maintainability - makes it easier to maintain, debug and update implementation code (likely changes)
\end{itemize}

\end{frame}

%%%%%%%%%%%%%%%%%%%%%%%%%%%%%%%%%%%%%%

\section[Installation]{Installation}

%%%%%%%%%%%%%%%%%%%%%%%%%%%%%%%%%%%%%%

\begin{frame}

\frametitle{Installation}

The release version provides three different build scripts:

\begin{itemize}
	\item XCode (for Mac/iPhones)
	\item MSVC (Microsoft Visual Studio C++)
	\item CMake build script
\end{itemize}

\end{frame}

%%%%%%%%%%%%%%%%%%%%%%%%%%%%%%%%%%%%%%

\section[Tests]{Test Cases}

%%%%%%%%%%%%%%%%%%%%%%%%%%%%%%%%%%%%%%

\begin{frame}
	
\frametitle{Test Cases}

\underline{System Testing} \\
\vspace{5pt}
Example: \textbf{Projectile motion} \\
\vspace{5pt}
Input: 
\begin{align*}
p_i & = \begin{bmatrix}
50 \\
50
\end{bmatrix}
\hspace{10pt}
v_i = \begin{bmatrix}
5 \\
4
\end{bmatrix}
\hspace{10pt}
a = \begin{bmatrix}
0 \\
-9.8
\end{bmatrix} \\
\vspace{10pt}
t & = 0.5, 1.0, 1.5, 2.0 \,s
\end{align*}
Expected output: 
\begin{align*}
p_f = \begin{bmatrix}
50 + 5t \\
50 + 4t - 4.9t^2
\end{bmatrix}
\hspace{10pt}
v_f = \begin{bmatrix}
5 \\
5 - 9.8t
\end{bmatrix}
\hspace{10pt}
a = \begin{bmatrix}
0 \\
-9.8
\end{bmatrix}
\end{align*}

\end{frame}

%%%%%%%%%%%%%%%%%%%%%%%%%%%%%%%%%%%%%%

\begin{frame}

\frametitle{Test Cases}

\underline{Unit Testing} \\
\vspace{5pt}
Example: \textbf{CPBody Module} 
\begin{figure}
\centering
\includegraphics[width=0.75\linewidth]{sample-test}
\label{fig:sample-test}
\end{figure}

\end{frame}

%%%%%%%%%%%%%%%%%%%%%%%%%%%%%%%%%%%%%%

\section[References]{References}

%%%%%%%%%%%%%%%%%%%%%%%%%%%%%%%%%%%%%%

\begin{frame}{References}
	\begin{itemize}
		\item S. Lembcke. \textit{Chipmunk Game Dynamics Manual}[Web]. Available at https://chipmunk-physics.net/release/ChipmunkLatest-Docs/
	\end{itemize}
	
	The following resources are all available at the se4sc repository:
	
	\begin{itemize}
		\item A. Halliwushka. (2015, August 4.). \textit{Module Guide for Open Source Game Physics Library}[PDF]. 
		\item A. Halliwushka. (2016, January 19.). \textit{Software Requirements Specification for Chipmunk2D}[PDF].
		\item A. Halliwushka. (2015, June 14.). \textit{Verification and Validation Plan for Open Source Game Physics Library}[PDF].
		\item S. Smith. (2015, June 4.). \textit{Modular Design}[Presentation].
	\end{itemize}
	
\end{frame}

\end{document}