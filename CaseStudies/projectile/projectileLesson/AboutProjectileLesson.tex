\documentclass[12pt]{article}

\usepackage{graphicx}
\usepackage{paralist}
\usepackage{amsfonts}
\usepackage{amsmath}
\usepackage{hhline}
\usepackage{booktabs}
\usepackage{multirow}
\usepackage{multicol}
\usepackage{url}
\usepackage{hyperref}

\oddsidemargin -10mm
\evensidemargin -10mm
\textwidth 160mm
\textheight 200mm
\renewcommand\baselinestretch{1.0}

\pagestyle {plain}
\pagenumbering{arabic}

%% Comments

\usepackage{color}

\newif\ifcomments\commentstrue

\ifcomments
\newcommand{\authornote}[3]{\textcolor{#1}{[#3 ---#2]}}
\newcommand{\todo}[1]{\textcolor{red}{[TODO: #1]}}
\else
\newcommand{\authornote}[3]{}
\newcommand{\todo}[1]{}
\fi

\newcommand{\wss}[1]{\authornote{blue}{SS}{#1}}

\title{Discussion of Projectile Lesson: What and Why}
\author{Spencer Smith}

\begin {document}

\maketitle

I've adapted Section 12.6 (Motion of a Projectile) from the classic Hibbler text
``Engineering Mechanics Dynamnics, 10th edition.''  To take into account that
Section 12.6 builds on the previous sections, I've added an initial section that
simply provides a summary of the equations that were previously derived.  The
derivation of these equations is fairly similar to how they are derived in the
current Projectile example in Drasil.  I've made an effort to change the symbols
and notation to match what is currently in Drasil.

Comments have been incorporated into the text using square brackets [].

I've tried to write the parts using the simplest possible markdown.  Equations
are added using LaTeX.  To get cross-references to equations to work I had to
add the LaTeX extensions to my local installation of Jupyter.  Maybe there is a
better way to do cross-references?

For the code part, I tried to leave the variabilities in code (as opposed to in
the markdown), so that they can be changed easily.  The figure is ``hardcoded''
with specific values for the variable inputs.

\section {Discussion}

The Projectile example should be updated to use the modern approach to textbook
writing.  The knowledge should be classified and divided so that the generic
patterns emerge.  By classifying the knowledge, we'll be able to generalize to
other lessons.  This could also help us find the mapping between knowledge in
the SRS and knowledge in a typical lesson.

Information on templates for textbook chapters can be found on-line:

\begin{itemize}
\item
  \href{https://canvas.umn.edu/courses/106630/pages/developing-a-textbook-structure?module_item_id=1306060}
  {Developing a Textbook Structure}
\item \href{http://edutechwiki.unige.ch/en/Textbook_writing_tutorial} {Textbook
    Writing Tutorial}
\item
  \href{https://open.ubc.ca/open-publishing-guide/phase-4/textbook-design-rules/}
  {Textbook Design Rules}
\item
  \href{http://edutechwiki.unige.ch/en/Textbook_genres_and_examples#E-Learning_and_the_Science_of_Instruction}
  {Textbook Genres and Examples}
\end{itemize}

The elements of a textbook chapter have been categorized by the above resources.
The presentation in the resources provide an overview of the ``program family''
of textbooks.  The commonalities and variabilities are:

\begin{itemize}
\item Openers: Express ``subject, theme, aims, topics, and organization of a
  chapter [... readers should] know at the outset what they are reading and why
  or to what end'' (Lepionka 2003:117). E.g. if you follow Gagné's nine events
  of instruction then you should include something to motivate and gain
  attention (step 1), something to help the frame and organize (step 2) and
  something to recall prior knowledge (step 3).
  \begin{itemize}
  \item overviews (previews)
  \item introductions
  \item outlines (text, bullets or graphics)
  \item focus questions (knowledge and comprehension questions)
  \item learning goals / objectives / outcomes / competences / skills
  \item A case problem
  \item In addition one may use the ``special features'' used inside chapters,
    e.g. vignettes, photos, quotations, ...
  \end{itemize}
\item Closers: Give students opportunities to review, reinforce, or extend their
  learning, i.e.\ help with transfer of learning (Lepionka 2003:118)
  \begin{itemize}
  \item conclusions and summaries (may include diagrams)
  \item list of definitions
  \item reference boxes (e.g. computer instructions)
  \item review questions
  \item self-assessment (usually simple quizzes)
  \item small exercises
  \item substantial exercises and problem cases
  \item fill-in tables (for "learning-in-action" books) to prepare a real world
    task
  \item ideas for projects (academic or real world)
  \item bibliographies and links (that can be annotated)
  \end{itemize}
\item Integrated Pedagogical Devices: These elements aid the learning process in
  several ways, e.g.\ by giving advice on how to understand / interpret or
  navigate, by engaging the learner in some reflection, by pointing out
  important elements, or to summarize key elements treated in previous text.
  \begin{itemize}
  \item Emphasis (bold face) of words
  \item Marginalia that summarize paragraphs
  \item Lists that highlight main points
  \item Summary tables and graphics
  \item Cross-references that link backwards (or sometimes forwards) to important
    concepts
  \item Markers to identify embedded subjects (e.g. an ``external'' term used
    and that needs explanation)
  \item Study and review questions
  \item Pedagogical illustrations (concepts rendered graphically)
  \item Tips (to insure that the learner doesn't get caught in misconceptions or
    procedural errors)
  \item Reminders (e.g.\ make sure that something that was previously introduced
    is remembered)
  \end{itemize}
\item Interior Feature Strands: ``Intext features, whether boxes or portions of
  text set off through design, function pedagogically to attract attention;
  arouse curiosity; increase motivation to read; stimulate critical thinking; and
  provide opportunities for reflection, application, or problem solving''
  (Lepionka, 2003: 118).
  \begin{itemize}
  \item Case studies
  \item Problem descriptions
  \item Debates and reflections
  \item Profiles (case descriptions)
  \item Primary sources and data
  \item Models
  \end{itemize}
\end{itemize}

To the last category of elements we could likely add ``historical notes, or
autobiographies of important figures''.  However, this may be covered under
``Profiles.''

\end {document}
