


% This LaTeX was auto-generated from an M-file by MATLAB.
% To make changes, update the M-file and republish this document.






  
    

\section{Kinematic Admissibility Tester} \label{sec:KinTests}
Testing results obtained from the KinAdm.m program, a
module in the Slope Stability Analysis program. The tester will test the
algorithm to ensure it can correctly identify when a slip surface fails
the 6 failure criterion.

\subsection{Testing}
The module will now be tested for the different failure criterion. A
kinematically inadmissible test will return a value pass=0, and a failure
code detailing the cause of the failure. All other values should return a
value pass=1. All failure criteria will pass boundary cases. The failure
criterion are tested individually with a simple slope surface, and slip
surfaces designed to test the failure. Each failure test will give a slip
designed to pass, a boundary case, and one designed to fail.

\subsubsection*{Failure (i)}
The first criteria of a kinematically admissible surface is that the
x-ordinates of the input slip surface vertexes do not decrease when
reading the list of vertexes from beginning to end. (A) is a test with
constantly increasing x-ordinates. (B) is a test case with equivalent
x-ordinates. (C) is a case with decreasing x-ordinates. Results follow:

        
\color{lightgray} \begin{verbatim}
Failure (i):
(A): pass=1
(B): pass=1
(C): pass=0 Failure Code1 - Non monotonic x
\end{verbatim} \color{black}
    

\subsubsection*{End Adjustments}
The second criteria of a kinematically admissible surface is that the
start and end vertexes of the slip surface match the y-ordinate of the
uppermost stratigraphic layer at the specified x-ordinate.
The module will not fail the slip surface, but will adjust the y value of
the end vertexes. (A) is a test with vertexes above the uppermost
stratigraphic layer. (B) is a test with vertexes on the uppermost
stratigraphic layer. (C) is a test with vertexes below the uppermost
stratigraphic layer. Vertice adjustment refers to the y values of the
vertexes.

        
\color{lightgray} \begin{verbatim}
End Adjustments:
(A): pass=1, Start vertice adjustment,21->20 End vertice adjustment,13->12
(B): pass=1, Start vertice adjustment,20->20 End vertice adjustment,12->12
(C): pass=1, Start vertice adjustment,19->20 End vertice adjustment,11->12
\end{verbatim} \color{black}
    

\subsubsection*{Failure (ii)}
The vertexes of the slip surface must be within the specified x-ordinate
range of the uppermost stratigraphic layer. (A) is a test case with
vertexes that stay within the uppermost stratigraphic layers range. (B)
is a test with a vertice x-ordinate that goes below the minimum range of
the uppermost stratigraphic layer. (C) is a test with a vertice
x-ordinate that goes above the maximum range of the uppermost
stratigraphic layer.

        
\color{lightgray} \begin{verbatim}
Failure (ii):
(A): pass=1
(B): pass=0 Failure Code2 - Vertex outside x range
(C): pass=0 Failure Code2 - Vertex outside x range
\end{verbatim} \color{black}
    

\subsubsection*{Failure (iii)}
The non end vertexes of the slip surface must be below the uppermost
stratigraphic layer. End vertexes will be moved onto the uppermost
stratigraphic layer, and therefore don't interact with this case. This
failure case checks only vertexes, line segments above the uppermost
strat are checked in failure (iv). (A) is a test with vertexes below the
uppermost stratigraphic layer. (B) is a test with vertexes on the
uppermost stratigraphic layer. (C) is a test with the first interior
vertice above the uppermost stratigraphic layer. (D) is a test with the
last interior vertice above the uppermost stratigraphic layer of the slip
surface.

        
\color{lightgray} \begin{verbatim}
Failure (iii):
(A): pass=1
(B): pass=1
(C): pass=0 Failure Code3 - Vertex above surface
(D): pass=0 Failure Code3 - Vertex above surface
\end{verbatim} \color{black}
    

\subsubsection*{Failure (iv)}
Line segments between vertexes of the slip surface cannot go above the
uppermost stratigraphic layer. (A) is a test with all line segments below
the uppermost stratigraphic surface. (B) is a test case with a line
segment on the uppermost stratigraphic layer. (C) is a test case with a
line segment below the uppermost stratigraphic layer. the test is
performed on the three interior line segments of the uppermost
stratigraphic layer, going from slip entrance to exit as Strat Line
Segments 1,2,3.

        
\color{lightgray} \begin{verbatim}
Failure (iv) Strat Line Segment 1:
(A): pass=1
(B): pass=1
(C): pass=0 Failure Code4 - Surface Intersection

Failure (iv) Strat Line Segment 2:
(A): pass=1
(B): pass=1
(C): pass=0 Failure Code4 - Surface Intersection

Failure (iv) Strat Line Segment 3:
(A): pass=1
(B): pass=1
(C): pass=0 Failure Code4 - Surface Intersection
\end{verbatim} \color{black}
    

\subsubsection*{Failure (v)}
The slip surface must be concave upwards. The slope of line segments
between vertexes of the slip surface must go from a large magnitude
negative number towards a large magnitude positive number when connecting
vertexes from slip entrance to exit. (A) test a case where slip surface
slopes are increasing. (B) tests a case where slip surface
slopes are constant. (C) tests a case where slip slopes experience a
decrease.

        
\color{lightgray} \begin{verbatim}
Failure (v):
(A): pass=1
(B): pass=1
(C): pass=0 Failure Code5 - Concave Down, mcur=-1.02 mprv=-0.98
\end{verbatim} \color{black}
    

\subsubsection*{Failure (vi)}
Slip surfaces cannot have angles less than 110 degrees (1.9199 rads)
between adjacent line segments connecting vertexes. (A) tests a case with
a greater than 110 degree slope. (B) tests a case with an exactly 110
degree slope. (C) tests a case with a less than 110 degree slope.

        
\color{lightgray} \begin{verbatim}
Failure (vi):
(A): pass=0 Failure Code6 - Sharp angle, Theta=1.9078
(B): pass=0 Failure Code6 - Sharp angle, Theta=1.9003
(C): pass=0 Failure Code6 - Sharp angle, Theta=1.8929
\end{verbatim} \color{black}
    

The results seen differ slightly from whats expected, with all cases
failing reporting angles just below the 1.9199 rad cut off, despite case
(A) being greater than this angle, and (B) being approximately
equivalent, based on geometric analysis. The minor error in angle
calculation likely comes from a \textit{pi} rounding error. Other than
this small error, all other tests were successful.
