\documentclass[12pt]{article}

\usepackage{amsmath}
\usepackage{fullpage}
%% Comments

\usepackage{color}

\newif\ifcomments\commentstrue

\ifcomments
\newcommand{\authornote}[3]{\textcolor{#1}{[#3 ---#2]}}
\newcommand{\todo}[1]{\textcolor{red}{[TODO: #1]}}
\else
\newcommand{\authornote}[3]{}
\newcommand{\todo}[1]{}
\fi

\newcommand{\wss}[1]{\authornote{blue}{SS}{#1}}
\newcommand{\an}[1]{\authornote{magenta}{Author}{#1}}


\begin{document}
The Matlab function "fmincon" can be used to solve for the critical slip 
surface as a generic optimization problem. fmincon takes as an argument an 
objective function, which itself must take as an argument a solution vector 
$x$. For this case, $x$ would be as follows:

\begin{equation*}
x = \begin{bmatrix}
x_0 \\
y_0 \\
x_1 \\
y_1 \\
... \\
x_n \\
y_n \\
\end{bmatrix}
\end{equation*}

where $n$ is the number of slices. This would be known based on the algorithm 
for slicing. 

The Kinematic Admissibility module checks for 6 constraints on a slip surface, 
which can be translated to mathematical linear and non-linear constraints as 
follows. The index $i$ in the following equations represents the index of the 
solution vector.

\begin{enumerate}
	\item Increasing $x$-ordinates
	\begin{equation*}
	x(i) - x(i + 2) \leq 0
	\end{equation*}
	
	\item Edge vertices of slip surface lie on slope surface
	\begin{equation*}
	y_0 = y_{sl} \text{ at } x_0
	\end{equation*}
	\begin{equation*}
	y_n = y_{sl} \text{ at } x _n
	\end{equation*}
	
	\item Non-edge vertices of slip surface are within slope
	\begin{equation*}
	y_i < y_{sl} \text{ for } 0 < i < n
	\end{equation*}
	
	\item Slip surface does not intersect slope surface except at edges. I 
	believe this is covered already by the above constraints.
	
	\item Slip surface is concave up
	\begin{equation*}
	\frac{x(i+3) - x(i+1)}{x(i+2) - x(i)} - \frac{x(i+5) - x(i+3)}{x(i+4) - 
	x(i+2)} \leq 0
	\end{equation*}
	
	\item No angles less than 110 degrees
	\begin{equation*}
	\arccos\left(\frac{a + b - c}{2*a*b}\right) - 1.9199 \leq 0
	\end{equation*}
	where
	\begin{equation*}
	a = (x(i+2) - x(i))^2 + (x(i+3) - x(i+1))^2
	\end{equation*}
	\begin{equation*}
	b = (x(i+4) - x(i+2))^2 + (x(i+5) - x(i+3))^2
	\end{equation*}
	\begin{equation*}
	c = (x(i+4) - x(i))^2 + (x(i+5) - x(i+1))^2
	\end{equation*}
\end{enumerate}

Additional constraints that may need to be introduced include constraints to 
ensure even-width slices (if desired) and constraints to ensure that the $x_0$, 
$x_n$, and all $y$ values are within the user-supplied bounds (input to SSP).

Some other notes and concerns:

\begin{itemize}
	\item An initial guess must be supplied. We need an algorithm for 
	determining this initial guess based on the slope geometry (initial guess 
	could be a circular slip surface)
	\item Objective function and function for non-linear constraints each must 
	only accept 1 argument (the solution vector $x$). However, they will also 
	need input parameters to SSP. Thus, input parameters must be global 
	variables or the objective function should be partially evaluated such that 
	only the slip surface coordinate variables remain (need to look into how 
	this can be done in Matlab)
	\item Would be interesting if we could find out why no one approaches slope 
	stability this way, opting for genetic algorithms or other search methods 
	instead.
\end{itemize}
\end{document}