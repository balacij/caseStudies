\documentclass[12pt, titlepage]{article}

\usepackage{tabularx}
\usepackage{longtable}
\usepackage{comment}
\usepackage{amsmath}
\usepackage{amssymb}
\usepackage{booktabs}
\usepackage{xr}
\usepackage{siunitx}
\usepackage{caption}
\usepackage{graphicx}
\usepackage{hyperref}
\usepackage[numbib,nottoc]{tocbibind}
\hypersetup{
	colorlinks,
	citecolor=green,
	filecolor=black,
	linkcolor=red,
	urlcolor=blue
}
\usepackage[round]{natbib}
\usepackage{enumitem}

\externaldocument[SRS-]{../../SRS/SRS}
\newcommand{\rref}[1]{R\ref{#1}}
\newcommand{\nfrref}[1]{NFR\ref{#1}}

\externaldocument[MG-]{../../Design/MG/MG}
\newcommand{\mref}[1]{M\ref{#1}}

\externaldocument[SVnV-]{../SystVnVPlan/SystVnVPlan}
\newcommand{\tcref}[1]{TC\ref{#1}}

\newcounter{utestnum} %Assumption Number
\newcommand{\utcthetestnum}{TC\theutestnum}
\newcommand{\utcref}[1]{TC\ref{#1}}

%% Comments

\usepackage{color}

\newif\ifcomments\commentstrue

\ifcomments
\newcommand{\authornote}[3]{\textcolor{#1}{[#3 ---#2]}}
\newcommand{\todo}[1]{\textcolor{red}{[TODO: #1]}}
\else
\newcommand{\authornote}[3]{}
\newcommand{\todo}[1]{}
\fi

\newcommand{\wss}[1]{\authornote{blue}{SS}{#1}}
\newcommand{\an}[1]{\authornote{magenta}{Author}{#1}}

%% Common Parts

\newcommand{\progname}{SSP} % PUT YOUR PROGRAM NAME HERE %Every program
                                % should have a name


\begin{document}

\title{Slope Stability Analysis: Unit Verification and Validation Plan for 
\progname{}} 
\author{Brooks MacLachlan}
\date{\today}
	
\maketitle

\pagenumbering{roman}

\section{Revision History}

\begin{tabularx}{\textwidth}{p{3cm}p{2cm}X}
\toprule {\bf Date} & {\bf Version} & {\bf Notes}\\
\midrule
2018/11/26 & 1.0 & Initial template fill-ins\\
\bottomrule
\end{tabularx}

~\newpage

\tableofcontents

\listoftables

\wss{Do not include if not relevant}

\listoffigures

\wss{Do not include if not relevant}

\newpage

\section{Symbols, Abbreviations and Acronyms}

The symbols, abbreviations, and acronyms used in this document include those 
defined in the table below, as well as any defined in the tables found in 
Section \ref{SRS-sec_RefMat} of the Software Requirements Specification (SRS) 
document.
\newline

\renewcommand{\arraystretch}{1.2}
\begin{tabular}{l l} 
	\toprule		
	\textbf{symbol} & \textbf{description}\\
	\midrule 
	$j$ & index representing a single coordinate\\
	MIS & Module Interface Specification\\
	MG & Module Guide\\
	TC & Test Case\\
	VnV & Verification and Validation\\
	\bottomrule
\end{tabular}\\

\newpage

\pagenumbering{arabic}

This document provides the unit Verification and Validation (VnV) plan for the 
software. General information related to the system under test is given in 
Section~\ref{sec_Info}. Section~\ref{sec_Plan} outlines at a high level the 
plan for verifying and validating the software. Section~\ref{sec_Tests} gives 
more detail about the specific tests that will be used to verify each module.

\section{General Information} \label{sec_Info}

\subsection{Purpose}

\noindent The software being tested is the Slope Stability analysis Program 
(\progname{}). Based on user-defined slope geometry and material properties, 
\progname{} determines the critical slip surface of the given slope, the 
corresponding factor of safety, and interslice normal and shear forces along 
the critical slip surface.

\noindent The purpose of the unit verification and validation activities is to 
confirm that every module of \progname{} performs its expected actions 
correctly. The tests described in this document cannot definitively prove 
correctness, but they can build confidence by verifying that the software is 
correct for the cases covered by tests.

\subsection{Scope}

\noindent \mref{MG-mHH} will not be unit tested as it is implemented by the 
operating system of the hardware on which \progname{} is running, and is 
assumed to work correctly. \mref{MG-mArrayOps}, \mref{MG-mRandNum}, and 
\mref{MG-mPlot} 
will also not be unit tested as they are all implemented by MatLab. 
Verification of the non-functional requirements is not included in the unit 
verification plan because they are sufficiently covered by the system tests 
outlined in the 
\href{https://github.com/smiths/caseStudies/blob/master/CaseStudies/ssp/docs/VnVPlan/SystVnVPlan/SystVnVPlan.pdf}
{System VnV Plan document}.

\section{Plan} \label{sec_Plan}
	
\subsection{Verification and Validation Team}

\noindent Brooks MacLachlan is responsible for the unit verification and 
validation of \progname{}, though input from various students and the 
professor, Dr.~Spencer Smith, of CAS 741 will also contribute.

\subsection{Automated Testing and Verification Tools}

\noindent MatLab's built-in unit testing framework will be used to 
automatically run the unit tests and display the results.

\bmac{Is it inherently better to use a unit testing framework if you already 
have a lot of tests written without one?}

\subsection{Non-Testing Based Verification}

Not applicable for \progname{}.

\section{Unit Test Description} \label{sec_Tests}

\noindent Test cases have been selected to verify that each module conforms to 
the specification for the module described in the 
\href{https://github.com/smiths/caseStudies/blob/master/CaseStudies/ssp/docs/Design/MIS/MIS.pdf}
{Module Interface Specification (MIS) document}. Where the MIS included 
conditional rules, at least one test case covers each branch of the conditional 
rule. Test cases are minimal, meaning that each test case verifies only one 
value. If the MIS for a module includes several results, there is a test case 
for each result, even if they all cover the same branch of a conditional. 
Throughout this section, if a test is verifying equality between two numbers, a 
relative tolerance for difference between the actual and expected values will 
be allowed. The tolerance will be described after running the tests. 

\subsection{Tests for Functional Requirements}

\subsubsection{Control Module}

\bmac{Should control module be out of scope? All it does is call other modules. 
Maybe I could unit test that the expected functions were called, but is that 
valuable?}

\subsubsection{Input Module}

As described in the MIS, the Input module is expected to read in many user 
inputs from a file. For each value contained in the file, there is a 
corresponding test case verifying that the value was properly read into the 
data structure containing the input parameters. In cases where the user input 
may take different forms, such as the input for when a water table exists and 
the input for when a water table does not exists, each potential form of input 
is covered by at least one test case. The input module is also responsible for 
verifying the input, so for each possible violation of an input constraint, 
there is a corresponding test case verifying that the correct exception was 
thrown.

\noindent The values in Table~\ref{Inputs} will be used as input for 
many of the test cases described throughout this section. These values were 
taken from the User's Guide for this project by \cite{UserGuide}. Individual 
test cases will reference the table as input but specify new values for any 
input parameter that should have a different value than specified by the table.

\begin{table}[!h]
	\renewcommand{\arraystretch}{1.5}
	\begin{tabularx}{1.0\textwidth}{p{7cm} l X}
		\toprule \textbf{Input} &
		\textbf{Unit} & \textbf{Value}\\ \midrule
		$\{\left(x_\text{us},y_\text{us}\right)\}$ & $\text{m}$ & \{(0, 25), 
		(20, 25), (30, 20), (40, 15), (70, 15)\}\\
		$\{\left(x_\text{wt},y_\text{wt}\right)\}$ & $\text{m}$ & \{(0, 22), 
		(10.87, 21.28), (21.14, 19.68), (31.21, 17.17), (38.69, 14.56), (40, 
		14), (70, 14)\}\\
		${x_\text{slip}^\text{minEtr}}$ & $\text{m}$ & 10\\
		${x_\text{slip}^\text{maxEtr}}$ & $\text{m}$ & 24\\
		${x_\text{slip}^\text{minExt}}$ & $\text{m}$ & 34\\
		${x_\text{slip}^\text{maxExt}}$ & $\text{m}$ & 53\\
		${y_\text{slip}^\text{min}}$ & $\text{m}$ & 5\\
		${y_\text{slip}^\text{max}}$ & $\text{m}$ & 26\\
		$c'$ & $\si{\pascal}$ & 5000 \\
		$\varphi'$ & \si{\degree} & 20\\
		$\gamma$ & $\si{\newton\per\meter\cubed}$ & 15000 \\
		$\gamma_{\text{Sat}}$ & $\si{\newton\per\meter\cubed}$ & 15000 \\
		$\gamma_{\text{w}}$ & $\si{\newton\per\meter\cubed}$ & 9800 \\
		\textit{const\_f} & N/A & 0\\ 
		\bottomrule
	\end{tabularx}
	\caption{Input to be used for test cases}
	\label{Inputs}
\end{table}

\paragraph{Valid User Input}

~\newline \noindent The test cases described in Table~\ref{InputTests} 
verify that each user input is correctly read. These test cases are identical 
to each other with the exception of the expected output which they assert. The 
input for each is a file containing the inputs specified in 
Table~\ref{ExValidInputs}.\bmac{Should the input be more specific and give the 
actual name of a file, or is this okay?} The type 
of these test cases is automatic. The initial state for each is a new session. 
The expected output for each is given in Table~\ref{InputTests}. The expected 
output is derived based on the given inputs. The tests will be performed as 
automated tests on a unit testing framework.

\begin{longtable}{  l  p{4cm}  p{6cm}  }
	\hline
	\textbf{Test Case} & \textbf{Test Name} & \textbf{Expected Output} \\
	\hline
	TC\refstepcounter{utestnum}\theutestnum  \label{TC_InputSlope} & 
	test-input\_slope &  \textit{slope.strat} = [(0, 25), (20, 25), (30, 20), 
	(40, 25), (70, 25)]\\
	\hline 
	TC\refstepcounter{utestnum}\theutestnum  \label{TC_InputPhi} & 
	test-input\_phi &  \textit{slope.phi} = 0.34906585\\ 
	\hline
	TC\refstepcounter{utestnum}\theutestnum  \label{TC_InputCoh} & 
	test-input\_coh &  \textit{slope.coh} = 5000\\ 
	\hline
	TC\refstepcounter{utestnum}\theutestnum  \label{TC_InputGam} & 
	test-input\_gam &  \textit{slope.gam} = 15000\\ 
	\hline
	TC\refstepcounter{utestnum}\theutestnum  \label{TC_InputGams} & 
	test-input\_gams &  \textit{slope.gams} = 15000\\ 
	\hline
	TC\refstepcounter{utestnum}\theutestnum  \label{TC_InputWT} & 
	test-input\_piez &  \textit{piez.piez} = [(0, 22), 
	(10.87, 21.28), (21.14, 19.68), (31.21, 17.17), (38.69, 14.56), (40, 
	14), (70, 14)]\\
	\hline 
	TC\refstepcounter{utestnum}\theutestnum  \label{TC_InputGamw} & 
	test-input\_gamw &  \textit{piez.gamw} = 9800\\ 
	\hline
	TC\refstepcounter{utestnum}\theutestnum  \label{TC_InputXextMin} & 
	test-input\_xExtMin &  \textit{search.Xext}[0] = 34\\ 
	\hline
	TC\refstepcounter{utestnum}\theutestnum  \label{TC_InputXextMax} & 
	test-input\_xExtMax &  \textit{search.Xext}[1] = 53\\ 
	\hline
	TC\refstepcounter{utestnum}\theutestnum  \label{TC_InputXetrMin} & 
	test-input\_xEtrMin &  \textit{search.Xetr}[0] = 10\\ 
	\hline
	TC\refstepcounter{utestnum}\theutestnum  \label{TC_InputXetrMax} & 
	test-input\_xEtrMax &  \textit{search.Xetr}[1] = 24\\ 
	\hline
	TC\refstepcounter{utestnum}\theutestnum  \label{TC_InputYlimMin} & 
	test-input\_yLimMin &  \textit{search.Ylim}[0] = 5\\ 
	\hline
	TC\refstepcounter{utestnum}\theutestnum  \label{TC_InputYlimMax} & 
	test-input\_yLimMax &  \textit{search.Ylim}[1] = 26\\ 
	\hline
	TC\refstepcounter{utestnum}\theutestnum  \label{TC_InputLtoR} & 
	test-input\_ltor &  \textit{soln.ltor} = 1\\ 
	\hline
	TC\refstepcounter{utestnum}\theutestnum  \label{TC_InputFtype} & 
	test-input\_ftype &  \textit{soln.ftype} = 0\\ 
	\hline
	TC\refstepcounter{utestnum}\theutestnum  \label{TC_InputEvenslc} & 
	test-input\_evenslc &  \textit{soln.evenslc} = 1\\ 
	\hline
	TC\refstepcounter{utestnum}\theutestnum  \label{TC_InputCncvu} & 
	test-input\_cncvu &  \textit{soln.cncvu} = 1\\ 
	\hline
	TC\refstepcounter{utestnum}\theutestnum  \label{TC_InputObtu} & 
	test-input\_obtu &  \textit{soln.obtu} = 1\\ 
	\hline
	\caption{Input Test Cases}
	\label{InputTests}
\end{longtable}

\bmac{Can I use MIS variable names for expected output?}

\begin{enumerate}[label=TC\arabic*:,ref={\arabic*}]

\item [TC\refstepcounter{utestnum}\theutestnum: \label{TC_InputRtoL}] 
test-input\_rtol

Type: Automatic
					
Initial State: New session
					
Input: As described in Table~\ref{Inputs}, except with slope coordinates 
$\{\left(x_\text{us},y_\text{us}\right)\}$ increasing as $x$ increases, as 
follows: \{(0, 
15), (30, 15), (40, 20), (50, 25), (70, 25)\}.
					
Output: \textit{soln.ltor} = 0.

Test Case Derivation: Based on the given slope stratigraphy, \progname{} should 
detect that the slope elevation is increasing as $x$ increases, and set 
\textit{soln.ltor} accordingly.

How test will be performed: Automated test on unit testing framework.
					
\item [TC\refstepcounter{utestnum}\theutestnum: \label{TC_InputNoWTpiez}] 
test-input\_noWTpiez

Type: Automatic

Initial State: New session

Input: As described in Table~\ref{Inputs}, except with no water table.

Output: \textit{piez.piez} = [].

Test Case Derivation: If the input includes no water table vertices, the 
\textit{piez.piez} variable should be the empty sequence.

How test will be performed: Automated test on unit testing framework.

\item [TC\refstepcounter{utestnum}\theutestnum: \label{TC_InputNoWTgamw}] 
test-input\_noWTgamw

Type: Automatic

Initial State: New session

Input: As described in Table~\ref{Inputs}, except with no water table.

Output: \textit{piez.gamw} = 0.

Test Case Derivation: If the input includes no water table vertices, the 
\textit{piez.gamw} variable should be 0.

How test will be performed: Automated test on unit testing framework.
    
\end{enumerate}

\paragraph{Invalid User Input}
~\newline \noindent See \tcref{SVnV-TC_InvalidSlopeDecToInc} - 
\tcref{SVnV-TC_InvalidUnitWtWaterNegative} in the System VnV Plan document.

\bmac{Is this okay? The tests described in the SystVnVPlan are really unit 
tests}

\subsubsection{Module 2}

...

\subsection{Tests for Nonfunctional Requirements}

Not applicable for any of the modules of \progname{}.

\subsection{Traceability Between Test Cases and Modules}

\wss{Provide evidence that all of the modules have been considered.}

\bibliographystyle{plainnat}

\bibliography{../../../refs/References}

\newpage

\section{Appendix}

\wss{This is where you can place additional information, as appropriate}

\subsection{Symbolic Parameters}

\wss{The definition of the test cases may call for SYMBOLIC\_CONSTANTS.
Their values are defined in this section for easy maintenance.}

\end{document}