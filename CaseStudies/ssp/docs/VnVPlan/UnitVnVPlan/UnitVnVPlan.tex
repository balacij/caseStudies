\documentclass[12pt, titlepage]{article}

\usepackage{tabularx}
\usepackage{longtable}
\usepackage{comment}
\usepackage{amsmath}
\usepackage{amssymb}
\usepackage{booktabs}
\usepackage{xr}
\usepackage{siunitx}
\usepackage{caption}
\usepackage{graphicx}
\usepackage{afterpage}
\usepackage{pdflscape}
\usepackage{hyperref}
\usepackage[numbib,nottoc]{tocbibind}
\hypersetup{
	colorlinks,
	citecolor=green,
	filecolor=black,
	linkcolor=red,
	urlcolor=blue
}
\usepackage[round]{natbib}
\usepackage{enumitem}

\externaldocument[SRS-]{../../SRS/SRS}
\newcommand{\rref}[1]{R\ref{#1}}
\newcommand{\nfrref}[1]{NFR\ref{#1}}

\externaldocument[MG-]{../../Design/MG/MG}
\newcommand{\mref}[1]{M\ref{#1}}

\externaldocument[SVnV-]{../SystVnVPlan/SystVnVPlan}
\newcommand{\tcref}[1]{TC\ref{#1}}

\newcounter{utestnum} %Assumption Number
\newcommand{\utcthetestnum}{TC\theutestnum}
\newcommand{\utcref}[1]{TC\ref{#1}}

%% Comments

\usepackage{color}

\newif\ifcomments\commentstrue

\ifcomments
\newcommand{\authornote}[3]{\textcolor{#1}{[#3 ---#2]}}
\newcommand{\todo}[1]{\textcolor{red}{[TODO: #1]}}
\else
\newcommand{\authornote}[3]{}
\newcommand{\todo}[1]{}
\fi

\newcommand{\wss}[1]{\authornote{blue}{SS}{#1}}
\newcommand{\an}[1]{\authornote{magenta}{Author}{#1}}

%% Common Parts

\newcommand{\progname}{SSP} % PUT YOUR PROGRAM NAME HERE %Every program
                                % should have a name


\begin{document}

\title{Slope Stability Analysis: Unit Verification and Validation Plan for 
\progname{}} 
\author{Brooks MacLachlan}
\date{\today}
	
\maketitle

\pagenumbering{roman}

\section{Revision History}

\begin{tabularx}{\textwidth}{p{3cm}p{2cm}X}
\toprule {\bf Date} & {\bf Version} & {\bf Notes}\\
\midrule
11/26/2018 & 1.0 & Initial template fill-ins\\
11/29/2018 & 1.1 & Updates based on presentation feedback\\
12/03/2018 & 1.2 & Added tests for remaining modules\\
\bottomrule
\end{tabularx}

~\newpage

\tableofcontents

\listoftables

\newpage

\section{Symbols, Abbreviations and Acronyms}

The symbols, abbreviations, and acronyms used in this document include those 
defined in the table below, as well as any defined in the tables found in 
Section \ref{SRS-sec_RefMat} of the Software Requirements Specification (SRS) 
document.
\newline

\renewcommand{\arraystretch}{1.2}
\begin{tabular}{l l} 
	\toprule		
	\textbf{symbol} & \textbf{description}\\
	\midrule
	MIS & Module Interface Specification\\
	MG & Module Guide\\
	TC & Test Case\\
	VnV & Verification and Validation\\
	\bottomrule
\end{tabular}\\

\newpage

\pagenumbering{arabic}

This document provides the unit Verification and Validation (VnV) plan for the 
software. General information related to the system under test is given in 
Section~\ref{sec_Info}. Section~\ref{sec_UnitPlan} outlines at a high level the 
plan for verifying and validating the software. Section~\ref{sec_Tests} gives 
more detail about the specific tests that will be used to verify each module.

\section{General Information} \label{sec_Info}

\subsection{Purpose}

\noindent The software being tested is the Slope Stability analysis Program 
(\progname{}). Based on user-defined slope geometry and material properties, 
\progname{} determines the critical slip surface of the given slope, the 
corresponding factor of safety, and interslice normal and shear forces along 
the critical slip surface.

\noindent The purpose of the unit verification and validation activities is to 
confirm that every module of \progname{} performs its expected actions 
correctly. The tests described in this document cannot definitively prove 
correctness, but they can build confidence by verifying that the software is 
correct for the cases covered by tests.

\subsection{Scope} \label{Scope}

The Hardware-Hiding module, \noindent \mref{MG-mHH}, will not be unit tested as 
it is implemented by the operating system of the hardware on which \progname{} 
is running, and is assumed to work correctly. The Sequence Data Structure, 
Random Number Generation, and Plotting modules (\mref{MG-mArrayOps}, 
\mref{MG-mRandNum}, and \mref{MG-mPlot}) will also not be unit tested as they 
are all implemented by MatLab. The Control Module, \mref{MG-mControl}, is out 
of scope for this document because it is considered to have been sufficiently 
tested by the system tests outlined in the 
\href{https://github.com/smiths/caseStudies/blob/master/CaseStudies/ssp/docs/VnVPlan/SystVnVPlan/SystVnVPlan.pdf}
{System VnV Plan document}.Verification of the non-functional requirements is 
not included in the unit verification plan because they are also sufficiently 
covered by the System VnV Plan.

\section{Plan} \label{sec_UnitPlan}
	
\subsection{Verification and Validation Team}

\noindent Brooks MacLachlan is responsible for the unit verification and 
validation of \progname{}, though input from various students and the 
professor, Dr.~Spencer Smith, of CAS 741 will also contribute.

\subsection{Automated Testing and Verification Tools}

\noindent MatLab's built-in unit testing framework will be used to 
automatically run the unit tests and display the results.

\subsection{Non-Testing Based Verification}

Not applicable for \progname{}.

\section{Unit Test Description} \label{sec_Tests}

\noindent Test cases have been selected to verify that each module conforms to 
the specification for the module described in the 
\href{https://github.com/smiths/caseStudies/blob/master/CaseStudies/ssp/docs/Design/MIS/MIS.pdf}
{Module Interface Specification (MIS) document}. Where the MIS included 
conditional rules, at least one test case covers each branch of the conditional 
rule. Test cases are minimal, meaning that each test case verifies only one 
value. If the MIS for a module includes several results, there is a test case 
for each result, even if they all cover the same branch of a conditional. 
Throughout this section, if a test is verifying equality between two numbers, a 
relative error for difference between the actual and expected values will 
be allowed. The error will be described after running the tests. 

~\newline \noindent Many of the tests refer to variables or constants as part 
of the initial state, input, or expected output. The specifications for these 
variables and constants can be found in the MIS document for this project.

~\newline \noindent The values in Table~\ref{Inputs} will be used as input for 
many of the test cases described throughout this section. These values were 
taken from the User's Guide for this project by \cite{UserGuide}. Individual 
test cases will reference the table as input but specify new values for any 
input parameter that should have a different value than specified by the table.

\begin{table}[!h]
	\renewcommand{\arraystretch}{1.5}
	\begin{tabularx}{1.0\textwidth}{p{3cm} l X}
		\toprule \textbf{Input} &
		\textbf{Unit} & \textbf{Value}\\ \midrule
		$\{\left(x_\text{slope},y_\text{slope}\right)\}$ & $\text{m}$ & \{(0, 
		25), 
		(20, 25), (30, 20), (40, 15), (70, 15)\}\\
		$\{\left(x_\text{wt},y_\text{wt}\right)\}$ & $\text{m}$ & \{(0, 22), 
		(10.87, 21.28), (21.14, 19.68), (31.21, 17.17), (38.69, 14.56), (40, 
		14), (70, 14)\}\\
		${x_\text{slip}^\text{minEtr}}$ & $\text{m}$ & 10\\
		${x_\text{slip}^\text{maxEtr}}$ & $\text{m}$ & 24\\
		${x_\text{slip}^\text{minExt}}$ & $\text{m}$ & 34\\
		${x_\text{slip}^\text{maxExt}}$ & $\text{m}$ & 53\\
		${y_\text{slip}^\text{min}}$ & $\text{m}$ & 5\\
		${y_\text{slip}^\text{max}}$ & $\text{m}$ & 26\\
		$c'$ & $\si{\pascal}$ & 5000 \\
		$\varphi'$ & \si{\degree} & 20\\
		$\gamma$ & $\si{\newton\per\meter\cubed}$ & 15000 \\
		$\gamma_{\text{Sat}}$ & $\si{\newton\per\meter\cubed}$ & 15000 \\
		$\gamma_{\text{w}}$ & $\si{\newton\per\meter\cubed}$ & 9800 \\
		\textit{const\_f} & N/A & 0\\ 
		\bottomrule
	\end{tabularx}
	\caption{Input to be used for test cases}
	\label{Inputs}
\end{table}

\subsection{Tests for Functional Requirements}

\subsubsection{Input Module}

As described in the MIS, the Input module is expected to read in many user 
inputs from a file. For each value contained in the file, there is a 
corresponding test case verifying that the value was properly read into the 
data structure containing the input parameters. In cases where the user input 
may take different forms, such as the input for when a water table exists and 
the input for when a water table does not exist, each potential form of input 
is covered by at least one test case. The input module is also responsible for 
verifying the input, so for each possible violation of an input constraint, 
there is a corresponding test case verifying that the correct exception was 
thrown.

\paragraph{Valid User Input}

~\newline \noindent The test cases described in Table~\ref{InputTests} 
verify that each user input is correctly read. These test cases are identical 
to each other with the exception of the expected output on which they assert. 
The input for each is a file containing the inputs specified in 
Table~\ref{Inputs}. The type of these test cases is automatic. The initial 
state for each is a new session. The expected output for each is given in 
Table~\ref{InputTests}. The expected output is derived based on the given 
inputs. The tests will be performed as automated tests on a unit testing 
framework.

\begin{longtable}{  l  p{4cm}  p{6cm}  }
	\hline
	\textbf{Test Case} & \textbf{Test Name} & \textbf{Expected Output} \\
	\hline
	TC\refstepcounter{utestnum}\theutestnum  \label{TC_InputSlope} & 
	test-input\_slope &  \textit{slope.strat} = [(0, 25), (20, 25), (30, 20), 
	(40, 25), (70, 25)]\\
	\hline 
	TC\refstepcounter{utestnum}\theutestnum  \label{TC_InputPhi} & 
	test-input\_phi &  \textit{slope.phi} = 0.34906585040\\ 
	\hline
	TC\refstepcounter{utestnum}\theutestnum  \label{TC_InputCoh} & 
	test-input\_coh &  \textit{slope.coh} = 5000\\ 
	\hline
	TC\refstepcounter{utestnum}\theutestnum  \label{TC_InputGam} & 
	test-input\_gam &  \textit{slope.gam} = 15000\\ 
	\hline
	TC\refstepcounter{utestnum}\theutestnum  \label{TC_InputGams} & 
	test-input\_gams &  \textit{slope.gams} = 15000\\ 
	\hline
	TC\refstepcounter{utestnum}\theutestnum  \label{TC_InputWT} & 
	test-input\_piez &  \textit{piez.piez} = [(0, 22), 
	(10.87, 21.28), (21.14, 19.68), (31.21, 17.17), (38.69, 14.56), (40, 
	14), (70, 14)]\\
	\hline 
	TC\refstepcounter{utestnum}\theutestnum  \label{TC_InputGamw} & 
	test-input\_gamw &  \textit{piez.gamw} = 9800\\ 
	\hline
	TC\refstepcounter{utestnum}\theutestnum  \label{TC_InputXextMin} & 
	test-input\_xExtMin &  \textit{search.Xext}[0] = 34\\ 
	\hline
	TC\refstepcounter{utestnum}\theutestnum  \label{TC_InputXextMax} & 
	test-input\_xExtMax &  \textit{search.Xext}[1] = 53\\ 
	\hline
	TC\refstepcounter{utestnum}\theutestnum  \label{TC_InputXetrMin} & 
	test-input\_xEtrMin &  \textit{search.Xetr}[0] = 10\\ 
	\hline
	TC\refstepcounter{utestnum}\theutestnum  \label{TC_InputXetrMax} & 
	test-input\_xEtrMax &  \textit{search.Xetr}[1] = 24\\ 
	\hline
	TC\refstepcounter{utestnum}\theutestnum  \label{TC_InputYlimMin} & 
	test-input\_yLimMin &  \textit{search.Ylim}[0] = 5\\ 
	\hline
	TC\refstepcounter{utestnum}\theutestnum  \label{TC_InputYlimMax} & 
	test-input\_yLimMax &  \textit{search.Ylim}[1] = 26\\ 
	\hline
	TC\refstepcounter{utestnum}\theutestnum  \label{TC_InputLtoR} & 
	test-input\_ltor &  \textit{soln.ltor} = 1\\ 
	\hline
	TC\refstepcounter{utestnum}\theutestnum  \label{TC_InputFtype} & 
	test-input\_ftype &  \textit{soln.ftype} = 0\\ 
	\hline
	TC\refstepcounter{utestnum}\theutestnum  \label{TC_InputEvenslc} & 
	test-input\_evnslc &  \textit{soln.evnslc} = 1\\ 
	\hline
	TC\refstepcounter{utestnum}\theutestnum  \label{TC_InputCncvu} & 
	test-input\_cncvu &  \textit{soln.cncvu} = 1\\ 
	\hline
	TC\refstepcounter{utestnum}\theutestnum  \label{TC_InputObtu} & 
	test-input\_obtu &  \textit{soln.obtu} = 1\\ 
	\hline
	\caption{Input Test Cases}
	\label{InputTests}
\end{longtable}

\begin{enumerate}[label=TC\arabic*:,ref={\arabic*}]

\item [TC\refstepcounter{utestnum}\theutestnum: \label{TC_InputRtoL}] 
test-input\_rtol

Type: Automatic
					
Initial State: New session
					
Input: As described in Table~\ref{Inputs}, except with slope coordinates 
$\{\left(x_\text{us},y_\text{us}\right)\}$ increasing as $x$ increases, as 
follows: \{(0, 
15), (30, 15), (40, 20), (50, 25), (70, 25)\}.
					
Output: \textit{soln.ltor} = 0.

Test Case Derivation: Based on the given slope stratigraphy, \progname{} should 
detect that the slope elevation is increasing as $x$ increases, and set 
\textit{soln.ltor} accordingly.

How test will be performed: Automated test on unit testing framework.
					
\item [TC\refstepcounter{utestnum}\theutestnum: \label{TC_InputNoWTpiez}] 
test-input\_noWTpiez

Type: Automatic

Initial State: New session

Input: As described in Table~\ref{Inputs}, except with no water table.

Output: \textit{piez.piez} = [].

Test Case Derivation: If the input includes no water table vertices, the 
\textit{piez.piez} variable should be the empty sequence.

How test will be performed: Automated test on unit testing framework.

\item [TC\refstepcounter{utestnum}\theutestnum: \label{TC_InputNoWTgamw}] 
test-input\_noWTgamw

Type: Automatic

Initial State: New session

Input: As described in Table~\ref{Inputs}, except with no water table.

Output: \textit{piez.gamw} = 0.

Test Case Derivation: If the input includes no water table vertices, the 
\textit{piez.gamw} variable should be 0.

How test will be performed: Automated test on unit testing framework.

\item [TC\refstepcounter{utestnum}\theutestnum: \label{TC_Inputgamw0NoWT}] 
test-input\_noWTgamw0

Type: Automatic

Initial State: New session

Input: As described in Table~\ref{Inputs}, except with no water table. The 
verify\_params method must be called as part of this test.

Output: \textit{piez.gamw} = 0.

Test Case Derivation: If the input includes no water table vertices, the 
\textit{piez.gamw} variable should be 0, and verify\_params should not throw an 
exception.

How test will be performed: Automated test on unit testing framework.

\item [TC\refstepcounter{utestnum}\theutestnum: \label{TC_InputMin}] 
test-valid\_input\_minimal

Control: Automatic

Initial State: New session

Input: As described in Table~\ref{Inputs}, except with only 
2 slope coordinates $\{\left(x_\text{slope},y_\text{slope}\right)\}$, as 
follows: 
\{(0, 25), (20, 25). Also, with no water table vertices 
$\{\left(x_\text{wt},y_\text{wt}\right)\}$.  The verify\_params method must be 
called as part of this test.

Output: \textit{slope.strat} = [(0, 25), (20, 25)]

Test Case Derivation: \progname{} should accept an input slope with at least 2 
coordinates.

How test will be performed: Automated test on unit testing framework

\item [TC\refstepcounter{utestnum}\theutestnum: \label{TC_InputXMinEtrEqual}] 
test-input\_xMinEtrEqual

Control: Automatic

Initial State: New session

Input: As described in Table~\ref{Inputs}, except with  
${x_\text{slip}^\text{minEtr}}$ set to 0.  The verify\_params method must be 
called as part of this test.

Output: \textit{search.Xetr}[0] = 0

Test Case Derivation: \progname{} should accept values for the search 
parameters even if they lie on the edges of the slope.

How test will be performed: Automated test on unit testing framework

\item [TC\refstepcounter{utestnum}\theutestnum: \label{TC_InputXMaxExtEqual}] 
test-input\_xMaxExtEqual

Control: Automatic

Initial State: New session

Input: As described in Table~\ref{Inputs}, except with 
${x_\text{slip}^\text{maxExt}}$ set to 70.  The verify\_params method must be 
called as part of this test.

Output: \textit{search.Xext}[1] = 70

Test Case Derivation: \progname{} should accept values for the search 
parameters even if they lie on the edges of the slope.

How test will be performed: Automated test on unit testing framework
    
\end{enumerate}

\paragraph{Invalid User Input}
~\newline \noindent See \tcref{SVnV-TC_InvalidSlopeNonMonotonic} - 
\tcref{SVnV-TC_InvalidUnitWtWaterNegative} in the System VnV Plan document.

\subsubsection{Genetic Algorithm Module}

The test cases described in Table~\ref{GenAlgTests} 
verify that the critical slip surface returned by the genetic algorithm 
satisfies certain properties. Since the genetic algorithm includes randomness, 
it is impossible to verify the exact output. This is why test cases were 
selected to verify properties of the output instead. These test cases are 
identical to each other with the exception of the expected output on which they 
assert. The input for each is nothing; the genetic\_alg method does not take 
any inputs, it uses state variables instead. The type of these test cases is 
automatic. As the initial state for each, \textit{slope}, \textit{piez}, 
\textit{search}, and \textit{soln} state variables are loaded with the inputs 
from Table~\ref{Inputs}. For each test case, a predicate on the output that is 
expected to hold true is given in Table~\ref{GenAlgTests}. These 
predicates include a variable called $cslip$, which represents the critical 
slip surface output by the genetic algorithm. The expected output is derived 
based on the specification for the genetic algorithm found in the MIS document. 
The tests will be performed as automated tests on a unit testing framework.

\begin{longtable}{  l  p{4cm}  p{6cm}  }
	\hline
	\textbf{Test Case} & \textbf{Test Name} & \textbf{Expected Output} \\
	\hline
	TC\refstepcounter{utestnum}\theutestnum  \label{TC_GenAlgXextMin} & 
	test-genAlg\_xExtMin &  $\textit{cslip}[|\textit{cslip}|-1].x \geq 
	\textit{search.Xext}[0]$\\
	\hline 
	TC\refstepcounter{utestnum}\theutestnum  \label{TC_GenAlgXextMax} & 
	test-genAlg\_xExtMax &  $\textit{cslip}[|\textit{cslip}|-1].x \leq 
	\textit{search.Xext}[1]$\\
	\hline 
	TC\refstepcounter{utestnum}\theutestnum  \label{TC_GenAlgXetrMin} & 
	test-genAlg\_xEtrMin &  $\textit{cslip}[0].x \geq \textit{search.Xetr}[0]$\\
	\hline 
	TC\refstepcounter{utestnum}\theutestnum  \label{TC_GenAlgXetrMax} & 
	test-genAlg\_xEtrMax &  $\textit{cslip}[0].x \leq \textit{search.Xetr}[1]$\\
	\hline 
	TC\refstepcounter{utestnum}\theutestnum  \label{TC_GenAlgYlimMin} & 
	test-genAlg\_yLimMin &  $\forall(i : \mathbb{Z} | i \in 
	[0..|\textit{cslip}|-1] : \textit{cslip}[i].y \geq 
	\textit{search.Ylim}[0])$ \\
	\hline 
	TC\refstepcounter{utestnum}\theutestnum  \label{TC_GenAlgYlimMax} & 
	test-genAlg\_yLimMax &  $\forall(i : \mathbb{Z} | i \in 
	[0..|\textit{cslip}|-1] : \textit{cslip}[i].y \leq 
	\textit{search.Ylim}[1])$ \\
	\hline
	TC\refstepcounter{utestnum}\theutestnum  \label{TC_GenAlgVertices} & 
	test-genAlg\_vertices &  $|\textit{cslip}| = 
	(\text{INIT\_NUM\_VERTICES}*(2^\text{NUM\_ADDS}))-(2^\text{NUM\_ADDS}-1)$ 
	\\
	\hline
	\caption{Genetic Algorithm Test Cases}
	\label{GenAlgTests}
\end{longtable}

\subsubsection{Kinematic Admissibility Module}

Based on the specification for this module in the MIS, there are six conditions 
under which the module should return false. Thus, test cases have been selected 
to test each of these six conditions. For each condition, there is one test 
case verifying the behaviour when the condition is met and another test case 
verifying the behaviour when the slip surface is on the border of meeting the 
condition. For example, if the condition is that one value must be strictly 
less than another, there would be a test case where the two values are equal. 
Two of the conditions are turned on or off based on a boolean, and so an 
additional test case was selected for each of those to confirm that if the 
condition is turned off, the module should return true even if the condition is 
met. 

\begin{enumerate}[label=TC\arabic*:,ref={\arabic*}]
	
	\item [TC\refstepcounter{utestnum}\theutestnum: \label{TC_KinAdmXDec}] 
	test-kinAdm\_xDec
	
	Type: Automatic
	
	Initial State: \textit{slope}, \textit{piez}, \textit{search}, and 
	\textit{soln} state variables are loaded with the inputs from 
	Table~\ref{Inputs}.
	
	Input: [(20, 25), (30, 15), (25, 15), (40, 15)].
	
	Output: false.
	
	Test Case Derivation: The input was designed so that the $x$-ordinates are 
	not monotonically increasing, so the kinematic admissibility module should 
	return false.
	
	How test will be performed: Automated test on unit testing framework.
	
	\item [TC\refstepcounter{utestnum}\theutestnum: \label{TC_KinAdmXEq}] 
	test-kinAdm\_xEq
	
	Type: Automatic
	
	Initial State: \textit{slope}, \textit{piez}, \textit{search}, and 
	\textit{soln} state variables are loaded with the inputs from 
	Table~\ref{Inputs}.
	
	Input: [(20, 25), (30, 14), (30, 14), (40, 15)].
	
	Output: true.
	
	Test Case Derivation: The input was designed so that the $x$-ordinates are 
	equal for two adjacent points. This does not violate the condition, so the 
	kinematic admissibility module should return true.
	
	How test will be performed: Automated test on unit testing framework.
	
	\item [TC\refstepcounter{utestnum}\theutestnum: 
	\label{TC_KinAdmXetrOnSlope}] 
	test-kinAdm\_xEtrOnSlope
	
	Type: Automatic
	
	Initial State: \textit{slope}, \textit{piez}, \textit{search}, and 
	\textit{soln} state variables are loaded with the inputs from 
	Table~\ref{Inputs}.
	
	Input: [(-10, 25), (30, 15), (35, 12), (40, 15)].
	
	Output: false.
	
	Test Case Derivation: The input was designed so that the $x$-ordinate of 
	the slip surface entry is not on the slope, so the kinematic admissibility 
	module should return false.
	
	How test will be performed: Automated test on unit testing framework.
	
	\item [TC\refstepcounter{utestnum}\theutestnum: 
	\label{TC_KinAdmXetrOnEdge}] 
	test-kinAdm\_xEtrOnEdge
	
	Type: Automatic
	
	Initial State: \textit{slope}, \textit{piez}, \textit{search}, and 
	\textit{soln} state variables are loaded with the inputs from 
	Table~\ref{Inputs}.
	
	Input: [(0, 25), (30, 15), (35, 14), (40, 15)].
	
	Output: true.
	
	Test Case Derivation: The input was designed so that the $x$-ordinate of 
	the slip surface entry is just on the edge of the slope, so the kinematic 
	admissibility module should return true.
	
	How test will be performed: Automated test on unit testing framework.
	
	\item [TC\refstepcounter{utestnum}\theutestnum: 
	\label{TC_KinAdmXextOnSlope}] 
	test-kinAdm\_xExtOnSlope
	
	Type: Automatic
	
	Initial State: \textit{slope}, \textit{piez}, \textit{search}, and 
	\textit{soln} state variables are loaded with the inputs from 
	Table~\ref{Inputs}.
	
	Input: [(20, 25), (30, 15), (35, 12), (80, 15)].
	
	Output: false.
	
	Test Case Derivation: The input was designed so that the $x$-ordinate of 
	the slip surface exit is not on the slope, so the kinematic admissibility 
	module should return false.
	
	How test will be performed: Automated test on unit testing framework.
	
	\item [TC\refstepcounter{utestnum}\theutestnum: 
	\label{TC_KinAdmXextOnEdge}] 
	test-kinAdm\_xExtOnEdge
	
	Type: Automatic
	
	Initial State: \textit{slope}, \textit{piez}, \textit{search}, and 
	\textit{soln} state variables are loaded with the inputs from 
	Table~\ref{Inputs}.
	
	Input: [(20, 25), (30, 15), (35, 12), (70, 15)].
	
	Output: true.
	
	Test Case Derivation: The input was designed so that the $x$-ordinate of 
	the slip surface exit is just on the edge of the slope, so the kinematic 
	admissibility module should return true.
	
	How test will be performed: Automated test on unit testing framework.
	
	\item [TC\refstepcounter{utestnum}\theutestnum: \label{TC_KinAdmInSlope}] 
	test-kinAdm\_inSlope
	
	Type: Automatic
	
	Initial State: \textit{slope}, \textit{piez}, \textit{search}, and 
	\textit{soln} state variables are loaded with the inputs from 
	Table~\ref{Inputs}.
	
	Input: [(20, 25), (30, 25), (35, 12), (40, 15)].
	
	Output: false.
	
	Test Case Derivation: The input was designed so that one $x$-ordinate of 
	the slip surface is above the slope, so the kinematic admissibility module 
	should return false.
	
	How test will be performed: Automated test on unit testing framework.
	
	\item [TC\refstepcounter{utestnum}\theutestnum: \label{TC_KinAdmOnSlope}] 
	test-kinAdm\_onSlope
	
	Type: Automatic
	
	Initial State: \textit{slope}, \textit{piez}, \textit{search}, and 
	\textit{soln} state variables are loaded with the inputs from 
	Table~\ref{Inputs} and with \textit{soln.cncvu}=0 and \textit{soln.obtu}=0.
	
	Input: [(20, 25), (30, 15), (34, 18), (38, 15), (40, 15)].
	
	Output: true.
	
	Test Case Derivation: The input was designed so that one interior 
	$x$-ordinate of the slip surface is on the slope. This does not violate the 
	condition, so the kinematic admissibility module should return true. Note 
	that \textit{soln.cncvu} and \textit{soln.obtu} were set to 0 to relax the 
	concave-up condition and obtuse-angles condition, as this test case would 
	violate those conditions otherwise.
	
	How test will be performed: Automated test on unit testing framework.
	
	\item [TC\refstepcounter{utestnum}\theutestnum: \label{TC_KinAdmCncvUp}] 
	test-kinAdm\_cncvUp
	
	Type: Automatic
	
	Initial State: \textit{slope}, \textit{piez}, \textit{search}, and 
	\textit{soln} state variables are loaded with the inputs from 
	Table~\ref{Inputs}.
	
	Input: [(20, 25), (30, 15), (32, 15), (35, 12), (40, 15)].
	
	Output: false.
	
	Test Case Derivation: The input was designed so that the slip surface is 
	not fully concave-up, so the kinematic admissibility module should return 
	false.
	
	How test will be performed: Automated test on unit testing framework.
	
	\item [TC\refstepcounter{utestnum}\theutestnum: \label{TC_KinAdmCncvUpOff}] 
	test-kinAdm\_cncvUpOff
	
	Type: Automatic
	
	Initial State: \textit{slope}, \textit{piez}, \textit{search}, and 
	\textit{soln} state variables are loaded with the inputs from 
	Table~\ref{Inputs} and with \textit{soln.cncvu}=0 and \textit{soln.obtu} = 
	0.
	
	Input: [(20, 25), (30, 15), (32, 15), (35, 12), (40, 15)].
	
	Output: true.
	
	Test Case Derivation: The input was designed so that the slip surface is 
	not fully concave-up, but \textit{cncvu}=0, so the condition is not 
	enforced, so the kinematic admissibility module should return true. Note 
	that \textit{soln.obtu} was set to 0 to relax the obtuse-angles condition, 
	as this test case would violate that condition otherwise.
	
	How test will be performed: Automated test on unit testing framework.
	
	\item [TC\refstepcounter{utestnum}\theutestnum: \label{TC_KinAdmCncvUpEq}] 
	test-kinAdm\_cncvUpEq
	
	Type: Automatic
	
	Initial State: \textit{slope}, \textit{piez}, \textit{search}, and 
	\textit{soln} state variables are loaded with the inputs from 
	Table~\ref{Inputs}.
	
	Input: [(20, 25), (30, 15), (35, 15), (40, 15)].
	
	Output: true.
	
	Test Case Derivation: The input was designed so that the slip surface has 
	adjacent slopes that are equal. This does not violate the condition, so the 
	kinematic admissibility module should return true.
	
	How test will be performed: Automated test on unit testing framework.
	
	\item [TC\refstepcounter{utestnum}\theutestnum: \label{TC_KinAdmObtu}] 
	test-kinAdm\_obtu
	
	Type: Automatic
	
	Initial State: \textit{slope}, \textit{piez}, \textit{search}, and 
	\textit{soln} state variables are loaded with the inputs from 
	Table~\ref{Inputs} and with \textit{soln.cncvu}=0.
	
	Input: [(20, 25), (30, 15), (31, 5), (40, 15)].
	
	Output: false.
	
	Test Case Derivation: The input was designed so that the slip surface has 
	an angle less than 110 degrees, so the kinematic admissibility module 
	should return false. Note that \textit{soln.cncvu} was set to 0 to relax 
	the concave-up condition, as this test case would violate that condition 
	otherwise.
	
	How test will be performed: Automated test on unit testing framework.
	
	\item [TC\refstepcounter{utestnum}\theutestnum: \label{TC_KinAdmObtuOff}] 
	test-kinAdm\_obtuOff
	
	Type: Automatic
	
	Initial State: \textit{slope}, \textit{piez}, \textit{search}, and 
	\textit{soln} state variables are loaded with the inputs from 
	Table~\ref{Inputs} and with \textit{soln.cncvu}=0 and \textit{soln.obtu}=0.
	
	Input: [(20, 25), (30, 15), (31, 5), (40, 15)].
	
	Output: true.
	
	Test Case Derivation: The input was designed so that the slip surface has 
	an angle less than 110 degrees, but \textit{soln.obtu}=0, so the condition 
	is not enforced, so the kinematic admissibility module should return true. 
	Note that \textit{soln.cncvu} was set to 0 to relax the concave-up 
	condition, as this test case would violate that condition otherwise.
	
	How test will be performed: Automated test on unit testing framework.
	
\end{enumerate}

\subsubsection{Slip Weighting Module}
As specified in the MIS for this module, it accepts a list of slip surfaces, 
sorts them based on their factor of safety, and assigns a weight to each slip 
surface. To cover this functionality, there are test cases verifying that the 
slip surfaces are correctly sorted, and test cases verifying that the 
weights are calculated correctly. A final test case verifies that the weights 
are correct in the special case where every slip surface has the same factor of 
safety.

\begin{enumerate}[label=TC\arabic*:,ref={\arabic*}]
	
	\item [TC\refstepcounter{utestnum}\theutestnum: \label{TC_WeighterSort}] 
	test-weighter\_sort
	
	Type: Automatic
	
	Initial State: New session.
	
	Input: [(surf : [], Fs : 1.0, G : [], X : [], wt : 0, slicedSlip : []), 
	(surf : [], Fs : 0.8, G : [], X : [], wt : 0, slicedSlip : []), 
	(surf : [], Fs : 1.5, G : [], X : [], wt : 0, slicedSlip : [])].
	
	Output: $\textit{slips}^\prime$ such that the following predicate holds 
	true:\\ 
	$\textit{slips}^\prime[0].\textit{Fs} = 0.8\\ 
	\land \textit{slips}^\prime[0].\textit{Fs} = 1.0 \\
	\land \textit{slips}^\prime[0].\textit{Fs} = 1.5$.
	
	Test Case Derivation: The input list of slip surfaces was manually sorted 
	in ascending order of factor of safety to derive the expected output.
	
	How test will be performed: Automated test on unit testing framework.
	
	\item [TC\refstepcounter{utestnum}\theutestnum: \label{TC_WeighterSize}] 
	test-weighter\_size
	
	Type: Automatic
	
	Initial State: New session.
	
	Input: [(surf : [], Fs : 1.0, G : [], X : [], wt : 0, slicedSlip : []), 
	(surf : [], Fs : 1.5, G : [], X : [], wt : 0, slicedSlip : []),
	(surf : [], Fs : 0.8, G : [], X : [], wt : 0, slicedSlip : [])].
	
	Output: $\textit{slips}^\prime$ such that the following predicate holds 
	true:\\ 
	$|\textit{slips}^\prime| = 3$.
	
	Test Case Derivation: The weighter function should not change the size of 
	the input list of slip surfaces, so the expected size of the output is 
	equal to the size of the input.
	
	How test will be performed: Automated test on unit testing framework.
	
	\item [TC\refstepcounter{utestnum}\theutestnum: \label{TC_WeighterFirst}] 
	test-weighter\_first
	
	Type: Automatic
	
	Initial State: New session.
	
	Input: [(surf : [], Fs : 0.8, G : [], X : [], wt : 0, slicedSlip : []), 
	(surf : [], Fs : 1.4, G : [], X : [], wt : 0, slicedSlip : []), 
	(surf : [], Fs : 1.8, G : [], X : [], wt : 0, slicedSlip : []),
	(surf : [], Fs : 2.0, G : [], X : [], wt : 0, slicedSlip : [])].
	
	Output: $\textit{slips}^\prime$ such that the following predicate holds 
	true:\\ 
	$\textit{slips}^\prime[0].wt = 0.6$.

	Test Case Derivation: The input list of slip surfaces was chosen to be 
	already 
	sorted so that this test case is independent of the sorting. The expected 
	output was derived by manually calculating the weight of the first slip 
	surface according to the specification in the MIS.

	How test will be performed: Automated test on unit testing framework.
	
	\item [TC\refstepcounter{utestnum}\theutestnum: \label{TC_WeighterMid}] 
	test-weighter\_mid
	
	Type: Automatic
	
	Initial State: New session.
	
	Input: [(surf : [], Fs : 0.8, G : [], X : [], wt : 0, slicedSlip : []), 
	(surf : [], Fs : 1.4, G : [], X : [], wt : 0, slicedSlip : []), 
	(surf : [], Fs : 1.8, G : [], X : [], wt : 0, slicedSlip : []), 
	(surf : [], Fs : 2.0, G : [], X : [], wt : 0, slicedSlip : [])].
	
	Output: $\textit{slips}^\prime$ such that the following predicate holds 
	true:\\ 
	$\textit{slips}^\prime[1].wt = 0.9$.
	
	Test Case Derivation: The input list of slip surfaces was chosen to be 
	already 
	sorted so that this test case is independent of the sorting. The expected 
	output was derived by manually calculating the weight of the second slip 
	surface according to the specification in the MIS.
	
	How test will be performed: Automated test on unit testing framework.
	
	\item [TC\refstepcounter{utestnum}\theutestnum: 
	\label{TC_WeighterSecondLast}] 
	test-weighter\_secondLast
	
	Type: Automatic
	
	Initial State: New session.
	
	Input: [(surf : [], Fs : 0.8, G : [], X : [], wt : 0, slicedSlip : []), 
	(surf : [], Fs : 1.4, G : [], X : [], wt : 0, slicedSlip : []), 
	(surf : [], Fs : 1.8, G : [], X : [], wt : 0, slicedSlip : []), 
	(surf : [], Fs : 2.0, G : [], X : [], wt : 0, slicedSlip : [])].
	
	Output: $\textit{slips}^\prime$ such that the following predicate holds 
	true:\\ 
	$\textit{slips}^\prime[2].wt = 1.0$.
	
	Test Case Derivation: The input list of slip surfaces was chosen to be 
	already 
	sorted so that this test case is independent of the sorting. The expected 
	output was derived by manually calculating the weight of the third slip 
	surface according to the specification in the MIS.
	
	How test will be performed: Automated test on unit testing framework.
	
	\item [TC\refstepcounter{utestnum}\theutestnum: \label{TC_WeighterLast}] 
	test-weighter\_last
	
	Type: Automatic
	
	Initial State: New session.
	
	Input: [(surf : [], Fs : 0.8, G : [], X : [], wt : 0, slicedSlip : []), 
	(surf : [], Fs : 1.4, G : [], X : [], wt : 0, slicedSlip : []), 
	(surf : [], Fs : 1.8, G : [], X : [], wt : 0, slicedSlip : []), 
	(surf : [], Fs : 2.0, G : [], X : [], wt : 0, slicedSlip : [])].
	
	Output: $\textit{slips}^\prime$ such that the following predicate holds 
	true:\\ 
	$\textit{slips}^\prime[3].wt = 1.0$.
	
	Test Case Derivation: The input list of slip surfaces was chosen to be 
	already 
	sorted so that this test case is independent of the sorting. The expected 
	output was derived by manually calculating the weight of the fourth slip 
	surface according to the specification in the MIS.
	
	How test will be performed: Automated test on unit testing framework.
	
	\item [TC\refstepcounter{utestnum}\theutestnum: \label{TC_WeighterNaN}] 
	test-weighter\_NaN
	
	Type: Automatic
	
	Initial State: New session.
	
	Input: [(surf : [], Fs : 1.0, G : [], X : [], wt : 0, slicedSlip : []), 
	(surf : [], Fs : 1.0, G : [], X : [], wt : 0, slicedSlip : []), 
	(surf : [], Fs : 1.0, G : [], X : [], wt : 0, slicedSlip : [])].
	
	Output: $\textit{slips}^\prime$ such that the following predicate holds 
	true:\\ 
	$\forall(i : \mathbb{Z} | i \in [0..2] : \textit{slips}^\prime[i].wt = 
	\text{NaN})$.
	
	Test Case Derivation: All slip surfaces in the input list were chosen to 
	have equal factors of safety. According to the specification in the MIS, 
	this results in a division by zero in the calculation of the weights. In 
	MatLab, the result of a division by zero is NaN, thus each weight in the 
	output list is expected to be NaN.
	
	How test will be performed: Automated test on unit testing framework.

\end{enumerate}

\subsubsection{Slip Slicing Module}
The MIS for this module shows that it uses two different slicing algorithms, 
depending on the value of the boolean \textit{soln.evnslc}. Thus, a test case 
was selected for each algorithm, to ensure that it slices the given slip 
surface properly.

\begin{enumerate}[label=TC\arabic*:,ref={\arabic*}]
	
	\item [TC\refstepcounter{utestnum}\theutestnum: 
	\label{TC_SlicerEvnslcSize}] 
	test-slicer\_evnslcSize
	
	Type: Automatic
	
	Initial State: \textit{soln.evnslc} = 1.
	
	Input: [(0, 10), (8, 2), (10, 3), (16, 5)], 6.
	
	Output: $\textit{slip}^\prime$ such that the following predicate holds 
	true:\\ 
	$|\textit{slip}^\prime| = 7$.
	
	Test Case Derivation: The input 6 means that the output slip surface should 
	have 6 slices. The number of slices is one less than the number of 
	vertices, so the expected size of the output is 7.
	
	How test will be performed: Automated test on unit testing framework.
	
	\item [TC\refstepcounter{utestnum}\theutestnum: 
	\label{TC_SlicerEvnslcVals}] 
	test-slicer\_evnslcVals
	
	Type: Automatic
	
	Initial State: \textit{soln.evnslc} = 1.
	
	Input: [(0, 10), (8, 2), (10, 3), (16, 5)], 6.
	
	Output: [(0, 10), (2, 8), (4, 6), (8, 2), (10, 3), (13, 4), (16, 5)].
	
	Test Case Derivation: The specification in the MIS for the case where 
	\textit{soln.evnslc} is true was applied manually to the input coordinates 
	to derive the expected output coordinates.
	
	How test will be performed: Automated test on unit testing framework.
	
		\item [TC\refstepcounter{utestnum}\theutestnum: 
	\label{TC_SlicerNotEvnslcSize}] 
	test-slicer\_notEvnslcSize
	
	Type: Automatic
	
	Initial State: \textit{soln.evnslc} = 0.
	
	Input: [(0, 10), (8, 2), (10, 3), (16, 5)], 6.
	
	Output: $\textit{slip}^\prime$ such that the following predicate holds 
	true:\\ 
	$|\textit{slip}^\prime| = 7$.
	
	Test Case Derivation: The input 6 means that the output slip surface should 
	have 6 slices. The number of slices is one less than the number of 
	vertices, so the expected size of the output is 7.
	
	How test will be performed: Automated test on unit testing framework.
	
	\item [TC\refstepcounter{utestnum}\theutestnum: 
	\label{TC_SlicerNotEvnslcVals}] 
	test-slicer\_notEvnslcVals
	
	Type: Automatic
	
	Initial State: \textit{soln.evnslc} = 0.
	
	Input: [(0, 10), (8, 2), (10, 3), (16, 5)], 6.
	
	Output: [(0, 10), (4, 6), (8, 2), (9, 2.5), (10, 3), (13, 4), (16, 5)].
	
	Test Case Derivation: The specification in the MIS for the case where 
	\textit{soln.evnslc} is false was applied manually to the input coordinates 
	to derive the expected output coordinates.
	
	How test will be performed: Automated test on unit testing framework.
	
	\item [TC\refstepcounter{utestnum}\theutestnum: 
	\label{TC_SlicerNotEvnslcIndivisible}] 
	test-slicer\_notEvnslcIndivisible
	
	Type: Automatic
	
	Initial State: \textit{soln.evnslc} = 0.
	
	Input: [(0, 10), (8, 2), (10, 3)], 5.
	
	Output: $\textit{slip}^\prime$ such that the following predicate holds 
	true:\\ 
	$|\textit{slip}^\prime| = 5$.
	
	Test Case Derivation: In the case where \textit{soln.evnslc} is false, if 
	the desired number of slices is not divisible by the starting number of 
	slices, the desired number of slices should be adjusted to the greatest 
	number divisible by the starting number of slices but less than the input 
	desired number of slices. In this case, that number is 4, so the size of 
	the output should be 5.
	
	How test will be performed: Automated test on unit testing framework.
	
\end{enumerate}

\subsubsection{Morgenstern-Price Calculation Module}
Correct calculation of the factor of safety is covered by \tcref{SVnV-TC_Ex1FS} 
and \tcref{SVnV-TC_OrigProgFS} in the System VnV Plan document. Correct 
calculation of the interslice normal forces and interslice shear forces are 
covered by \tcref{SVnV-TC_OrigProgNormal} and \tcref{SVnV-TC_OrigProgShear}, 
respectively, in the System VnV Plan document. The additional test cases 
specified here cover special cases not covered by the System VnV Plan: 
non-converging and spurious solutions. The MIS states that \progname{} should 
return a factor of safety of 1000 and empty lists for the interslice forces in 
both of these cases. For each of these special cases, there are three test 
cases. One covers the factor of safety, one covers the interslice normal force, 
and one covers the interslice shear force.

\begin{enumerate}[label=TC\arabic*:,ref={\arabic*}]
	
	\item [TC\refstepcounter{utestnum}\theutestnum: 
	\label{TC_MorgPriceNonConvergingFs}] 
	test-morgPrice\_nonConvergingFs
	
	Type: Automatic
	
	Initial State: \textit{slope}, \textit{piez}, \textit{search}, and 
	\textit{soln} state variables are loaded with the inputs from 
	Table~\ref{Inputs}.
	
	Input: [(20, 25), (30, 10), (40, 15)].
	
	Output: $\textit{slip}^\prime$ such that the following predicate holds 
	true:\\ 
	$\textit{slip}^\prime.\textit{Fs} = 1000$.
	
	Test Case Derivation: The input slip surface was chosen as one that is 
	known to not converge within MAX\_ITER iterations, and thus will meet the 
	non-converging condition specified in the MIS. In this case, the 
	specification states that the module should return a factor of safety of 
	1000.
	
	How test will be performed: Automated test on unit testing framework.
	
	\item [TC\refstepcounter{utestnum}\theutestnum: 
	\label{TC_MorgPriceNonConvergingG}] 
	test-morgPrice\_nonConvergingG
	
	Type: Automatic
	
	Initial State: \textit{slope}, \textit{piez}, \textit{search}, and 
	\textit{soln} state variables are loaded with the inputs from 
	Table~\ref{Inputs}.
	
	Input: [(20, 25), (30, 10), (40, 15)].
	
	Output: $\textit{slip}^\prime$ such that the following predicate holds 
	true:\\ 
	$\textit{slip}^\prime.\textit{G} = []$.
	
	Test Case Derivation: The input slip surface was chosen as one that is 
	known to not converge within MAX\_ITER iterations, and thus will meet the 
	non-converging condition specified in the MIS. In this case, the 
	specification states that the module should return an empty list of 
	interslice normal forces.
	
	How test will be performed: Automated test on unit testing framework.
	
	\item [TC\refstepcounter{utestnum}\theutestnum: 
	\label{TC_MorgPriceNonConvergingX}] 
	test-morgPrice\_nonConvergingX
	
	Type: Automatic
	
	Initial State: \textit{slope}, \textit{piez}, \textit{search}, and 
	\textit{soln} state variables are loaded with the inputs from 
	Table~\ref{Inputs}.
	
	Input: [(20, 25), (30, 10), (40, 15)].
	
	Output: $\textit{slip}^\prime$ such that the following predicate holds 
	true:\\ 
	$\textit{slip}^\prime.\textit{X} = []$.
	
	Test Case Derivation: The input slip surface was chosen as one that is 
	known to not converge within MAX\_ITER iterations, and thus will meet the 
	non-converging condition specified in the MIS. In this case, the 
	specification states that the module should return an empty list of 
	interslice shear forces.
	
	How test will be performed: Automated test on unit testing framework.
	
	\item [TC\refstepcounter{utestnum}\theutestnum: 
	\label{TC_MorgPriceSpuriousFs}] 
	test-morgPrice\_spuriousFs
	
	Type: Automatic
	
	Initial State: \textit{slope}, \textit{piez}, \textit{search}, and 
	\textit{soln} state variables are loaded with the inputs from 
	Table~\ref{Inputs}, except with $\varphi' = 1$ and $c' = 0.1$.
	
	Input: [(20, 25), (25, 20), (30, 15), (35, 14), (40, 15)].
	
	Output: $\textit{slip}^\prime$ such that the following predicate holds 
	true:\\ 
	$\textit{slip}^\prime.\textit{Fs} = 1000$.
	
	Test Case Derivation: The input slip surface was chosen as one that is 
	known to have a factor of safety less than 0.5, and thus will meet the 
	spurious condition specified in the MIS. In this case, the specification 
	states that the module should return a factor of safety of 1000.
	
	How test will be performed: Automated test on unit testing framework.
	
	\item [TC\refstepcounter{utestnum}\theutestnum: 
	\label{TC_MorgPriceSpuriousG}] 
	test-morgPrice\_spuriousG
	
	Type: Automatic
	
	Initial State: \textit{slope}, \textit{piez}, \textit{search}, and 
	\textit{soln} state variables are loaded with the inputs from 
	Table~\ref{Inputs}, except with $\varphi' = 1$ and $c' = 0.1$.
	
	Input: [(20, 25), (25, 20), (30, 15), (35, 14), (40, 15)].
	
	Output: $\textit{slip}^\prime$ such that the following predicate holds 
	true:\\ 
	$\textit{slip}^\prime.\textit{G} = []$.
	
	Test Case Derivation: The input slip surface was chosen as one that is 
	known to have a factor of safety less than 0.5, and thus will meet the 
	spurious condition specified in the MIS. In this case, the specification 
	states that the module should return an empty list of interslice shear 
	forces.
	
	How test will be performed: Automated test on unit testing framework.
	
	\item [TC\refstepcounter{utestnum}\theutestnum: 
	\label{TC_MorgPriceSpuriousX}] 
	test-morgPrice\_spuriousX
	
	Type: Automatic
	
	Initial State: \textit{slope}, \textit{piez}, \textit{search}, and 
	\textit{soln} state variables are loaded with the inputs from 
	Table~\ref{Inputs}, except with $\varphi' = 1$ and $c' = 0.1$.
	
	Input: [(20, 25), (25, 20), (30, 15), (35, 14), (40, 15)].
	
	Output: $\textit{slip}^\prime$ such that the following predicate holds 
	true:\\ 
	$\textit{slip}^\prime.\textit{X} = []$.
	
	Test Case Derivation: The input slip surface was chosen as one that is 
	known to have a factor of safety less than 0.5, and thus will meet the 
	spurious condition specified in the MIS. In this case, the specification 
	states that the module should return an empty list of interslice shear 
	forces.
	
	How test will be performed: Automated test on unit testing framework.
	
\end{enumerate}

\subsubsection{Slice Property Calculation Module}
The MIS shows that this module is responsible for calculating seven properties 
for each slice of a slip surface. The calculations of some of these properties 
are themselves dependent on conditions. Test cases have been selected such that 
each possible calculation is covered by at least one test case. 

\begin{table}[!h]
	\renewcommand{\arraystretch}{1.5}
	\begin{tabularx}{1.0\textwidth}{p{3cm} l X}
		\toprule \textbf{Input} &
		\textbf{Unit} & \textbf{Value}\\ \midrule
		$\{\left(x_\text{slope},y_\text{slope}\right)\}$ & $\text{m}$ & \{(-10, 
		10), (0, 20), (30, 20), (40, 25), (70, 25)\}\\
		$\{\left(x_\text{wt},y_\text{wt}\right)\}$ & $\text{m}$ & \{(-10, 25),
		(10, 25), (20, 10), (70, 10)\}\\
		$\gamma$ & $\si{\newton\per\meter\cubed}$ & 15000 \\
		$\gamma_{\text{Sat}}$ & $\si{\newton\per\meter\cubed}$ & 20000 \\
		$\gamma_{\text{w}}$ & $\si{\newton\per\meter\cubed}$ & 10000 \\ 
		\bottomrule
	\end{tabularx}
	\caption{Input to be used for slice property calculation test cases}
	\label{SlicePropInputs}
\end{table}

\begin{enumerate}[label=TC\arabic*:,ref={\arabic*}]
	
	\item [TC\refstepcounter{utestnum}\theutestnum: 
	\label{TC_PropertyWWet}] 
	test-prop\_wWet
	
	Type: Automatic
	
	Initial State: \textit{slope}, \textit{piez}, \textit{search}, and 
	\textit{soln} state variables are loaded with the inputs from 
	Table~\ref{SlicePropInputs}. For any values not specified in 
	Table~\ref{SlicePropInputs}, the input is based on Table~\ref{Inputs}.
	
	Input: [(0, 15), (10, 10)].
	
	Output: $\textit{force.W}[0] = 1500000$.
	
	Test Case Derivation: The expected output was derived by manually 
	calculating the weight of the input slice based on the specification in the 
	MIS for the case where the water table lies completely above the slice.
	
	How test will be performed: Automated test on unit testing framework.
	
	\item [TC\refstepcounter{utestnum}\theutestnum: 
	\label{TC_PropertyWWetDry}] 
	test-prop\_wWetDry
	
	Type: Automatic
	
	Initial State: \textit{slope}, \textit{piez}, \textit{search}, and 
	\textit{soln} state variables are loaded with the inputs from 
	Table~\ref{SlicePropInputs}. For any values not specified in 
	Table~\ref{SlicePropInputs}, the input is based on Table~\ref{Inputs}.
	
	Input: [(20, 5), (30, 0)].
	
	Output: $\textit{force.W}[0] = 3000000$.
	
	Test Case Derivation: The expected output was derived by manually 
	calculating the weight of the input slice based on the specification in the 
	MIS for the case where the water table lies partially below and partially 
	above the slice.
	
	How test will be performed: Automated test on unit testing framework.
	
	\item [TC\refstepcounter{utestnum}\theutestnum: 
	\label{TC_PropertyWDry}] 
	test-prop\_wDry
	
	Type: Automatic
	
	Initial State: \textit{slope}, \textit{piez}, \textit{search}, and 
	\textit{soln} state variables are loaded with the inputs from 
	Table~\ref{SlicePropInputs}. For any values not specified in 
	Table~\ref{SlicePropInputs}, the input is based on Table~\ref{Inputs}.
	
	Input: [(20, 15), (30, 10)].
	
	Output: $\textit{force.W}[0] = 1125000$.
	
	Test Case Derivation: The expected output was derived by manually 
	calculating the weight of the input slice based on the specification in the 
	MIS for the case where the water table lies completely below the slice.
	
	How test will be performed: Automated test on unit testing framework.
	
	\item [TC\refstepcounter{utestnum}\theutestnum: 
	\label{TC_PropertyUbWet}] 
	test-prop\_ubWet
	
	Type: Automatic
	
	Initial State: \textit{slope}, \textit{piez}, \textit{search}, and 
	\textit{soln} state variables are loaded with the inputs from 
	Table~\ref{SlicePropInputs}. For any values not specified in 
	Table~\ref{SlicePropInputs}, the input is based on Table~\ref{Inputs}.
	
	Input: [(0, 15), (10, 10)].
	
	Output: $\textit{force.Ub}[0] = 1397542.4859$.
	
	Test Case Derivation: The expected output was derived by manually 
	calculating the base hydrostatic force on the input slice based on the 
	specification in the MIS for the case where the water table lies completely 
	above the slice.
	
	How test will be performed: Automated test on unit testing framework.
	
	\item [TC\refstepcounter{utestnum}\theutestnum: 
	\label{TC_PropertyUbDry}] 
	test-prop\_ubDry
	
	Type: Automatic
	
	Initial State: \textit{slope}, \textit{piez}, \textit{search}, and 
	\textit{soln} state variables are loaded with the inputs from 
	Table~\ref{SlicePropInputs}. For any values not specified in 
	Table~\ref{SlicePropInputs}, the input is based on Table~\ref{Inputs}.
	
	Input: [(20, 15), (30, 10)].
	
	Output: $\textit{force.Ub}[0] = 0$.
	
	Test Case Derivation: The expected output was derived by manually 
	calculating the base hydrostatic force on the input slice based on the 
	specification in the MIS for the case where the water table lies completely 
	below the slice.
	
	How test will be performed: Automated test on unit testing framework.
	
	\item [TC\refstepcounter{utestnum}\theutestnum: 
	\label{TC_PropertyUtWet}] 
	test-prop\_utWet
	
	Type: Automatic
	
	Initial State: \textit{slope}, \textit{piez}, \textit{search}, and 
	\textit{soln} state variables are loaded with the inputs from 
	Table~\ref{SlicePropInputs}. For any values not specified in 
	Table~\ref{SlicePropInputs}, the input is based on Table~\ref{Inputs}.
	
	Input: [(-10, 5), (0, 10)].
	
	Output: $\textit{force.Ut}[0] = 1414213.5624$.
	
	Test Case Derivation: The expected output was derived by manually 
	calculating the surface hydrostatic force on the input slice based on the 
	specification in the MIS for the case where the water table lies completely 
	above the slice.
	
	How test will be performed: Automated test on unit testing framework.
	
	\item [TC\refstepcounter{utestnum}\theutestnum: 
	\label{TC_PropertyUtDry}] 
	test-prop\_utDry
	
	Type: Automatic
	
	Initial State: \textit{slope}, \textit{piez}, \textit{search}, and 
	\textit{soln} state variables are loaded with the inputs from 
	Table~\ref{SlicePropInputs}. For any values not specified in 
	Table~\ref{SlicePropInputs}, the input is based on Table~\ref{Inputs}.
	
	Input: [(20, 15), (30, 10)].
	
	Output: $\textit{force.Ut}[0] = 0$.
	
	Test Case Derivation: The expected output was derived by manually 
	calculating the surface hydrostatic force on the input slice based on the 
	specification in the MIS for the case where the water table lies completely 
	below the slice.
	
	How test will be performed: Automated test on unit testing framework.
	
	\item [TC\refstepcounter{utestnum}\theutestnum: 
	\label{TC_PropertyHWet}] 
	test-prop\_hWet
	
	Type: Automatic
	
	Initial State: \textit{slope}, \textit{piez}, \textit{search}, and 
	\textit{soln} state variables are loaded with the inputs from 
	Table~\ref{SlicePropInputs}. For any values not specified in 
	Table~\ref{SlicePropInputs}, the input is based on Table~\ref{Inputs}.
	
	Input: [(-10, 10), (0, 15), (10, 10)].
	
	Output: $\textit{force.H}[0] = 375000$.
	
	Test Case Derivation: The expected output was derived by manually 
	calculating the interslice water force on the first interface of the input 
	slice based on the specification in the MIS for the case where the water 
	table lies above the slice interface.
	
	How test will be performed: Automated test on unit testing framework.
	
	\item [TC\refstepcounter{utestnum}\theutestnum: 
	\label{TC_PropertyHWetDry}] 
	test-prop\_hWetDry
	
	Type: Automatic
	
	Initial State: \textit{slope}, \textit{piez}, \textit{search}, and 
	\textit{soln} state variables are loaded with the inputs from 
	Table~\ref{SlicePropInputs}. For any values not specified in 
	Table~\ref{SlicePropInputs}, the input is based on Table~\ref{Inputs}.
	
	Input: [(0, 15), (20, 5), (30, 0)].
	
	Output: $\textit{force.H}[0] = 125000$.
	
	Test Case Derivation: The expected output was derived by manually 
	calculating the interslice water force on the first interface of the input 
	slice based on the specification in the	MIS for the case where the water 
	table crosses the slice interface.
	
	How test will be performed: Automated test on unit testing framework.
	
	\item [TC\refstepcounter{utestnum}\theutestnum: 
	\label{TC_PropertyHDry}] 
	test-prop\_hDry
	
	Type: Automatic
	
	Initial State: \textit{slope}, \textit{piez}, \textit{search}, and 
	\textit{soln} state variables are loaded with the inputs from 
	Table~\ref{SlicePropInputs}. For any values not specified in 
	Table~\ref{SlicePropInputs}, the input is based on Table~\ref{Inputs}.
	
	Input: [(0, 15), (20, 15), (30, 10)].
	
	Output: $\textit{force.H}[0] = 0$.
	
	Test Case Derivation: The expected output was derived by manually 
	calculating the interslice water force on the first interface of the input 
	slice based on the specification in the MIS for the case where the water 
	table lies below the slice interface.
	
	How test will be performed: Automated test on unit testing framework.
	
	\item [TC\refstepcounter{utestnum}\theutestnum: 
	\label{TC_PropertyAlpha}] 
	test-prop\_alpha
	
	Type: Automatic
	
	Initial State: \textit{slope}, \textit{piez}, \textit{search}, and 
	\textit{soln} state variables are loaded with the inputs from 
	Table~\ref{SlicePropInputs}. For any values not specified in 
	Table~\ref{SlicePropInputs}, the input is based on Table~\ref{Inputs}.
	
	Input: [(20, 15), (30, 10)].
	
	Output: $\textit{angles.alpha}[0] = -0.46364760900$.
	
	Test Case Derivation: The expected output was derived by manually 
	calculating the base angle of the input slice based on the specification in 
	the MIS.
	
	How test will be performed: Automated test on unit testing framework.
	
	\item [TC\refstepcounter{utestnum}\theutestnum: 
	\label{TC_PropertyBeta}] 
	test-prop\_beta
	
	Type: Automatic
	
	Initial State: \textit{slope}, \textit{piez}, \textit{search}, and 
	\textit{soln} state variables are loaded with the inputs from 
	Table~\ref{SlicePropInputs}. For any values not specified in 
	Table~\ref{SlicePropInputs}, the input is based on Table~\ref{Inputs}.
	
	Input: [(-10, 10), (0, 10)].
	
	Output: $\textit{angles.beta}[0] = 0.78539816340$.
	
	Test Case Derivation: The expected output was derived by manually 
	calculating the surface angle of the input slice based on the specification 
	in the MIS.
	
	How test will be performed: Automated test on unit testing framework.
	
	\item [TC\refstepcounter{utestnum}\theutestnum: 
	\label{TC_PropertyHeight}] 
	test-prop\_height
	
	Type: Automatic
	
	Initial State: \textit{slope}, \textit{piez}, \textit{search}, and 
	\textit{soln} state variables are loaded with the inputs from 
	Table~\ref{SlicePropInputs}. For any values not specified in 
	Table~\ref{SlicePropInputs}, the input is based on Table~\ref{Inputs}.
	
	Input: [(0, 15), (10, 10)].
	
	Output: $\textit{heights}[0] = 7.5$.
	
	Test Case Derivation: The expected output was derived by manually 
	calculating the height of the input slice based on the specification in the 
	MIS.
	
	How test will be performed: Automated test on unit testing framework.
	
\end{enumerate}

\subsubsection{Output Module}
The verification of the output is covered by \tcref{SVnV-TC_ValidOutFS} in the 
System VnV Plan document, but this plan adds more tests for the case where the 
verification of the output triggers an exception. The delivery of each output 
is covered by 
\tcref{SVnV-TC_OutInputs} to \tcref{SVnV-TC_OutShear} in the System VnV Plan 
document. This document only adds test cases for items not covered by 
the System VnV Plan document. The second test case here verifies that an output 
file with the correct name is created.

\begin{enumerate}[label=TC\arabic*:,ref={\arabic*}]
	
	\item [TC\refstepcounter{utestnum}\theutestnum: \label{TC_InvalidFs}] 
	test-verify\_output\_FsNegative
	
	Type: Automatic
	
	Initial State: \textit{slope}, \textit{piez}, \textit{search}, and 
	\textit{soln} state variables are loaded with the inputs from 
	Table~\ref{Inputs}.
	
	Input: -1.0.
	
	Output: An exception is triggered.
	
	Test Case Derivation: It makes no physical sense for a factor of safety to 
	be less than 0, so in accordance with the MIS, \progname{} should throw an 
	exception.
	
	How test will be performed: Automated test on unit testing framework.
	
	\item [TC\refstepcounter{utestnum}\theutestnum: \label{TC_InvalidFs0}] 
	test-verify\_output\_Fs0
	
	Type: Automatic
	
	Initial State: \textit{slope}, \textit{piez}, \textit{search}, and 
	\textit{soln} state variables are loaded with the inputs from 
	Table~\ref{Inputs}.
	
	Input: 0.
	
	Output: An exception is triggered.
	
	Test Case Derivation: It makes no physical sense for a factor of safety to 
	be 0, so in accordance with the MIS, \progname{} should throw an exception.
	
	How test will be performed: Automated test on unit testing framework.
	
	\item [TC\refstepcounter{utestnum}\theutestnum: \label{TC_OutputFile}] 
	test-output\_file
	
	Type: Automatic
	
	Initial State: \textit{slope}, \textit{piez}, \textit{search}, and 
	\textit{soln} state variables are loaded with the inputs from 
	Table~\ref{Inputs}.
	
	Input: 1.0, [], [], [], ``outputtest''.
	
	Output: A file called outputtest.out exists in the file system.
	
	Test Case Derivation: As stated in the MIS, the output module should write 
	a file with extension ``.out'' and the name given as string input to the 
	module.
	
	How test will be performed: Automated test on unit testing framework.
	
\end{enumerate}

\subsection{Tests for Nonfunctional Requirements}

Not applicable for any of the modules of \progname{}.

\subsection{Traceability Between Test Cases and Modules}

Table~\ref{TestModuleTrace} shows the traceability between test cases and 
modules. Every module is covered by test cases except for those that were 
declared out of scope in Section~\ref{Scope} of this document.

\afterpage{
\begin{landscape}
\begin{table}[!h]
	\centering
	\begin{tabular}{|c|c|c|c|c|c|c|c|c|c|c|c|c|c|}
		\hline
		& \mref{MG-mHH}& \mref{MG-mControl}& \mref{MG-mInput}& 
		\mref{MG-mGenAlg}& \mref{MG-mKinAdm} & \mref{MG-mSlipWeight}& 
		\mref{MG-mSlipSlicer}& \mref{MG-mMorgPrice}& 
		\mref{MG-mSliceProperty}& 
		\mref{MG-mOutput}& \mref{MG-mArrayOps}& \mref{MG-mRandNum}& 
		\mref{MG-mPlot} \\
		\hline
		\tcref{TC_InputSlope} - \tcref{TC_InputXMaxExtEqual}                 
		& & & X& & & & & & & & & & \\ \hline
		\tcref{TC_GenAlgXextMin} - \tcref{TC_GenAlgVertices} 
		& & & & X& & & & & & & & &\\ \hline
		\tcref{TC_KinAdmXDec} - \tcref{TC_KinAdmObtuOff}          
		& & & & & X& & & & & & & & \\ \hline
		\tcref{TC_WeighterSort} - \tcref{TC_WeighterLast}       
		& & & & & & X& & & & & & & \\ \hline
		\tcref{TC_SlicerEvnslcSize} - \tcref{TC_SlicerNotEvnslcVals}
		& & & & & & & X& & & & & & \\ \hline
		\tcref{TC_MorgPriceNonConvergingFs} - \tcref{TC_MorgPriceSpuriousX}   
		& & & & & & & & X& & & & & \\ \hline
		\tcref{TC_PropertyWWet} - \tcref{TC_PropertyHeight}
		& & & & & & & & & X& & & & \\ \hline
		\tcref{TC_InvalidFs} - \tcref{TC_OutputFile}
		& & & & & & & & & & X& & & \\
		\hline
	\end{tabular}
	\caption{Traceability Matrix Showing the Connections Between Modules and 
	Test Cases}
	\label{TestModuleTrace}
\end{table}
\end{landscape}
}

\newpage

\bibliographystyle{plainnat}

\bibliography{../../../refs/References}

\end{document}