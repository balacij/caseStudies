\documentclass[12pt, titlepage]{article}

\usepackage{tabularx}
\usepackage{longtable}
\usepackage{comment}
\usepackage{amsmath}
\usepackage{amssymb}
\usepackage{booktabs}
\usepackage{xr}
\usepackage{siunitx}
\usepackage{caption}
\usepackage{hyperref}
\usepackage[numbib,nottoc]{tocbibind}
\hypersetup{
    colorlinks,
    citecolor=green,
    filecolor=black,
    linkcolor=red,
    urlcolor=blue
}
\usepackage[round]{natbib}
\usepackage{enumitem}

\usepackage{xr}
\externaldocument[SRS-]{../../SRS/SRS}
\newcommand{\rref}[1]{R\ref{#1}}

\newcounter{testnum} %Assumption Number
\newcommand{\tcthetestnum}{TC\thetestnum}
\newcommand{\tcref}[1]{TC\ref{#1}}

%% Comments

\usepackage{color}

\newif\ifcomments\commentstrue

\ifcomments
\newcommand{\authornote}[3]{\textcolor{#1}{[#3 ---#2]}}
\newcommand{\todo}[1]{\textcolor{red}{[TODO: #1]}}
\else
\newcommand{\authornote}[3]{}
\newcommand{\todo}[1]{}
\fi

\newcommand{\wss}[1]{\authornote{blue}{SS}{#1}}
\newcommand{\an}[1]{\authornote{magenta}{Author}{#1}}


\newcommand{\progname}{SSP}

\begin{document}

\title{Software Stability Analysis: System Verification and Validation Plan} 
\author{Brooks MacLachlan}
\date{\today}
	
\maketitle

\pagenumbering{roman}

\section{Revision History}

\begin{tabularx}{\textwidth}{p{3cm}p{2cm}X}
\toprule {\bf Date} & {\bf Version} & {\bf Notes}\\
\midrule
10/10/18 & 1.0 & Initial template fill-ins\\
\bottomrule
\end{tabularx}

~\newpage

\section{Symbols, Abbreviations and Acronyms}
The symbols, abbreviations, and acronyms used in this document include those 
defined in the table below, as well as any defined in the tables found in 
Section \ref{SRS-sec_RefMat} of the Software Requirements Specification (SRS) 
document.
\newline

\renewcommand{\arraystretch}{1.2}
\begin{tabular}{l l} 
  \toprule		
  \textbf{symbol} & \textbf{description}\\
  \midrule 
  MIS & Module Interface Specification\\
  MG & Module Guide\\
  TC & Test Case\\
  VnV & Verification and Validation\\
  \bottomrule
\end{tabular}\\

\newpage

\tableofcontents

\listoftables

\listoffigures

\newpage

\pagenumbering{arabic}

\noindent This document outlines the system verification and validation plan 
for the software. General information regarding the system under test and the 
objectives of the verification and validation activities is provided in Section 
\ref{sec_GenInfo}. Overviews of verification plans for the SRS, design, and 
implementation are given in Section \ref{sec_Plan}, along with a summary of the 
validation plan for the software. Section \ref{sec_System} details specific  
system test cases for verifying the requirements outlined in Section 
\ref{SRS-sec_Reqs} of the SRS. A summary of planned static verification 
activities can be found in Section \ref{sec_Static} of this document.

\section{General Information} \label{sec_GenInfo}

\subsection{Summary}

\noindent The software being tested is the Slope Stability Analysis Program 
(\progname{}). Based on user-defined slope geometry and material properties, 
\progname{} determines the critical slip surface of the given slope, the 
corresponding factor of safety, and interslice normal and shear forces along 
the critical slip surface.

\subsection{Objectives}

\noindent The purpose of the verification and validation activities is to 
confirm that \progname{} exhibits desired software qualities. The primary 
objective is to build confidence in the correctness of the software. The tests 
described in this document cannot definitively prove correctness, but they can 
build confidence by verifying that the software is correct for the cases 
covered by tests. Other important qualities to be verified are the 
understandability, maintainability, and reusability of the software.

\subsection{References}

\noindent Extensive information about the purpose and requirements of 
\progname{} can be found in the SRS document. This System VnV Plan is 
complemented by the System VnV Report, where the results of the tests planned 
in this document are discussed. For more details on test cases specific to the 
implementation of \progname{}, consult the Unit VnV Plan document. The latest 
documentation for \progname{} can be found on GitHub, at \newline 
\href{https://github.com/smiths/caseStudies/tree/master/CaseStudies/ssp}{https://github.com/smiths/caseStudies/tree/master/CaseStudies/ssp}.

\section{Plan} \label{sec_Plan}
	
\subsection{Verification and Validation Team}

\noindent Brooks MacLachlan is responsible for the verification and validation 
of \progname{}, though input from various students and the professor, Dr. 
Spencer 
Smith, of CAS 741 will also contribute.

\subsection{SRS Verification Plan}

\noindent SRS verification will be carried out by reviews. Brooks MacLachlan 
will extensively review the SRS, including reviewing the sources to confirm 
that the theories and models described in the SRS are correct. Any issues 
identified during this review will be fixed immediately by Brooks MacLachlan. 
This review will be followed up by additional reviews by Dr. Spencer Smith and 
Vajiheh Motamer, a student in CAS 741. Any issues identified by these reviewers 
will be recorded through the issue tracker on GitHub, and addressed by Brooks 
MacLachlan. Brooks MacLachlan will also informally present the SRS to the 
entire class of CAS 741. Any issues discovered during this presentation will be 
subsequently fixed by Brooks MacLachlan. A focus of these reviews will be to 
verify the non-functional requirements of correctness and understandability by 
identifying information in the SRS that is incorrect or ambiguous.

\subsection{Design Verification Plan}

\noindent The design of \progname{} is outlined in the Module Guide (MG) and 
Module Interface Specification (MIS) documents. The design will be verified by 
review of these documents. Brooks MacLachlan will extensively review both 
documents. To verify correctness, part of this review will be to ensure that 
every module traces to a requirement and that every requirement is traced to by 
a module. To evaluate the modularity of the design, "uses" relationships 
between modules will be examined to ensure modules do not mutually use each 
other. The review will also check that each module hides exactly one secret. 
Ensuring the program is well-modularized contributes to verification of the 
non-functional requirements of maintainability and reusability. Any issues 
identified during these reviews will be fixed immediately by Brooks MacLachlan. 
Dr. Spencer Smith will also review both documents. An additional review will be 
conducted by students in CAS 741: Karol Serkis for the MG and Malavika 
Srinivasan for the MIS. Any issues identified by these reviews will be recorded 
through the issue tracker on GitHub, and addressed by Brooks 
MacLachlan. Brooks MacLachlan will also informally present the MG to the entire 
class of CAS 741. Any design issues discovered during this presentation will be 
subsequently fixed by Brooks MacLachlan. Reviews of these design documents will 
also focus on ensuring understandability by identifying descriptions and 
specifications that are ambiguous.

\subsection{Implementation Verification Plan} \label{sec_ImpPlan}

\noindent The implementation of \progname{} will be verified by review and by 
testing. Brooks MacLachlan will extensively review the implementation. Any 
issues identified during these reviews will be fixed immediately by Brooks 
MacLachlan. This review will be followed up by reviews by Dr. Spencer Smith and 
Robert White, a student in CAS 741. Any  issues identified by these reviews 
will be recorded through the issue tracker on GitHub, and addressed by Brooks 
MacLachlan. Brooks MacLachlan will also informally present the implementation 
to the entire class of CAS 741. Any implementation issues discovered during 
this presentation will be subsequently fixed by Brooks MacLachlan. These 
reviews will contribute to verifying correctness and understandability of the 
software by identifying code that is not traceable to any specifications 
described in the SRS, MG, or MIS.
\newline

\noindent The implementation will also be verified by testing. Specific test 
cases are outlined in Section \ref{sec_System} of this document. Test cases 
that are directly dependent on implementation details are outlined in an 
accompanying document, the Unit VnV Plan. All tests will be written (where 
applicable), reviewed, executed, and reported on by Brooks MacLachlan.

\subsection{Software Validation Plan}

There is no validation plan for \progname{}.

\section{System Test Description} \label{sec_System}

The values in Table~\ref{ExValidInputs} will be used as input for 
many of the test cases described throughout this section. These values were 
taken from the User's Guide for this project by \cite{UserGuide}. Individual 
test cases will reference the table as input but specify new values for any 
input parameter that should have a different value than specified by the table.

\begin{table}[!h]
	\renewcommand{\arraystretch}{1.5}
	\begin{tabularx}{1.0\textwidth}{p{7cm} l X}
		\toprule \textbf{Input} &
		\textbf{Unit} & \textbf{Value}\\ \midrule
		$\{\left(x_\text{us},y_\text{us}\right)\}$ & $\text{m}$ & \{(0, 25), 
		(20, 25), (30, 20), (40, 15), (70, 15)\}\\
		$\{\left(x_\text{wt},y_\text{wt}\right)\}$ & $\text{m}$ & \{(0, 22), 
		(10.87, 21.28), (21.14, 19.68), (31.21, 17.17), (38.69, 14.56), (40, 
		14), (70, 14)\}\\
		${x_\text{slip}^\text{minStart}}$ & $\text{m}$ & 10\\
		${x_\text{slip}^\text{maxStart}}$ & $\text{m}$ & 24\\
		${x_\text{slip}^\text{minEnd}}$ & $\text{m}$ & 34\\
		${x_\text{slip}^\text{maxEnd}}$ & $\text{m}$ & 53\\
		${y_\text{slip}^\text{min}}$ & $\text{m}$ & 5\\
		${y_\text{slip}^\text{max}}$ & $\text{m}$ & 26\\
		$c'$ & $\si{\pascal}$ & 5000 \\
		$\varphi'$ & \si{\degree} & 20\\
		$\gamma$ & $\si{\newton\per\meter\cubed}$ & 15000 \\
		$\gamma_{\text{Sat}}$ & $\si{\newton\per\meter\cubed}$ & 15000 \\
		$\gamma_{\text{w}}$ & $\si{\newton\per\meter\cubed}$ & 9800 \\
		Boolean representing the form of the interslice variation function $f$. 
		1 means $f$ is constant, 0 means $f$ is a half-sine & N/A & 0\\ 
		\bottomrule
	\end{tabularx}
	\caption{Input to be used for test cases}
	\label{ExValidInputs}
\end{table}
	
\subsection{Tests for Functional Requirements}

\subsubsection{User Input Tests}
		
\paragraph{Valid User Input}

~\newline \noindent The following set of test cases is intended to cover 
different forms of valid user input. Valid user input includes: slopes with or 
without a water table, slopes with one layer or multiple layers, slopes that 
increase or decrease as $x$ increases, slopes described by the minimum number 
of points, 2, or more than the minimum number of points, and slopes solved with 
$f$ as a constant or as a half-sine.

\begin{enumerate}[label=TC\arabic*:,ref={\arabic*}]

\item [TC\refstepcounter{testnum}\thetestnum: \label{TC_ValidInDec}] 
test-valid\_input\_decreasing

Control: Automatic
					
Initial State: New session
					
Input: As described in Table~\ref{ExValidInputs}.
					
Output: \progname{} runs to completion with no errors

How test will be performed: Automated test on unit testing framework
					
\item [TC\refstepcounter{testnum}\thetestnum: \label{TC_ValidInInc}] 
test-valid\_input\_increasing

Control: Automatic

Initial State: New session

Input: As described in Table~\ref{ExValidInputs}, except with slope coordinates 
$\{\left(x_\text{us},y_\text{us}\right)\}$ increasing as $x$ increases, as 
follows: \{(0, 
15), (30, 15), (40, 20), (50, 25), (70, 25)\}.

Output: \progname{} runs to completion with no errors

How test will be performed: Automated test on unit testing framework

\item [TC\refstepcounter{testnum}\thetestnum: \label{TC_ValidInMultiple}] 
test-valid\_input\_multiple

Control: Automatic

Initial State: New session

Input: As described in Table~\ref{ExValidInputs}, except with an additional 
slope layer described by the following coordinates: \{(0, 20), (36, 17), (40, 
15), (70, 15)\}.

Output: \progname{} runs to completion with no errors

How test will be performed: Automated test on unit testing framework

\item [TC\refstepcounter{testnum}\thetestnum: \label{TC_ValidInNoWT}] 
test-valid\_input\_noWT

Control: Automatic

Initial State: New session

Input: As described in Table~\ref{ExValidInputs}, except with no water table 
vertices $\{\left(x_\text{wt},y_\text{wt}\right)\}$.

Output: \progname{} runs to completion with no errors

How test will be performed: Automated test on unit testing framework

\item [TC\refstepcounter{testnum}\thetestnum: \label{TC_ValidInMin}] 
test-valid\_input\_minimal

Control: Automatic

Initial State: New session

Input: As described in Table~\ref{ExValidInputs}, except with  except with only 
2 slope coordinates $\{\left(x_\text{us},y_\text{us}\right)\}$, as follows: 
\{(0, 
25), (20, 25). Also, with no water table vertices 
$\{\left(x_\text{wt},y_\text{wt}\right)\}$.

Output: \progname{} runs to completion with no errors

How test will be performed: Automated test on unit testing framework

\item [TC\refstepcounter{testnum}\thetestnum: \label{TC_ValidInConstant}] 
test-valid\_input\_fConstant

Control: Automatic

Initial State: New session

Input: As described in Table~\ref{ExValidInputs}, except with $f$ as a 
constant. This is accomplished by setting the Boolean-type input to 1.

Output: \progname{} runs to completion with no errors

How test will be performed: Automated test on unit testing framework

\item [TC\refstepcounter{testnum}\thetestnum: \label{TC_ValidInXMinStartEqual}] 
test-valid\_input\_xMinStartEqual

Control: Automatic

Initial State: New session

Input: As described in Table~\ref{ExValidInputs}, except with 
${x_\text{slip}^\text{minStart}}$ set to 0.

Output: \progname{} runs to completion with no errors

How test will be performed: Automated test on unit testing framework

\item [TC\refstepcounter{testnum}\thetestnum: \label{TC_ValidInXMaxEndEqual}] 
test-valid\_input\_xMaxEndEqual

Control: Automatic

Initial State: New session

Input: As described in Table~\ref{ExValidInputs}, except with 
${x_\text{slip}^\text{maxEnd}}$ set to 70.

Output: \progname{} runs to completion with no errors

How test will be performed: Automated test on unit testing framework

\end{enumerate}

\paragraph{Invalid User Input}

~\newline \noindent The test cases described in Table~\ref{FaultyInputTests} 
are intended to cover all invalid input possibilities. Invalid input is input 
that defies the data constraints described in Section 
\ref{SRS-sec_DataConstraints} of the SRS. These test cases are identical to 
each other with the exception of their input. The input for each is specified 
in Table~\ref{FaultyInputTests}. For each test case, the inputs not specified 
in this table are specified in Table~\ref{ExValidInputs}. The control method 
for these test cases is automatic. The initial state for each is a new session. 
The expected output is the generation of an exception. The tests will be 
performed as automated tests on a unit testing framework.

\begin{longtable}{  l  p{5cm}  l  p{5cm}  }
	\hline
	\textbf{Test Case} & \textbf{Test Name} & \textbf{Input} & \textbf{Value} \\
	\hline
	TC\refstepcounter{testnum}\thetestnum \label{TC_InvalidSlopeDecToInc} & 
	test-invalid\_slope\_decToInc & 
	$\{\left(x_\text{us},y_\text{us}\right)\}$ & \{(0, 25), (20, 25), (30, 20), 
	(40, 25), 
	(70, 25)\}\\ 
	\hline
	TC\refstepcounter{testnum}\thetestnum \label{TC_InvalidSlopeIncToDec} & 
	test-invalid\_slope\_incToDec & 
	$\{\left(x_\text{us},y_\text{us}\right)\}$ & \{(0, 15), (20, 15), (30, 20), 
	(40, 15), 
	(70, 15)\}\\ 
	\hline
	TC\refstepcounter{testnum}\thetestnum \label{TC_InvalidSlopeDiffStart} & 
	test-invalid\_slope\_diffStart & 
	$\{\left(x_\text{us},y_\text{us}\right)\}$ & \{(0, 15), (20, 15), (30, 20), 
	(40, 25), 
	(70, 25)\}, \{(10, 15), (20, 15), (30, 20), (40, 25), (70, 25)\}\\ 
	\hline
	TC\refstepcounter{testnum}\thetestnum \label{TC_InvalidSlopeDiffEnd} & 
	test-invalid\_slope\_diffEnd & 
	$\{\left(x_\text{us},y_\text{us}\right)\}$ & \{(0, 15), (20, 15), (30, 20), 
	(40, 25), 
	(70, 25)\}, \{(10, 15), (20, 15), (30, 20), (40, 25), (60, 25)\}\\ 
	\hline
	TC\refstepcounter{testnum}\thetestnum \label{TC_InvalidSlopeOnePt} & 
	test-invalid\_slope\_onePt & 
	$\{\left(x_\text{us},y_\text{us}\right)\}$ & \{(0, 15)\}\\ 
	\hline
	TC\refstepcounter{testnum}\thetestnum \label{TC_InvalidSlopeDiffStartWT} & 
	test-invalid\_slope\_diffStartWT & 
	$\{\left(x_\text{wt},y_\text{wt}\right)\}$ & \{(10, 22), (10.87, 21.28), 
	(21.14, 
	19.68), (31.21, 17.17), (38.69, 14.56), (40, 14), (70, 14)\}\\ 
	\hline
	TC\refstepcounter{testnum}\thetestnum \label{TC_InvalidSlopeDiffEndWT} & 
	test-invalid\_slope\_diffEndWT & 
	$\{\left(x_\text{wt},y_\text{wt}\right)\}$ & \{(0, 22), (10.87, 21.28), 
	(21.14, 
	19.68), (31.21, 17.17), (38.69, 14.56), (40, 14), (60, 14)\}\\ 
	\hline
	TC\refstepcounter{testnum}\thetestnum \label{TC_InvalidSlopeOnePtWT} & 
	test-invalid\_slope\_onePtWT & 
	$\{\left(x_\text{wt},y_\text{wt}\right)\}$ & \{(0, 22)\}\\ 
	\hline
	TC\refstepcounter{testnum}\thetestnum \label{TC_InvalidSlipXMinStart} & 
	test-invalid\_slip\_xMinStart & ${x_\text{slip}^\text{minStart}}$ & -5\\ 
	\hline
	TC\refstepcounter{testnum}\thetestnum \label{TC_InvalidSlipXMaxStart} & 
	test-invalid\_slip\_xMaxStart & ${x_\text{slip}^\text{maxStart}}$ & 5\\ 
	\hline
	TC\refstepcounter{testnum}\thetestnum \label{TC_InvalidSlipXMinEnd} & 
	test-invalid\_slip\_xMinEnd & ${x_\text{slip}^\text{minStart}}$ & 20\\ 
	\hline
	TC\refstepcounter{testnum}\thetestnum \label{TC_InvalidSlipXMaxEnd} & 
	test-invalid\_slip\_xMaxEnd & ${x_\text{slip}^\text{minStart}}$ & 30\\ 
	\hline
	TC\refstepcounter{testnum}\thetestnum \label{TC_InvalidSlipXMaxEndOut} & 
	test-invalid\_slip\_xMaxEndOut & ${x_\text{slip}^\text{minStart}}$ & 75\\ 
	\hline
	TC\refstepcounter{testnum}\thetestnum \label{TC_InvalidSlipYMin} & 
	test-invalid\_slip\_yMin & ${y_\text{slip}^\text{min}}$ & 30\\ 
	\hline
	TC\refstepcounter{testnum}\thetestnum \label{TC_InvalidSlipYMax} & 
	test-invalid\_slip\_yMax & ${y_\text{slip}^\text{min}}$ & 0\\ 
	\hline
	TC\refstepcounter{testnum}\thetestnum \label{TC_InvalidSlipYEqual} & 
	test-invalid\_slip\_yEqual & ${y_\text{slip}^\text{min}}$ & 5\\ 
	\hline
	TC\refstepcounter{testnum}\thetestnum \label{TC_InvalidCohesion0} & 
	test-invalid\_cohesion\_0 & $c'$ & 0\\ 
	\hline
	TC\refstepcounter{testnum}\thetestnum \label{TC_InvalidCohesionNegative} & 
	test-invalid\_cohesion\_negative & $c'$ & -5\\ 
	\hline
	TC\refstepcounter{testnum}\thetestnum \label{TC_InvalidAngFric0} & 
	test-invalid\_angFric\_0 & $\varphi'$ & 0\\ 
	\hline
	TC\refstepcounter{testnum}\thetestnum \label{TC_InvalidAngFricNegative} & 
	test-invalid\_angFric\_negative & $\varphi'$ & -5\\ 
	\hline
	TC\refstepcounter{testnum}\thetestnum \label{TC_InvalidAngFric90} & 
	test-invalid\_angFric\_90 & $\varphi'$ & 90\\ 
	\hline
	TC\refstepcounter{testnum}\thetestnum \label{TC_InvalidAngFricObtuse} & 
	test-invalid\_angFric\_obtuse & $\varphi'$ & 100\\ 
	\hline
	TC\refstepcounter{testnum}\thetestnum \label{TC_InvalidUnitWt0} & 
	test-invalid\_unitWt\_0 & $\gamma$ & 0\\ 
	\hline
	TC\refstepcounter{testnum}\thetestnum \label{TC_InvalidUnitWtNegative} & 
	test-invalid\_unitWt\_negative & $\gamma$ & -5\\ 
	\hline
	TC\refstepcounter{testnum}\thetestnum \label{TC_InvalidUnitWtSat0} & 
	test-invalid\_unitWtSat\_0 & $\gamma_\text{Sat}$ & 0\\ 
	\hline
	TC\refstepcounter{testnum}\thetestnum \label{TC_InvalidUnitWtSatNegative} & 
	test-invalid\_unitWtSat\_negative & $\gamma_\text{Sat}$ & -5\\ 
	\hline
	TC\refstepcounter{testnum}\thetestnum \label{TC_InvalidUnitWtWater0} & 
	test-invalid\_unitWtWater\_0 & $\gamma_\text{w}$ & 0\\ 
	\hline
	TC\refstepcounter{testnum}\thetestnum \label{TC_InvalidUnitWtWaterNegative} 
	& test-invalid\_unitWtWater\_negative & $\gamma_\text{w}$ & -5\\ 
	\hline
	\caption{Faulty Input Test Cases}
	\label{FaultyInputTests}
\end{longtable}

\subsubsection{Calculation Tests}

\subsubsection{Output Tests}

\subsection{Tests for Nonfunctional Requirements}

\subsubsection{Correctness tests}
		
\paragraph{Title for Test}

\begin{enumerate}

\item{test-id1\\}

Type: 
					
Initial State: 
					
Input/Condition: 
					
Output/Result: 
					
How test will be performed: 
					
\item{test-id2\\}

Type: Functional, Dynamic, Manual, Static etc.
					
Initial State: 
					
Input: 
					
Output: 
					
How test will be performed: 

\end{enumerate}

\subsubsection{Area of Testing2}

...

\subsection{Traceability Between Test Cases and Requirements}

\noindent The purpose of the traceability matrix shown in 
Table~\ref{Table:T_trace} is to provide easy 
references on which requirements are verified by which test cases, and which 
test cases need to be updated if a requirement changes.  If a requirement is 
changed, the items in the column of that requirement that are marked
with an ``X'' may have to be modified as well. 

\begin{table}[!h]
	\centering
	\begin{tabular}{|c|c|c|c|c|c|c|c|c|c|c|}
		\hline
		& \rref{SRS-R_Inputs}& \rref{SRS-R_KinAdm}& \rref{SRS-R_InitGen}& 
		\rref{SRS-R_FS}& \rref{SRS-R_Minimize} & \rref{SRS-R_VerifyOutput}& 
		\rref{SRS-R_CritGraph}& \rref{SRS-R_OutputFS}& 
		\rref{SRS-R_NormalGraph}& \rref{SRS-R_ShearGraph} \\
		\hline
		\tcref{TC_ValidInInc}       & & & & & & & & & & \\ \hline
		\hline
	\end{tabular}
	\caption{Traceability Matrix Showing the Connections Between Requirements 
	and Test Cases}
	\label{Table:T_trace}
\end{table}

\section{Static Verification Techniques} \label{sec_Static}

The reviews described in Section \ref{sec_ImpPlan} will employ the static 
verification techniques of code walkthroughs and code inspection to attempt to 
uncover issues with the implementation.
				
\newpage

\bibliographystyle {plainnat}
\bibliography {../../../refs/References}

\newpage

\section{Appendix}

This is where you can place additional information.

\subsection{Symbolic Parameters}

The definition of the test cases will call for SYMBOLIC\_CONSTANTS.
Their values are defined in this section for easy maintenance.

\subsection{Usability Survey Questions?}

\wss{This is a section that would be appropriate for some projects.}

\end{document}