
\documentclass[12pt]{article}

\usepackage{tabularx}
\usepackage{booktabs}

\title{Problem Statement\\Slope Stability Analysis}
\author{Henry Frankis and Brooks MacLachlan}
\date{\today}

\begin{document}

\maketitle

\begin{table}[hp]
	\caption{Revision History} \label{TblRevisionHistory}
	\begin{tabularx}{\textwidth}{llX}
		\toprule
		\textbf{Date} & \textbf{Developer(s)} & \textbf{Change}\\
		\midrule
		09/13/18 & B. MacLachlan & Formatting updates, error fixes, and content updates related to changed scope of outputs\\
		\bottomrule
	\end{tabularx}
\end{table}

Slope stability analysis is the assessment of the safety of a slope. The Slope Stability Program (SSP) is intended to be
an introductory educational tool for demonstration
of slope stability issues, slope stability analysis 
software, and the process of assessing and
designing stable slopes to students at an undergraduate
university level. The stakeholders include students or anyone who designs excavated slopes or assesses the safety of natural or existing slopes. The SSP program should perform stability analysis of a slope under any and all of the
following conditions of:

\begin{itemize}
\item {a slope composed of multiple homogeneous layers of soil each with individual properties and specified geometry under the influence of gravity,}
\item {a water table interacting with the layers of the soil introducing a mix of dry and saturated soil conditions}
\end{itemize}

Analysis will evaluate a stability metric, called a factor of safety,
as an indicator of the stability of the slip surface under
investigation. The critical slip
surface with the lowest stability metric should be identified, and its factor of safety reported.

The program should run in Windows 10, Mac OS X 10.13, and Ubuntu 18.04 environments.

\end{document}