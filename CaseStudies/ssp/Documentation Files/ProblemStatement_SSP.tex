
\documentclass[12pt]{article}

\usepackage{fullpage}

\begin{document}

\title{Slope Stability Analysis}
\author{Henry Frankis\\ McMaster University\\ Document Driven Design}
\date{\today}
\maketitle

\section*{Problem Statement}

\begin{par} \indent \indent

The Slope Stability Program (SSP) is intended to be
an introductory educational tool for demonstration
of slope stability issues, slope stability analysis 
software, and the process of assessing and
designing stable slopes to students at an undergrad
university level. The SSP program is to exhibit the following
characteristics.  A program that can perform stability
analysis of a slope under any and all of the
following conditions of:

\begin{itemize}
\item {a slope composed of multiple heterogeneous layers of soil each with individual properties,}
\item {slope/layers of any specified geometry,}
\item {a water table interacting with the layers of the soil introducing a mix of dry and saturated soil conditions,}
\end{itemize}

Analysis will evaluate a stability metric, \textit{Factor of Safety},
as an indicator of the stability of the slip surface under
investigation. Employ an efficient method to find the critical slip
surface with the lowest stability metric, including slip surfaces
of non circular geometry. For the critical slip surface evaluate 
the displacement of the slope and the local stability
metric along the critical slip surface.

\end{par}

% ---------------------------------------------- %
% References
% Code - 


\end{document}