\documentclass[12pt]{article}
\usepackage{multirow}
\usepackage{caption}
\usepackage{bm}
\usepackage{amsmath}
\usepackage{amsfonts}
\usepackage{amssymb}
\usepackage{graphicx}
\usepackage{colortbl}
\usepackage{xr}
\usepackage{hyperref}
\usepackage{pdfpage}
\usepackage[all]{hypcap} 
\usepackage{longtable}
\usepackage{xfrac}
\usepackage{tabularx}
\usepackage{float}
\usepackage{siunitx}
\usepackage{booktabs}
\usepackage[toc, page]{appendix}
\usepackage{url}
\usepackage[usenames,dvipsnames]{xcolor}
\usepackage{array}
%\usepackage{refcheck}

\hypersetup{
      colorlinks=true,       % false: boxed links; true: coloured links
      linkcolor=red,          % colour of internal links (change box colour with
                              % %linkbordercolor)
    citecolor=blue,        % colour of links to bibliography
    filecolor=magenta,      % colour of file links
    urlcolor=cyan           % colour of external links
}

%% Comments
\newif\ifcomments\commentstrue

\ifcomments
\newcommand{\authornote}[3]{\textcolor{#1}{[#3 ---#2]}}
\newcommand{\todo}[1]{\textcolor{red}{[TODO: #1]}}
\else
\newcommand{\authornote}[3]{}
\newcommand{\todo}[1]{}
\fi

\newcommand{\wss}[1]{\authornote{magenta}{SS}{#1}}
\newcommand{\nk}[1]{\authornote{blue}{NK}{#1}}

\newcommand{\InputFormat}{\hyperref[SecInF]{Input Format Module}}
\newcommand{\InputParam}{\hyperref[SecInP]{Input Parameters Module}}
\newcommand{\InputCons}{\hyperref[SecInC]{Input Constraints Module}}
\newcommand{\DerivedVal}{\hyperref[SecDeV]{Derived Values Module}}
\newcommand{\OutputFormat}{\hyperref[SecOutF]{Output Format Module}}
\newcommand{\Calculations}{\hyperref[SecCalc]{Calculations Module}}
\newcommand{\Control}{\hyperref[SecCtrl]{Control Module}}
\newcommand{\InterpData}{\hyperref[SecInterpD]{Interpolation Data Module}}
\newcommand{\Interp}{\hyperref[SecInterp]{Interpolation Module}}

%\newommand{\colZwidth}{1.0\textwidth} 
%\newcommand{\blt}{- } 
%\newcommand{\colAwidth}{0.13\textwidth}
%\newcommand{\colBwidth}{0.82\textwidth}
%\newcommand{\colCwidth}{0.1\textwidth}
%\newcommand{\colDwidth}{0.05\textwidth}
%\newcommand{\colEwidth}{0.8\textwidth}
%\newcommand{\colFwidth}{0.17\textwidth}
%\newcommand{\colGwidth}{0.5\textwidth}
%\newcommand{\colHwidth}{0.28\textwidth}
%\newcommand{\colDescrip}{0.7\textwidth}

\newcommand{\progname}[1]{Glass-BR}

\externaldocument[SRS-]{../SRS/glassbr_srs}
\externaldocument[MG-]{../MG/glassbr_mg}


\begin{document}

\title{Module Interface Specification for \progname} 
\author{Jingwei Huang}
\date{\today}

\maketitle

\tableofcontents

\newpage

\section{Introduction}

The following document details the Module Interface
Specifications for the implemented modules in a program \progname{}. It is
intended to ease navigation through the program for design and
maintenance purposes.  Complementary documents include the 
\href{../SRS/glassbr_srs.pdf}{System Requirement Specifications} (SRS) and 
\href{../MG/glassbr_mg.pdf}{Module Guide} (MG).

\section{Notation}

\progname{} uses three primitive data types: Boolean, Integer and Float. 
\progname{} also uses derived data types: Array, String, Tuple and File.
These data types are summarized in the following table. The table lists the name of the data type, 
its notation in this document, and a description of the data type.
\newline

\noindent \begin{tabular}{l l p{0.6\textwidth}} \hline
\textbf{Data Type} & \textbf{Notation} & \textbf{Description}\\ \hline
Boolean & $boolean$ & an element of \{true, false\}\\ \hline
Integer & $integer$ & a number without a fractional component in (-$\infty$, $\infty$)\\ \hline
Float & $float$ & any number in (-$\infty$, $\infty$) \\ \hline
Array & $array$ & a collection of data items of the same kind \\ \hline
String & $string$ & a varying length array of characters \\ \hline
Tuple & $tuple$ & a collection of elements of possibly different types \\ \hline
File & $FILE{}\ast$ & a pointer to an input or output file \\ \hline
\end{tabular} 

\section{Module Hierarchy} 

To view the Module Hierarchy, see section 3 of the \href{../MG/glassbr_mg.pdf}{MG}.

%%%%%%%%%%%%%%%%%%%%%%%%%%%%%%%%%%%%%%%%%%%
\section{MIS of the Input Format Module} \label{SecInF}

\subsection{Module Name: Input Format}

\subsection{Uses}

\subsubsection{Imported Constants}

None

\subsubsection{Imported Variables}

$params$

\subsubsection{Imported Data Types}

Param: tuple 

\subsubsection{Imported Access Programs}

None

\subsection{Interface Syntax}

\subsubsection{Exported Constants}

None

\subsubsection{Exported Variables}

None

\subsubsection{Exported Data Types}

None

\subsubsection{Exported Access Programs}

\begin{center}
\begin{tabular}{l l l l}\hline 
\textbf{Routine Name} & \textbf{In} &\textbf{Out} & \textbf{Exceptions} \\ \hline 
$s\_get\_input$ & FILE $\ast$, tuple & - & badFile $\vee$ badFormat \\ \hline
\end{tabular}
\end{center}

\subsection{Interface Semantics}

\subsubsection{Environment Variables}

$filename$: FILE $\ast$ (input file)

\subsubsection{State Variables}

$params.a$, $params.b$, $params.t$, $params.w$, $params.tnt$,
$sd_{x}$, $sd_{y}$, $sd_{z}$, \newline $params.pb_{tol}$: float \\
$params.gt$: string \\
$params.sdvect$: tuple of ($sd_{x}$: float, $sd_{y}$: float, $sd_{z}$: float)\\

\subsubsection{State Invariant}

None

\subsubsection{Assumption}

The user input values are properly constrained and the data structure Param has been initialized.

\subsubsection{Access Program Semantics}

\begin{itemize}
\item\textbf{$s\_get\_input$($filename$, $params$):} \\ \newline
\begin{tabular}{l p{0.70\textwidth}}
\textbf{Transition:} & $params.a$, $params.b$, $params.t$, $params.gt$, 
$params.w$, $params.tnt$, $sd_{x}$, $sd_{y}$, $sd_{z}$, $params.sdvect$,
$params.pb_{tol}$ := $a$, $b$, $t$, $gt$, $w$, $tnt$, $sd_{x}$, $sd_{y}$, 
$sd_{z}$, ($sd_{x}$, $sd_{y}$, $sd_{z}$), $pb_{tol}$ stored in the input file  \\
\textbf{Exceptions:} & $exc$ := \newline
(error reading file $\Rightarrow$ badFile \newline $|$ inconsistent input format $\Rightarrow$ badFormat)\\
\end{tabular}
\end{itemize}

\subsubsection{Considerations}

The data type of $params.t$ has been mutated from String to Float.

%%%%%%%%%%%%%%%%%%%%%%%%%%%%%%%%%%%%%%%%%

\section{MIS of the Input Parameters Module} \label{SecInP}

\subsection{Module Name: Input Parameters}

\subsection{Uses}
N/A
%\subsubsection{Imported Constants}

%\subsubsection{Imported Variables}

%\subsubsection{Imported Data Types}

%\subsubsection{Imported Access Programs}
\subsection{Interface Syntax}

\subsubsection{Exported Constants}

\#define $params.E$ $7.17 \times 10^{7}$ \\
\#define $params.t_{d}$ $3$ \\
\#define $params.m$ $7$ \\
\#define $params.k$ $2.86 \times 10^{-53}$ \\
\#define $params.lsf$ $1$

\subsubsection{Exported Variables}

$params$

\subsubsection{Exported Data Types}

Param: tuple
\newline

\noindent The parameters for this structure correspond to 
the state variables of this module, listed in section 
\ref{SecSVparams}.
  
\subsubsection{Exported Access Programs} \label{SecEAPparams}

\begin{center} \noindent
\begin{tabular}{l l l l}
\hline \textbf{Routine Name} & \textbf{In} & \textbf{Out} & \textbf{Exceptions} \\ \hline 
$s\_params.a$&float&-&- \\ \hline
$g\_params.a$&-&float&- \\ \hline
$s\_params.b$&float&-&- \\ \hline
$g\_params.b$&-&float&- \\ \hline
$s\_params.t$&string&-&- \\ \hline
$g\_params.t$&-&string&- \\ \hline
$s\_params.gt$&string&-&- \\ \hline
$g\_params.gt$&-&string&- \\ \hline
$s\_params.w$&float&-&- \\ \hline
$g\_params.w$&-&float&- \\ \hline
$s\_params.tnt$&float&-&- \\ \hline
$g\_params.tnt$&-&float&-\\ \hline
$s\_params.sdvect$&tuple&-&- \\ \hline
$g\_params.sdvect$&-&tuple&- \\ \hline
$s\_params.pb_{tol}$&float&-&- \\ \hline
$g\_params.pb_{tol}$&-&float&- \\ \hline
$s\_params.asprat$&float&-&- \\ \hline
$g\_params.asprat$&-&float&- \\ \hline
$s\_params.sd$&float&-&- \\ \hline
$g\_params.sd$&-&float&- \\ \hline
$s\_params.h$&float&-&- \\ \hline
$g\_params.h$&-&float&- \\ \hline
$s\_params.gtf$&float&-&- \\ \hline
$g\_params.gtf$&-&float&- \\ \hline
$s\_params.ldf$&float&-&- \\ \hline
$g\_params.ldf$&-&float&- \\ \hline
$s\_params.wtnt$&float&-&- \\ \hline
$g\_params.wtnt$&-&float&- \\ \hline

\end{tabular}
\end{center}

\subsection{Interface Semantics}

\subsubsection{Environment Variables} 

None

\subsubsection{State Variables} \label{SecSVparams}

$a$, $b$, $w$, $tnt$, $pb_{tol}$, $asprat$, $sd$, $h$, $gtf$, $ldf$, $wtnt$: float \\
$t$, $gt$: string \\
$sdvect$: tuple of ($sd_{x}$: float, $sd_{y}$: float, $sd_{z}$: float)

\subsubsection{State Invariant}

None

\subsubsection{Assumption}

Before a get function is used, the necessary set functions have been called.

\subsubsection{Access Program Semantics}

\begin{itemize}
\item\textbf{$s\_params.\ast$ :} \\ \newline
\begin{tabular}{l p{0.7\textwidth}}
\textbf{Transition:} & $params.a$, $params.b$, $params.t$, $params.gt$, $params.w$,
$params.tnt$, $params.sdvect$, $params.pb_{tol}$, $params.asprat$, $params.sd$,
$params.h$, $params.gtf$, $params.ldf$, $params.wtnt$ := 0.0, 0.0, ``2.5", ``AN", 0.0, 0.0, (0.0, 0.0, 0.0),
0.0, 0.0, 0.0, 0.0, 0.0, 0.0, 0.0 \\
\textbf{Exceptions:} & none \\
\end{tabular}

\item\textbf{$g\_params.\ast$:} \\ \newline
\begin{tabular}{l p{0.7\textwidth}}
\textbf{Output:} & $out$ := $a$, $b$, $t$, $gt$, $w$, $tnt$, $sdvect$, $pb_{tol}$, $asprat$, 
$sd$, $h$, $gtf$, $ldf$, $wtnt$ stored in the data structure Param.\\
\textbf{Exceptions:} & none\\
\end{tabular}
\end{itemize}

\subsubsection{Considerations}
Param is a data structure designed to store the input information entered by 
the \InputFormat{} and \DerivedVal. 

%%%%%%%%%%%%%%%%%%%%%%%%%%%%%%%%%%%%%%%%

\section{MIS of the Input Constraints Module} \label{SecInC}

\subsection{Module Name: Input Constraints}

\subsection{Uses}

\subsubsection{Imported Constants}

None

\subsubsection{Imported Variables}

$params$

\subsubsection{Imported Data Types}

Param: tuple

\subsubsection{Imported Access Programs}

None

\subsection{Interface Syntax}

\subsubsection{Exported Constants}

None

\subsubsection{Exported Variables}

None

\subsubsection{Exported Data Types}

None

\subsubsection{Exported Access Programs}

\begin{center}
\begin{tabular}{l l l p{0.5\textwidth}}\hline 
\textbf{Routine Name} & \textbf{In} &\textbf{Out} & \textbf{Exceptions} \\ \hline 
$s\_check\_constraints$ & tuple & - & badLength $\vee$ badWidth $\vee$ badAspectRatio 
$\vee$ badThickness $\vee$ badTNT $\vee$ badWTNT $\vee$ badSD \\ \hline
\end{tabular}
\end{center}

\subsection{Interface Semantics}

\subsubsection{Environment Variables}

$scn$ : the terminal screen

\subsubsection{State Variables}

$params.a$, $params.b$, $params.asprat$, $params.t$, $params.tnt$, $params.wtnt$, $params.sd$: float

\subsubsection{State Invariant}

$params.a > 0$ \\
$params.b > 0$ \\
$params.asprat \geq 1$ $\wedge$ $params.asprat \leq 5$ \\
$params.t \in \{2.5, 2.7, 3.0, 4.0, 5.0, 6.0, 8.0, 10.0, 12.0, 16.0, 19.0, 22.0\}$ \\
$params.tnt > 0$ \\
$params.wtnt \geq 4.5$ $\wedge$ $params.wtnt \leq 910$ \\
$params.sd \geq 6$ $\wedge$ $params.sd \leq 130$

\subsubsection{Assumption}

The function get\_input in the \InputFormat{} has been called and the data in
the input file have been stored in the data structure Param.

\subsubsection{Access Program Semantics}

\begin{itemize}
\item\textbf{$s\_check\_constraints$($params$):}  \\ \newline
\begin{tabular}{l p{0.7\textwidth}}
\textbf{Transition:} & display the exceptions on $scn$\\
\textbf{Exceptions:} & $exc$ := \newline ($params.a \leq 0$ $\vee$ $params.b \leq 0$ $\Rightarrow$ badLength or badWidth 
\newline $|$ $params.asprat < 1$ $\vee$ $params.asprat > 5$ $\Rightarrow$ badAspectRatio \newline 
$|$ $params.t \notin \{2.5, 2.7, 3.0, 4.0, 5.0, 8.0, 10.0, 12.0, 16.0,\newline 19.0, 22.0\}$ $\Rightarrow$ badThickness
\newline $|$ $params.tnt \leq 0$ $\Rightarrow$ badTNT \newline
$|$ $params.wtnt < 4.5$ $\vee$ $params.wtnt > 910$ $\Rightarrow$ badWTNT \newline
$|$ $params.sd < 6$ $\vee$ $params.sd > 130$ $\Rightarrow$ badSD)\\
\end{tabular}
\end{itemize}

\subsubsection{Considerations}

The data type of $params.t$ has been mutated from String to Float.

%%%%%%%%%%%%%%%%%%%%%%%%%%%%%%%%%%%%%%%%%%%%%%%%%%

\section{MIS of Output Format Module} \label{SecOutF}

\subsection{Module Name: Output Format}

\subsection{Uses}

\subsubsection{Imported Constants}

None

\subsubsection{Imported Variables}

$params$, $q$, $j$, $\hat{q}_{tol}$, $pb$, $lr$, $nfl$, $is\_safe1$, $is\_safe2$, $safe$ 

\subsubsection{Imported Data Types}

Param: tuple \\
$q$, $j$, $\hat{q}_{tol}$, $pb$, $lr$, $nfl$: float \\
$is\_safe1$, $is\_safe2$: boolean \\
$safe$: string

\subsubsection{Imported Access Programs}

None

\subsection{Interface Syntax}

\subsubsection{Exported Constants}

None

\subsubsection{Exported Variables}

None

\subsubsection{Exported Data Types}

None

\subsubsection{Exported Access Programs}
\begin{center}
\begin{tabular}{l p{5cm} l l} \hline 
\textbf{Routine Name} & \textbf{In} &\textbf{Out} & \textbf{Exceptions} \\ \hline 
$s\_display\_output$ &string, float, float, float, \newline float, float, float, boolean, \newline boolean, string& - & badPath \\ \hline
\end{tabular}
\end{center}

\subsection{Interface Semantics}

\subsubsection{Environment Variables}

$filename$: FILE $\ast$ (output file)

\subsubsection{State Variables}

None

\subsubsection{State Invariant}

None

\subsubsection{Assumption}

The functions in the \Calculations{} have been called and 
the values for the imported variables have been calculated.

\subsubsection{Access Program Semantics}

\begin{itemize}
\item\textbf{$s\_display\_output$($filename$, $q$, $j$, $\hat{q}_{tol}$, $pb$, $lr$, $nrl$, $is\_safe1$, $is\_safe2$, $safe$, $params$):}  \\ \newline
\begin{tabular}{l p{0.7\textwidth}}
\textbf{Transition:} & display the outputs in the output file $filename$\\
\textbf{Exceptions:} & $exc$ := error writing file $\Rightarrow$ badPath \\
\end{tabular}
\end{itemize}

%%%%%%%%%%%%%%%%%%%%%%%%%%%%%%%%%%%%%%%%%%%%%%%%%%

\section{MIS of Derived Values Module} \label{SecDeV}

\subsection{Module Name: Derived Values}

\subsection{Uses}

\subsubsection{Imported Constants}

\#define $params.t_{d}$ $3$ \\
\#define $params.m$ $7$

\subsubsection{Imported Variables}

$params$

\subsubsection{Imported Data Types}

Param: tuple

\subsubsection{Imported Access Programs}

None

\subsection{Interface Syntax}

\subsubsection{Exported Constants}

\#define $params.ldf$ $0.2696493494752911$

\subsubsection{Exported Variables}

$params$

\subsubsection{Exported Data Types}

Param: tuple

\subsubsection{Exported Access Programs}
\begin{center}
\begin{tabular}{l l l p{0.5\textwidth}} \hline 
\textbf{Routine Name} & \textbf{In} &\textbf{Out} & \textbf{Exceptions} \\ \hline 
$sg\_derived\_params$ & tuple & tuple & badFormat $\vee$ notIndustrialStandard $\vee$ wrongGlassType\\ \hline
\end{tabular}
\end{center}

\subsection{Interface Semantics}

\subsubsection{Environment Variables}

$scn$: the terminal screen

\subsubsection{State Variables}

$params.asprat$, $params.sd$, $params.ldf$, $params.wtnt$, $params.h$, \newline $params.gtf$: float

\subsubsection{State Invariant}

None

\subsubsection{Assumption}

The function get\_input in the \InputFormat{} has been called and the data in 
the input file have been stored in the data structure Param.

\subsubsection{Access Program Semantics}

\begin{itemize}
\item\textbf{$sg\_derived\_params$($params$):}  \\ \newline
\begin{tabular}{p{2.5cm} p{0.7\textwidth}}
\textbf{Transition \newline \& Output:} & (display exceptions on $scn$ \newline $|$ 
$out$ := $params.asprat$, $params.sd$, $params.ldf$, $params.wtnt$, $params.h$, $params.gtf$ calculated using 
the functions defined in the \href{../SRS/glassbr_srs.pdf}{SRS}) \\
\textbf{Exceptions:} & $exc$ := \newline
(inconsistent format $\Rightarrow$ badFormat \newline 
$|$ $params.t \notin \{2.5, 2.7, 3.0, 4.0, 5.0, 6.0, 8.0, 10.0, 12.0, \newline16.0, 19.0, 22.0\}$ $\Rightarrow$ notIndustrialStandard
\newline $|$ $params.gt \notin \{``AN", ``an", ``HS", ``hs", ``FT", ``ft"\}$ $\Rightarrow$ wrongGlassType)\\
\end{tabular}
\end{itemize}

%%%%%%%%%%%%%%%%%%%%%%%%%%%%%%%%%%%%%%%%%%%%%%%%%

\section{MIS of Calculations Module} \label{SecCalc}

\subsection{Module Name: Calculations}

\subsection{Uses}

\subsubsection{Imported Constants}

\#define $E$ $7.17 \times 10^{7}$ \\
\#define $m$ $ 7$ \\
\#define $k$ $2.86 \times 10^{-53}$ \\
\#define $ldf$ $0.2696493494752911$ \\
\#define $lsf$ $1$

\subsubsection{Imported Variables}

$params$

\subsubsection{Imported Data Types}

Param: tuple 

\subsubsection{Imported Access Programs}

\textbf{Uses} \Interp{} \textbf{Imports} interp

\subsection{Interface Syntax}

\subsubsection{Exported Constants}

None

\subsubsection{Exported Variables}

$q$, $j$, $\hat{q}_{tol}$, $pb$, $lr$, $nfl$, $is\_safe1$, $is\_safe2$, $safe$

\subsubsection{Exported Data Types}

$q$, $j$, $\hat{q}_{tol}$, $pb$, $lr$, $nfl$: float \\
$is\_safe1$, $is\_safe2$: boolean \\
$safe$: string
 
\subsubsection{Exported Access Programs}
\begin{center}
\begin{tabular}{l p{3cm} l l} \hline 
\textbf{Routine Name} & \textbf{In} &\textbf{Out} & \textbf{Exceptions} \\ \hline 
$g\_calc\_q$ &array of floats, array of floats, array of floats, tuple& float & badFormat \\ \hline
$g\_calc\_j$ &array of floats, array of floats, array of floats, float, tuple& float, float & badFormat \\ \hline
$g\_calc\_pb$ &float, tuple& float & - \\ \hline
$g\_calc\_lr$ &float, tuple& float, float & - \\ \hline
$g\_is\_safe$ &float, float, float, tuple& boolean, boolean, string & - \\ \hline
\end{tabular}
\end{center}

\subsection{Interface Semantics}

\subsubsection{Environment Variables}

None

\subsubsection{State Variables}

$q$, $j$, $\hat{q}_{tol}$, $pb$, $lr$, $nfl$: float \\
$w\_array$, $data\_sd$, $data\_q$, $j\_array$, $data\_asprat$, $data\_qstar$: array of floats\\
$is\_safe1$, $is\_safe2$: boolean \\
$safe$: string

\subsubsection{State Invariant}

None

\subsubsection{Assumption}

The get\_input function in the \InputFormat{} and the derived\_params function
in the \DerivedVal{} have been called and the data in the input file as well as
the derived values have been stored in the data structure Param.
\noindent The \Interp{} has been successfully implemented.

\subsubsection{Access Program Semantics}

\begin{itemize}
\item\textbf{$g\_calc\_q$($w\_array$, $data\_sd$, $data\_q$, $params$):} \\ \newline
\begin{tabular}{l p{0.7\textwidth}}
\textbf{Output:} & $out$ := $q$ calculated using interpolation\\ 
\textbf{Exceptions:} & $exc$ := badFormat \\
\end{tabular}

\item\textbf{$g\_calc\_j$($j\_array$, $data\_asprat$, $data\_qstar$, $q$, $params$):} \\ \newline
\begin{tabular}{l p{0.7\textwidth}}
\textbf{Output:} & $out$ := $j$, $\hat{q}_{tol}$ calculated using 
interpolation and the functions defined in the \href{../SRS/glassbr_srs.pdf}{SRS} \\
\textbf{Exceptions:} & $exc$ := badFormat \\
\end{tabular}

\item\textbf{$g\_calc\_pb$($j$, $params$):} \\ \newline
\begin{tabular}{l p{0.7\textwidth}}
\textbf{Output:} & $out$ := $pb$ calculated using the functions defined in the \href{../SRS/glassbr_srs.pdf}{SRS} \\ 
\textbf{Exceptions:} & none \\ 
\end{tabular}

\item\textbf{$g\_calc\_lr$($\hat{q}_{tol}$, $params$):} \\ \newline
\begin{tabular}{l p{0.7\textwidth}}
\textbf{Output:} & $out$ := $lr$, $nfl$ calculated using the functions defined in the \href{../SRS/glassbr_srs.pdf}{SRS} \\ 
\textbf{Exceptions:} & none \\ 
\end{tabular}

\item\textbf{$g\_is\_safe$($pb$, $lr$, $q$, $params$):}  \\ \newline
\begin{tabular}{l p{0.7\textwidth}}
\textbf{Output:} & $out$ := \newline
($pb < params.pb_{tol}$ $\Rightarrow$ $is\_safe1 := True$ 
$|$ $pb \geq params.pb_{tol}$ $\Rightarrow$ $is\_safe1 := False$) \newline $|$
($lr > q$ $\Rightarrow$ $is\_safe2 := True$ $|$ $lr \leq q$ $\Rightarrow$ $is\_safe2 := False$)
\newline $|$ ($is\_safe1 == True$ $\wedge$ $is\_safe2 == True$ $\Rightarrow$ $safe$ := `For
the given input parameters, the glass is considered safe' $|$ $is\_safe1 == False$ $\vee$ 
$is\_safe2 == False$ $\Rightarrow$ $safe$ := `For the given input parameters, the glass is
NOT considered safe')\\ 
\textbf{Exceptions:} & none \\
\end{tabular}
\end{itemize}

%%%%%%%%%%%%%%%%%%%%%%%%%%%%%%%%%%%%%%%%%%%%%

\section{MIS of the Control Module} \label{SecCtrl}

\subsection{Module Name: Control}

\subsection{Uses}

\subsubsection{Imported Constants}

None

\subsubsection{Imported Variables}

None

\subsubsection{Imported Data Types}

None

\subsubsection{Imported Access Programs}

\textbf{Uses} \InputParam{} \textbf{Imports} param \\
\textbf{Uses} \InputFormat{} \textbf{Imports} inputFormat \\
\textbf{Uses} \DerivedVal{} \textbf{Imports} derivedValues \\
\textbf{Uses} \InputCons{} \textbf{Imports} checkConstraints \\
\textbf{Uses} \InterpData{} \textbf{Imports} readTable \\
\textbf{Uses} \Calculations{} \textbf{Imports} calculations \\
\textbf{Uses} \OutputFormat{} \textbf{Imports} outputFormat

\subsection{Interface Syntax}

\subsubsection{Exported Constants}

None

\subsubsection{Exported Variables}

None

\subsubsection{Exported Data Types}

None

\subsubsection{Exported Access Programs}
\begin{center}
\begin{tabular}{l l l l} \hline 
\textbf{Routine Name} & \textbf{In} & \textbf{Out} & \textbf{Exceptions} \\ \hline 
$s\_main$ & FILE $\ast$ & - & badFile $\vee$ badPath $\vee$ badFormat \\ \hline
\end{tabular}
\end{center}

\subsection{Interface Semantics}

\subsubsection{Environment Variables}

$filename$: FILE $\ast$ (input file and output file) \\
$scn$: the terminal screen

\subsubsection{State Variables}

None 

\subsubsection{State Invariant}

None

\subsubsection{Assumption}

The imported modules have been successfully implemented.

\subsubsection{Access Program Semantics}

\begin{itemize}
\item\textbf{$s\_main$($filename$):} \\ \newline
\begin{tabular}{l p{0.7\textwidth}} 
\textbf{Transition:} & 
(calculate the outputs  \newline $|$ display the results in the output file $filename$ \newline 
$|$ display the message ``Main has been executed and the results have been written to 
`outputfile'" on $scn$) \\
\textbf{Exceptions:} & $exc$ := \newline
(error reading file $\Rightarrow$ badFile \newline $|$ error writing file $\Rightarrow$ badPath 
\newline $|$ inconsistent input format $\Rightarrow$ badFormat)\\ \\
\end{tabular}
\end{itemize}

%%%%%%%%%%%%%%%%%%%%%%%%%%%%%%%%%%%%%%%%%%%%%%%%%%

\section{MIS of the Interpolation Data Module} \label{SecInterpD}

\subsection{Module Name: Interpolation Data}

\subsection{Uses}

\subsubsection{Imported Constants}

None

\subsubsection{Imported Variables}

None

\subsubsection{Imported Data Types}

None

\subsubsection{Imported Access Programs}

None

\subsection{Interface Syntax}

\subsubsection{Exported Constants}

None

\subsubsection{Exported Variables}

$num\_col$, $array1$, $array2$

\subsubsection{Exported Data Types}

$num\_col$, $array1$, $array2$: array of floats

\subsubsection{Exported Access Programs}

\begin{center}
\begin{tabular}{l l p{2.6cm} l}\hline 
\textbf{Routine Name} & \textbf{In} &\textbf{Out} & \textbf{Exceptions} \\ \hline 
$g\_read\_table$ & FILE $\ast$ & array of floats, array of floats, array of floats & badFile $\vee$ badFormat\\ \hline
\end{tabular}
\end{center}

\subsection{Interface Semantics}

\subsubsection{Environment Variables}

$filename$: FILE $\ast$ (input file)

\subsubsection{State Variables}

$num\_col$, $array1$, $array2$: array of floats\\

\subsubsection{State Invariant}

None

\subsubsection{Assumption}

The user input values are properly constrained.

\subsubsection{Access Program Semantics}

\begin{itemize}
\item\textbf{$g\_read\_table$($filename$):} \\ \newline
\begin{tabular}{l p{0.7\textwidth}}
\textbf{Output:} & $out$ := $num\_col$, $array1$, $array2$ \\
\textbf{Exceptions:} & $exc$ := \newline
(error reading file $\Rightarrow$ badFile \newline $|$ inconsistent input format $\Rightarrow$ badFormat)\\
\end{tabular}
\end{itemize}

%%%%%%%%%%%%%%%%%%%%%%%%%%%%%%%%%%%%%%%%%%%%%%%%%%

\section{MIS of the Interpolation Module} \label{SecInterp}

\subsection{Module Name: Interpolation}

\subsection{Uses}

N/A

\subsection{Interface Syntax}

\subsubsection{Exported Constants}

None

\subsubsection{Exported Variables}

$y_{0}$, $idx$, $jdx$, $kdx$, $num\_interp1$, $num\_interp2$, $interp\_value$

\subsubsection{Exported Data Types}

$y_{0}$, $interp\_value$: float \\
$idx$, $jdx$, $kdx$, $num\_interp1$, $num\_interp2$: integer 

\subsubsection{Exported Access Programs}

\begin{center}
\begin{tabular}{l p{2.8cm} p{3cm} l}\hline 
\textbf{Routine Name} & \textbf{In} &\textbf{Out} & \textbf{Exceptions} \\ \hline 
$g\_lin\_interp$ & float, float, float, float, float & float & - \\ \hline
$g\_find\_bounds$ & array of floats, array of floats, float, float & integer, integer, integer, integer, integer & - \\ \hline
$g\_interp$ & integer, integer, integer, integer, integer, array of floats, array of floats, array of floats, float, float & float & - \\ \hline
\end{tabular}
\end{center}

\subsection{Interface Semantics}

\subsubsection{Environment Variables}

None

\subsubsection{State Variables}

$y_{0}$, $y_{1}$, $y_{2}$, $x_{1}$, $x_{2}$, $input\_param$, $value1$, $value2$, $interp\_value$: float \\
$idx$, $jdx$, $kdx$, $num\_interp1$, $num\_interp2$: integer \\
$data1$, $data2$, $data3$: array of floats

\subsubsection{State Invariant}

$idx \geq 0$ \\
$jdx \geq 0$ \\
$kdx \geq 0$ \\
$num\_interp1 \in \{0, 1\}$ \\
$num\_interp2 \in \{0, 1, 2, 3\}$

\subsubsection{Assumption}

None

\subsubsection{Access Program Semantics}

\begin{itemize}
\item\textbf{$g\_lin\_interp$($y_{1}$, $y_{2}$, $x_{1}$, $x_{2}$, $input\_param$):} \\ \newline
\begin{tabular}{l p{0.7\textwidth}}
\textbf{Output:} & $out$ := $y_{0}$ calculated using the linear interpolation algorithm \\ 
\textbf{Exceptions:} & none\\
\end{tabular}

\item\textbf{$g\_find\_bounds$($data1$, $data2$, $value1$, $value2$):} \\ \newline
\begin{tabular}{l p{0.7\textwidth}}
\textbf{Output:} & $out$ := $idx$, $jdx$, $kdx$, $num\_interp1$, $num\_interp2$\\ 
\textbf{Exceptions:} & none\\ 
\end{tabular}

\item\textbf{$g\_interp$($idx$, $jdx$, $kdx$, $num\_interp1$, $num\_interp2$, $data1$, $data2$, $data3$, $value1$, $value2$):} \\ \newline
\begin{tabular}{l p{0.7\textwidth}}
\textbf{Output:} & $out$ := $interp\_value$\\ 
\textbf{Exceptions:} & none\\ 
\end{tabular}
\end{itemize}

\subsubsection{Local Functions}

\begin{itemize}
\item\textbf{$g\_proper\_index$($index1$, $index2$, $data$, $value$):} \\ \newline
\begin{tabular}{l p{0.7\textwidth}} 
\textbf{Output:} & $out$ := $index1$\\
\textbf{Exceptions:} & none\\
\end{tabular}
\end{itemize}

\subsubsection{Local Data Types}

\begin{tabular}{l p{0.7\textwidth}}
\textbf{Imported:} & $index1$, $index2$: integer \newline $data$: array of floats \newline $value$: float \\ \\
\textbf{Exported:} & $index1$: integer
\end{tabular}

\subsubsection{Considerations}

The local function finds the proper values for indices (jdx, kdx) in the function find\_bounds. 

\end{document}
