\documentclass[12pt,letterpaper]{article} 

\usepackage[utf8]{inputenc}
\usepackage{amsmath} 
\usepackage{amsfonts} 
\usepackage{amssymb}
\usepackage{tabularx} 
\usepackage{booktabs} 
\usepackage[left=1in, right=1in,
top=1in, bottom=1in]{geometry} 

\title{CAS 741: Problem Statement\\Glass Breakage Analysis} 

\author{Vajiheh Motamer} \date{\today}

%% Comments

\usepackage{color}

\newif\ifcomments\commentstrue

\ifcomments
\newcommand{\authornote}[3]{\textcolor{#1}{[#3 ---#2]}}
\newcommand{\todo}[1]{\textcolor{red}{[TODO: #1]}}
\else
\newcommand{\authornote}[3]{}
\newcommand{\todo}[1]{}
\fi

\newcommand{\wss}[1]{\authornote{blue}{SS}{#1}}
\newcommand{\an}[1]{\authornote{magenta}{Author}{#1}}


\begin{document}

\maketitle

\begin{table}[hp] \caption{Revision History} \label{TblRevisionHistory}
\begin{tabularx}{\textwidth}{llX} 
\toprule 
\textbf{Date} & \textbf{Developer(s)} & \textbf{Change}\\ 
\midrule 
09/16/18 & Vajiheh Motamer & Initial Draft\\
09/23/18 & Vajiheh Motamer & Formatting updates, content updates related to the
                             project scope\\ 
\bottomrule 
\end{tabularx} 
\end{table}


\section{Introduction}

Glass Breakage Analysis is a computer program developed to
interpret the inputs to give out the outputs which predict whether the glass
slab can withstand the blast under the given conditions. The blast under
consideration is any type of man-made blast load. Software is helpful to
efficiently and correctly predict the blast risk involved with the glass slab
using an intuitive interface.


\section{Importance} 

Historical records show that fragments from shattered glass
as the result of blast load present a serious threat of personal injury for
people.


\section{Context}

The stakeholders for this software are software tutorial makers, software
tutorial users, technical support, technical support users, group members, and
future developers. A Team will maintain it throughout the product’s lifespan.
The consumers will be the end-users who will consume a safe glass. The software
should run on a variety of personal desktop or laptop computers using Linux,
Windows, or MacOS for use by a broad user base.

\pagenumbering{gobble} 
\end{document}