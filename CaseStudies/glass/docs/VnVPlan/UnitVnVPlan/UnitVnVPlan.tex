\documentclass[12pt]{article}
\usepackage{multirow}
\usepackage{amsmath, mathtools}
\usepackage{geometry}
\usepackage{caption}
\usepackage{bm}
\usepackage{amsmath}
\usepackage{amsfonts}
\usepackage{amssymb}
\usepackage{graphicx}
\usepackage{colortbl}
\usepackage{adjustbox}
\usepackage{xr}
\usepackage{hyperref}
\usepackage[all]{hypcap} 
\usepackage{longtable}
\usepackage{xfrac}
\usepackage{tabularx}
\usepackage{float}
\usepackage{siunitx}
\usepackage{booktabs}
\usepackage[toc, page]{appendix}
\usepackage{url}
\usepackage[usenames,dvipsnames]{xcolor}
\usepackage{array}
\usepackage{tabu}
\usepackage{enumitem}
\usepackage[numbib,nottoc]{tocbibind}
%\usepackage{refcheck}

\hypersetup{
	colorlinks=true,       % false: boxed links; true: coloured links
	linkcolor=red,          % colour of internal links (change box colour with
	% %linkbordercolor)
	citecolor=blue,        % colour of links to bibliography
	filecolor=magenta,      % colour of file links
	urlcolor=cyan           % colour of external links
}

%% Comments

\usepackage{color}

\newif\ifcomments\commentstrue

\ifcomments
\newcommand{\authornote}[3]{\textcolor{#1}{[#3 ---#2]}}
\newcommand{\todo}[1]{\textcolor{red}{[TODO: #1]}}
\else
\newcommand{\authornote}[3]{}
\newcommand{\todo}[1]{}
\fi

\newcommand{\wss}[1]{\authornote{blue}{SS}{#1}}
\newcommand{\an}[1]{\authornote{magenta}{Author}{#1}}

\externaldocument[SRS-]{../../SRS/glassbr_srs}
\newcommand{\rref}[1]{R\ref{#1}}
\newcommand{\nfrref}[1]{NFR\ref{#1}}

\externaldocument[MG-]{../../Design/MG/glassbr_mg}
\newcommand{\mref}[1]{M\ref{#1}}

\externaldocument[SVnV-]{../SystVnVPlan/SystVnVPlan}
\newcommand{\tcref}[1]{TC\ref{#1}}

\externaldocument[MIS-]{../../Design/MIS/glassbr_mis}


\newcounter{utestnum} %Assumption Number
\newcommand{\utcthetestnum}{TC\theutestnum}
\newcommand{\utcref}[1]{TC\ref{#1}}

\newcommand{\colZwidth}{1.0\textwidth}
\newcommand{\blt}{- } %used for bullets in a list
\newcommand{\colAwidth}{0.13\textwidth}
\newcommand{\colBwidth}{0.82\textwidth}
\newcommand{\colCwidth}{0.1\textwidth}
\newcommand{\colDwidth}{0.05\textwidth}
\newcommand{\colEwidth}{0.8\textwidth}
\newcommand{\colFwidth}{0.17\textwidth}
\newcommand{\colGwidth}{0.5\textwidth}
\newcommand{\colHwidth}{0.28\textwidth}
\newcounter{defnum} %Definition Number
\newcommand{\dthedefnum}{GD\thedefnum}
\newcommand{\dref}[1]{GD\ref{#1}}
\newcounter{datadefnum} %Datadefinition Number
\newcommand{\ddthedatadefnum}{DD\thedatadefnum}
\newcommand{\ddref}[1]{DD\ref{#1}}
\newcounter{theorynum} %Theory Number
\newcommand{\tthetheorynum}{T\thetheorynum}
\newcommand{\tref}[1]{T\ref{#1}}
\newcounter{tablenum} %Table Number
\newcommand{\tbthetablenum}{T\thetablenum}
\newcommand{\tbref}[1]{TB\ref{#1}}
\newcounter{assumpnum} %Assumption Number
\newcommand{\atheassumpnum}{P\theassumpnum}
\newcommand{\aref}[1]{A\ref{#1}}
\newcounter{physsysnum} %Physical System Description Number
\newcommand{\pthephyssysnum}{P\thephyssysnum}
\newcommand{\psref}[1]{PS\ref{#1}}
\newcounter{goalnum} %Goal Number
\newcommand{\gthegoalnum}{P\thegoalnum}
\newcommand{\gsref}[1]{GS\ref{#1}}
\newcounter{instnum} %Instance Number
\newcommand{\itheinstnum}{IM\theinstnum}
\newcommand{\iref}[1]{IM\ref{#1}}
\newcounter{reqnum} %Requirement Number
\newcommand{\rthereqnum}{P\thereqnum}

\newcounter{lafnum}
\newcommand{\lthelafnum}{L\thelafnum}
\newcommand{\lref}[1]{L\laf{#1}}
\newcounter{uahnum}
\newcommand{\utheuahnum}{U\theuahnum}
\newcommand{\uref}[1]{U\uaf{#1}}
\newcounter{perfnum}%Appendix
\newcommand{\ptheperfnum}{AP\theperfnum}
\newcommand{\perf}[1]{AP\perf{#1}}
\newcounter{oaenum}
\newcommand{\otheoaenum}{O\theoaenum}
\newcommand{\oae}[1]{P\oae{#1}}
\newcounter{masnum}
\newcommand{\mthemasnum}{M\themasnum}
\newcommand{\mas}[1]{P\mas{#1}}
\newcounter{secunum}
\newcommand{\sthesecunum}{S\thesecunum}
\newcommand{\secu}[1]{S\secu{#1}}
\newcounter{culnum}
\newcommand{\ctheculnum}{C\theculnum}
\newcommand{\cul}[1]{P\cul{#1}}
\newcounter{apnum}
\newcommand{\atheapnum}{L\theapnum}
\newcommand{\apref}[1]{AP\ap{#1}}

\newcounter{lcnum} %Likely change number
\newcommand{\lthelcnum}{LC\thelcnum}
\newcommand{\lcref}[1]{LC\ref{#1}}

\newcounter{ucnum} %Unlikely change number
\newcommand{\utheucnum}{UC\theucnum}
\newcommand{\ucref}[1]{UC\ref{#1}}

\newcommand{\tclad}{T_\text{CL}}
\newcommand{\degree}{\ensuremath{^\circ}}
\newcommand{\progname}{GlassBR}
\newcommand{\euler}{e}
\newcommand{\complex}{i}


\newcolumntype{P}[1]{>{\centering\arraybackslash}p{#1}}

%\oddsidemargin 0mm
%\evensidemargin 0mm
%\textwidth 160mm
%\textheight 200mm
%\usepackage{fullpage}
\newgeometry{margin=2cm}

\begin{filecontents}{../../../refs/References.bib}
\end{filecontents}


\begin{document}

\title{Project Title: Unit Verification and Validation Plan for \progname{}} 
\author{Vajiheh Motamer}
\date{\today}
	
\maketitle

\pagenumbering{roman}

\section{Revision History}

\begin{tabularx}{\textwidth}{p{3cm}p{2cm}X}
\toprule {\bf Date} & {\bf Version} & {\bf Notes}\\
\midrule
27/11/18 & 1.0 & initial UnitVnVPlan based on new template\\
07/12/18 & 2.0 & Major changes based on existing modules in the MIS\\

\bottomrule
\end{tabularx}

~\newpage

\tableofcontents

\listoftables

\wss{Do not include if not relevant}

\listoffigures

\wss{Do not include if not relevant}

\newpage

\section{Symbols, Abbreviations and Acronyms}

\renewcommand{\arraystretch}{1.2}
\begin{tabular}{l l} 
  \toprule		
  \textbf{symbol} & \textbf{description}\\
  \midrule 
  T & Test\\
  \bottomrule
\end{tabular}\\

\wss{symbols, abbreviations or acronyms -- you can reference the SRS, MG or MIS
  tables if needed}

\newpage

\pagenumbering{arabic}

This document provides the unit Verification and Validation (VnV) plan for the 
software. Section~\ref{sec_Plan} outlines at a high level the 
plan for verifying and validating the software. Section~\ref{sec_Tests} gives 
more details about the specific tests that will be used to verify each module and Access Routin Semantics.

\section{General Information}

\subsection{Purpose}

\noindent This document will identify what testing is going to be done on \progname{}. 
This document is different from the System VnV Plan in that it considers each 
module as already written according to the Module Instance Specification (MIS) 
document.These test cases can be tested using a white box approach, where the inputs, outputs, and 
transitions between the inputs and outputs are known.

\noindent The purpose of the unit verification and validation activities is to 
confirm that every module of \progname{} performs its expected actions 
correctly. The tests described in this document cannot definitively prove 
correctness, but they can build confidence by verifying that the software is 
correct for the cases covered by tests.

\subsection{Scope} \label{Scope}

\noindent The Hardware-Hiding module is implemented by the operating system of the hardware on which \progname{} 
is running, will not be unit tested as 
 and is assumed to work correctly. Verification of the non-functional requirements is 
not included in the unit verification plan because they are also sufficiently 
covered by the System VnV Plan \href{https://github.com/smiths/caseStudies/blob/glass_UnitVnV/CaseStudies/glass/docs/VnVPlan/SystVnVPlan/SystVnVPlan.pdf}
{System VnV Plan document}.


\section{Plan} \label{sec_Plan}
	
\subsection{Verification and Validation Team}
Responsible member for the verification and validation of GlassBR is Vajiheh Motamer.


\subsection{Automated Testing and Verification Tools}
According to that all test files are going to develop based on Python. The following tools have been considered:
\begin{itemize}
	\item Unit Testing Tools :
	\begin{itemize}
		\item pytest:  no API. Automatic collection of tests; simple asserts; strong support for test fixture/state management via setup/teardown hooks; strong debugging support via customized traceback. In additional, it is considered as tests runner which Selectivly run tests; Stop on first failure 
		\item unittest : Strong support for test organization and reuse via test suites 
		 \end{itemize}
	 \end{itemize}


\subsection{Non-Testing Based Verification}
This section for \progname{} is not applicable.

\section{Unit Test Description}

\noindent Test cases have been selected to verify that each module conforms to 
the specification for the module described in the 
\href{https://github.com/smiths/caseStudies/blob/master/CaseStudies/glass/docs/Design/MIS/glassbr_mis.pdf}
{Module Interface Specification (MIS) document}. The test cases has been intended to cover all Access Routine Semantics both with exceptions and without exceptions. 

~\newline \noindent Many of the tests refer to variables or constants as part 
of the initial state, input, or expected output. The specifications for these 
variables and constants can be found in the MIS document for this project.

~\newline \noindent The values in Table~\ref{defaultInputTBL} will be used as input for 
many of the test cases described throughout this section. These values were 
taken from the \href{https://github.com/smiths/caseStudies/blob/master/CaseStudies/glass/src/Python/NewImplementation/TestFiles/defaultInputFile.txt} {Default Values} for this project. Individual 
test cases will reference the table as input but specify new values for any 
input parameter that should have a different value than specified by the table.

\begin{table}[!h]
	\centering
	
	\renewcommand{\arraystretch}{1.2}
	\begin{tabular}{ | p{3cm} | p{3cm}| p{3cm} | }  
		\toprule
		\textbf{Input} & \textbf{Value} & \textbf{Unit}\\
		\midrule 
		$\text{a}$ &1600 & \text{m} \\
		$\text{b}$ &1500 & \text{m}\\
		$\text{g}$ &\text{HS} & \text{-}\\
		$P_{b_{\text{tol}}}$ &0.008& \text{-}\\
		$\text{SD}_x$ & 0 &  \si{\meter}\\
		$\text{SD}_y$ &1.5 & \si{\metre}\\
		$\text{SD}_z$ & 11.0 &\si{\metre}\\
		$\text{t}$ &10.0 & \text{mm}\\
		$\text{TNT}$ &1.0 & \text{-}\\
		$w$ &10.0	& \si{\kilo\gram}\\
		\bottomrule
	\end{tabular}
	\caption{Input to be used for test cases}
	\label{defaultInputTBL}
\end{table}


\subsection{Tests for Functional Requirements}

\subsubsection{Input Module}
~\newline As described in the MIS, the Input module is expected to read in many user 
inputs from a file. For each value contained in the file, there is a 
corresponding test case verifying that the value was properly read into the 
data structure containing the input parameters.  The input module is also responsible for 
verifying the input using describing in verify\_params() of \ref{MIS-InputARS} from MIS, so for each possible violation of an input constraint, 
there is a corresponding test case verifying that the correct exception was 
thrown.

\paragraph{Valid User Input}

~\newline ~\newline \noindent The test cases described in Table~\ref{InputTests} 
verify that each user input is correctly read. These test cases are identical 
to each other with the exception of the expected output on which they assert. 
The input for each is a file containing the inputs specified in 
Table~\ref{defaultInputTBL}. The type of these test cases is automatic. The initial 
state for each is a new session. The expected output for each is given in 
Table~\ref{InputTests}. The expected output is derived based on the given 
inputs. The tests will be performed as automated tests on a unit testing 
framework.



\begin{longtable}{  l  p{4cm}  p{6cm}  }
	\hline
	\textbf{Test Case} & \textbf{Test Name} & \textbf{Expected Output} \\
	\hline
	TC\refstepcounter{utestnum}\theutestnum  \label{TC_Length} & 
	test-input\_a &  \textit{a} = 1600\\
	\hline 
	TC\refstepcounter{utestnum}\theutestnum  \label{TC_Breadth} & 
	test-input\_b &  \textit{b} = 1500\\ 
	\hline
	TC\refstepcounter{utestnum}\theutestnum  \label{TC_GlassType} & 
	test-input\_g &  \textit{g} = \textit{HS}\\ 
	\hline
	TC\refstepcounter{utestnum}\theutestnum  \label{TC_PbTol} & 
	test-input\_$P_{b_\text{tol}}$ &  $P_{b_\text{tol}}$ = 0.008\\ 
	\hline
	TC\refstepcounter{utestnum}\theutestnum  \label{TC_SDx} & 
	test-input\_${SD}_x$ &  \textit{SDx} = 15000\\ 
	\hline
	TC\refstepcounter{utestnum}\theutestnum  \label{TC_Sdy} & 
	test-input\_$\text{SD}_y$ &  \textit{SDy} = 1.5\\
	\hline 
	TC\refstepcounter{utestnum}\theutestnum  \label{TC_SDz} & 
	test-input\_$\text{SD}_z$ &  \textit{SDz} = 11.0\\ 
	\hline
	TC\refstepcounter{utestnum}\theutestnum  \label{TC_Thickness} & 
	test-input\_$t$ &  $t$= 10.0\\ 
	\hline
	TC\refstepcounter{utestnum}\theutestnum  \label{TC_TNT} & 
	test-input\_TNT &  \textit{TNT} = 1.0\\ 
	\hline
	TC\refstepcounter{utestnum}\theutestnum  \label{TC_W} & 
	test-input\_$w$ &  $w$ = 10.0\\ 
	\hline
	\caption{Input Test Cases}
	\label{InputTests}
\end{longtable}

\paragraph{Derived Quantities}

~\newline ~\newline \noindent The test cases described in Table~\ref{DerivedValueTests} 
to ensure initial inputs from Table~\ref{defaultInputTBL}  regarding to the Constants Values and Calculations describing in load\_params($s$) of \ref{MIS-InputARS} from MIS have been correctly converted into derived quantities. The expected output for each test case have been considered in 
Table~\ref{DerivedValueTests}. The type of these test cases is automatic. The initial 
state for each test case is a new session. The tests will be performed as automated tests on a unit testing 
framework.	

\begin{longtable}{  l  p{4cm}  p{6cm}  }
	\hline
	\textbf{Test Case} & \textbf{Test Name} & \textbf{Expected Output} \\
	\hline
	TC\refstepcounter{utestnum}\theutestnum  \label{TC_LDF} & 
	test-input\_LDF &  $\text{LDF}$ = 0.2696493494752911\\
	\hline 
	TC\refstepcounter{utestnum}\theutestnum  \label{TC_h} & 
	test-input\_h &  $h$ = 9.02\\ 
	\hline
	TC\refstepcounter{utestnum}\theutestnum  \label{TC_GTF} & 
	test-input\_GTF &   $\mbox{GTF}$ = \textit{2.0}\\ 
	\hline
	TC\refstepcounter{utestnum}\theutestnum  \label{TC_SD} & 
	test-input\_SD &  $\mbox{SD}$ = 11.10180165558726\\ 
	\hline
	TC\refstepcounter{utestnum}\theutestnum  \label{TC_AR} & 
	test-input\_AR &  $\mbox{AR}$ = 1.0666666666666667\\ 
	\hline
	\caption{Derived Quantities Test Cases}
	\label{DerivedValueTests}
\end{longtable}


\paragraph{Invalid User Input}


~\newline ~\newline \noindent The test cases described in Table~\ref{SVnV-testCheckConstraints} from System VnV Plan document are intended to cover all invalid input possibilities. Invalid input is input 
that defies the data constraints described in Section 
\ref{SRS-sec_DataConstraints} of the SRS. These test cases are identical to 
each other with the exception of their input. 
~\newline \noindent  \tcref{SVnV-TC_checkAPositiveTest} - 
\tcref{SVnV-TC_incorrectWEqUpprBndTest} in the System VnV Plan document cover these set of tests.
	

\subsubsection{LoadASTM Module}	
	
~\newline This module Reads the data from each ASTM file as a text file and create an object of FunctT and add it to the ContoursT object that is described in \ref{MIS-LoadARS} from MIS as two (Access Routins Semantics) : i) LoadTSD($s$) which is for 3 second equivalent, 
pressure ($q$) versus stand off distance (SD) versus charge weight ($w$) and ii) LoadSDF($s$) which is Non dimensional lateral load ($\hat q$) versus aspect ration versus Stress
distribution factor ($J$). 


\begin{longtable}{  l  p{4cm} p{4cm}  p{6cm}  p{6cm}  }
	\hline
	\textbf{Test Case} & \textbf{Test Name} & \textbf{Expected \textit{w}} & \textbf{Expected\textit{[SD]}} & \textbf{Expected \textit{[q]}}\\ 
	\hline
	TC\refstepcounter{utestnum}\theutestnum  \label{TC_TSD} & 	test-input\_sTSD &
	[4.5, 9.1, 14., 18.] &  [[6.143397364345, 6.128184215473, 6.344639409554, 7.199850837298],[6.158648279689, 6.158648279689, 6.344639409554, 7.21772438659],[6.189263785047, 6.189263785047, 6.376179502933, 7.253604707984],[6.189263785047, 6.189263785047, 6.407876386546, 7.271611700659],[6.204628563268, 6.22003148438, 6.407876386546, 7.289663395493]] & [[6.143397364345, 6.128184215473, 6.344639409554, 7.199850837298],[6.158648279689, 6.158648279689, 6.344639409554, 7.21772438659],[6.189263785047, 6.189263785047, 6.376179502933, 7.253604707984],[6.189263785047, 6.189263785047, 6.407876386546, 7.271611700659],[6.204628563268, 6.22003148438, 6.407876386546, 7.289663395493]]  \\                                                                                                                                                                 
	\hline
	\caption{Test LoadTSD(s)}
	\label{TSDTests}
\end{longtable}



\subsubsection{Calc Module}
	
This Module has been designed for the equations for predicting the probability of glass 
breakage, capacity, and demand and other routines in the Calc module in the MIS using the input parameters.
~\newline The test cases for this Module from \tcref{SVnV-TC_defultInput} - 
\tcref{SVnV-TC_LowThicknessInput} in the System VnV Plan document cover these set of tests. 

\subsubsection{GlassType ADT Module}	
From Section \ref{MIS-GlassTypeADT} in the MIS, The implementation of the ``glass type'' and 'Glass Type Factor'. In this section it represents test cases which covers all semantic routins in this module.

\subsubsection{Thickness ADT Module}	
From Section \ref{MIS-ThicknessADT} in the MIS, the implementation of the thickness type and it Defines abstract data type for thickness. In this section it represents test cases which covers all semantic routins in this module.

	
\subsubsection{SeqServices Module}

\begin{enumerate}[label=TC\arabic*:,ref={\arabic*}]
	
	\item [TC\refstepcounter{utestnum}\theutestnum: \label{isAscendingTest}] 
	test\_isAscending\_true
	
	Type: Automatic
	
	Initial State: New Session
	
	Input: (X : [1, 2, 3, 4, 5.5]).
	
	Output: $out$ := true
	
	Test Case Derivation: This test case has been designed to Check to see if a given list is monotonically increasing. In this test case, the input has been sorted in the ascending sequence. So, the expected result would be true.
	
	How test will be performed: Automated test on unit testing framework.
	
	\item [TC\refstepcounter{utestnum}\theutestnum: \label{notisAscendingTest}] 
	test\_isAscending\_false
	
	Type: Automatic
	
	Initial State: New Session
	
	Input: (X : [1, 2, 3, 4, 0]).

	
	Output: $out$ := false
	
	Test Case Derivation: In this test case, the input has not been sorted in the ascending sequence. So, the expected result would be false.
	
	How test will be performed: Automated test on unit testing framework.
	
		\item [TC\refstepcounter{utestnum}\theutestnum: \label{isInBoundsTest}] 
	test\_isInBounds\_true
	
	Type: Automatic
	
	Initial State: New Session
	
	Input: (X : [1, 2, 3, 4, 5.5] , x : 3.5)
	
	Output: $out$ := true
	
	Test Case Derivation: Regarding with MIS, this routine Checks to see if a given value is within the bounds of a provided list. In this test case, the input has not been is within the bounds of the given list. So, the expected result would be true.
	
	How test will be performed: Automated test on unit testing framework.

	
	\item [TC\refstepcounter{utestnum}\theutestnum: \label{interpLinTest}] 
	test\_interpLin\_x
	
	Type: Automatic
	
	Initial State: It is assumed that isAscending is True and isInBounds is True as well.
	
	Input: (X : [1, 2, 3, 4, 5.5] , x : 1 , x : 3.12345678 )
	
	Output: $out$ := 0 , $out$ := 2
	
	Test Case Derivation: Regarding to th MIS, this test case Determines the best "placement" for a value against a provided list. In this test case, in the first x, it is present in list. In the second x, $x$ $<$ $X(2+1)$ and it will return $i$ which is 2.
	
	How test will be performed: Automated test on unit testing framework.
	
		
		\item [TC\refstepcounter{utestnum}\theutestnum: \label{interpQuadTest}] 
	test-interpQuad\_x
	
	Type: Automatic
	
	Initial State: It is assumed that isAscending is True.
	
	Input: ($x_0, y_0, x_1, y_1, x_2, y_2, x$) = (0, 0, 3 , 6 , 4.5 , 5.67 , 4)
	
	Output: $out := y_1 + \frac{y_2 - y_0}{x_2-x_0} (x - x_1) + \frac{y_2 - 2 y_1 + y_0}{2
		(x_2-x_1)^2} (x - x_1)^2$ = 5.8533333333333335
	
	Test Case Derivation: Regarding to the MIS, $out$ and input values are members of $\mathbb{R}$, this test case is to test of Quad interpolation function which returns interpolated value which is calculated using provided inputs and .
	
	How test will be performed: Automated test on unit testing framework.
	
	\item [TC\refstepcounter{utestnum}\theutestnum: \label{indexTest}] 
	test\_index\_x
	
	Type: Automatic
	
	Initial State: It is assumed that isAscending is True.
	
	Input: (X : [1, 2, 3, 4, 5.5] )
	
	Output: $out := y_1 + \frac{y_2 - y_0}{x_2-x_0} (x - x_1) + \frac{y_2 - 2 y_1 + y_0}{2
		(x_2-x_1)^2} (x - x_1)^2$ = 5.8533333333333335
	
	Test Case Derivation: Regarding to the MIS, $out$ and input values are members of $\mathbb{R}$, this test case is to test of Quad interpolation function which returns interpolated value which is calculated using provided inputs and .
	
	How test will be performed: Automated test on unit testing framework.


Output: $out$ := false

Test Case Derivation: In this test case, the input has not been sorted in the ascending sequence. So, the expected result would be false.

How test will be performed: Automated test on unit testing framework.

\item [TC\refstepcounter{utestnum}\theutestnum: \label{isInBoundsTest}] 
test\_isInBounds\_true

Type: Automatic

Initial State: New Session

Input: (X : [1, 2, 3, 4, 5.5] , x : 3.5)

Output: $out$ := true

Test Case Derivation: Regarding with MIS, this routine Checks to see if a given value is within the bounds of a provided list. In this test case, the input has not been is within the bounds of the given list. So, the expected result would be true.

How test will be performed: Automated test on unit testing framework.


\subsubsection{FuncADT Module}	
From Section \ref{MIS-FunctADT} in the MIS, this module is used to represent a two dimensional
function.  For instance, the data may be represented by a table of data, or a
polynomial, or a piecewise linear function, etc.
It Defines the abstract data type for functions, providing a
constructor and a means to evaluate the function for a given value of the
independent variable. In this section it represents test cases which covers all semantic routins.

\paragraph{Testing exceptions of FuncT} 

\item [TC\refstepcounter{utestnum}\theutestnum: \label{FuncTnotAscendingErrTest}] 
test\_FuncT\_notAscendingErr

Type: Automatic

Initial State: New Session

Input: (X : [5,4,6] , Y : [0, 1, 3], i: 2).

Output: $out$ := raise IndepVarNotAscending('Independent variables are not in ascending order!')


Test Case Derivation: This test case uses of isAscending from SeqServices module. Because X is not in the ascending sequence, the system must raise an exception. 

How test will be performed: Automated test on unit testing framework.

\item [TC\refstepcounter{utestnum}\theutestnum: \label{FuncTssmErrTest}] 
test\_FuncT\_ssmErr

Type: Automatic

Initial State: New Session

Input: (X : [0,1,2], Y : [0], i: 1).

Output: $out$ := raise SeqSizeMismatch('Sequences are not of the same length!')


Test Case Derivation: Because Sequences are not in the same length, the system raises an exception. 

How test will be performed: Automated test on unit testing framework.

\item [TC\refstepcounter{utestnum}\theutestnum: \label{FuncTssmErrTest}] 
test\_FuncT\_InvalidInterpErr

Type: Automatic

Initial State: New Session

Input: (X :[1, 2, 3], Y: [0, 1, 3], i : 0 or 3).

Output: $out$ := raise InvalidInterpOrder("Invalid interpretation error!")


Test Case Derivation: According to the MIS, $i$ must be between [1..\mbox{MAX\_ORDER}]. So, this test case covers this exception.

How test will be performed: Automated test on unit testing framework.

\item [TC\refstepcounter{utestnum}\theutestnum: \label{FuncToodErrTest}] 
test\_FuncT\_oodErr

Type: Automatic

Initial State: New Session

Input: ( X : [1, 2, 10] , Y : [1, 2, 10] , $o$ : 1 , $x$: 100).

Output: $out$ :=  raise OutOfDomain('Out of domain!')


Test Case Derivation:  $x$ must be between \mbox{minX} and \mbox{maxX}. So, this test case covers this exception.

How test will be performed: Automated test on unit testing framework.

\subsubsection{ContoursADT Module}	
From Section \ref{MIS-ContoursADT} in the MIS, The data structure used to represent a
sequence of two dimensional functions in the form of contours. Each function
is associated with a particular value, like a contour in a contour map.
It Defines the abstract data type for a sequence of functions.
Allows adding/removing functions to/from the sequence.  Provides evaluation of
the 3d function at values between the contours. In this section it represents test cases which covers all semantic routins in this module.

\subsubsection{Output Module}	

The verification of the output is covered by \ref{SVnV-TC_ValidOutFS} in the 
System VnV Plan document. The delivery of each output is covered by 
\tcref{SVnV-TstDrvdValsHSGlTy} to \tcref{SVnV-TstDrvdValsFTGlTy} in the System VnV Plan 
document. This document therefore only adds test cases for items not covered by 
the System VnV Plan document.


\end{enumerate}

\subsection{Traceability Between Test Cases and Modules}

Every module is covered by test cases except for those that were 
declared out of scope in Section~\ref{Scope} of this document.

\bibliographystyle{plainnat}

\bibliography{SRS}

\newpage

\bibliographystyle{plainnat}



\end{document}