\documentclass[12pt, titlepage]{article}

\usepackage{amsmath, mathtools}

\usepackage[round]{natbib}
\usepackage{amsfonts}
\usepackage{amssymb}
\usepackage{graphicx}
\usepackage{colortbl}
\usepackage{xr}
\usepackage{hyperref}
\usepackage{longtable}
\usepackage{xfrac}
\usepackage{tabularx}
\usepackage{float}
\usepackage{siunitx}
\usepackage{booktabs}
\usepackage{multirow}
\usepackage[section]{placeins}
\usepackage{caption}
\usepackage{fullpage}

\hypersetup{
bookmarks=true,     % show bookmarks bar?
colorlinks=true,       % false: boxed links; true: colored links
linkcolor=red,          % color of internal links (change box color with linkbordercolor)
citecolor=blue,        % color of links to bibliography
filecolor=magenta,  % color of file links
urlcolor=cyan          % color of external links
}

\usepackage{array}

%% Comments
\newif\ifcomments\commentstrue

\ifcomments
\newcommand{\authornote}[3]{\textcolor{#1}{[#3 ---#2]}}
\newcommand{\todo}[1]{\textcolor{red}{[TODO: #1]}}
\else
\newcommand{\authornote}[3]{}
\newcommand{\todo}[1]{}
\fi

\newcommand{\wss}[1]{\authornote{blue}{SS}{#1}}

\newcommand{\progname}{GlassBR}

\externaldocument[SRS-]{../../SRS/glassbr_srs}
\externaldocument[MG-]{../MG/glassbr_mg}

\begin{document}

\title{Module Interface Specification for \progname} 
\author{Spencer Smith}
\date{\today}

\maketitle

\tableofcontents

\newpage

\section{Introduction}

The following document details the Module Interface Specifications for the
implemented modules in the program \progname{}. It is intended to ease navigation
through the program for design and maintenance purposes.  Complementary
documents include the \href{../SRS/glassbr_srs.pdf}{System Requirement
  Specifications} (SRS) and \href{../MG/glassbr_mg.pdf}{Module Guide} (MG).  The
full documentation and implementation can be found at
\url{https://github.com/smiths/caseStudies/tree/master/CaseStudies/glass}.

\section{Notation}

The structure of the MIS for modules comes from \citet{HoffmanAndStrooper1995},
with the addition that template modules have been adapted from
\cite{GhezziEtAl2003}.  The mathematical notation comes from Chapter 3 of
\citet{HoffmanAndStrooper1995}.  For instance, the symbol := is used for a
multiple assignment statement and conditional rules follow the form $(c_1
\Rightarrow r_1 | c_2 \Rightarrow r_2 | ... | c_n \Rightarrow r_n )$.

The following table summarizes the primitive data types used by \progname. 

\begin{center}
\renewcommand{\arraystretch}{1.2}
\noindent 
\begin{tabular}{l l p{7.5cm}} 
\toprule 
\textbf{Data Type} & \textbf{Notation} & \textbf{Description}\\ 
\midrule
character & char & a single symbol or digit\\
integer & $\mathbb{Z}$ & a number without a fractional component in (-$\infty$, $\infty$) \\
natural number & $\mathbb{N}$ & a number without a fractional component in [1, $\infty$) \\
real & $\mathbb{R}$ & any number in (-$\infty$, $\infty$)\\
string & $\mathbb{S}$ & all finite sequences of symbols from the ASCII character set\\
\bottomrule
\end{tabular} 
\end{center}

\noindent
The specification of \progname{} uses some derived data types: sequences, strings, and
tuples. Sequences are lists that represent a countable number of ordered values of the same data type, where 
the same value may occur more than once. Strings are sequences of characters. 
Tuples contain a list of values, potentially of different types. 
In addition, \progname{} uses functions, which
are defined by the data types of their inputs and outputs. Local functions are
described by giving their type signature followed by their specification.

\section{Module Hierarchy} 

To view the Module Hierarchy, see section 3 of the \href{../MG/glassbr_mg.pdf}{MG}.

\bibliographystyle {plainnat}
\bibliography {../../../refs/References}

\newpage

%%%%%%%%%%%%%%%%%%%%%%%%%%%%%%%%%%%%%%%%%%%

\section{MIS of Control Module} \label{Main}

\subsection{Module}

Control

\subsection{Uses}

Input (Section~\ref{Input}), LoadASTM (Section~\ref{LoadASTM}), Calc
(Section~\ref{Calc}), Output (Section~\ref{Output})

\subsection{Syntax}

\subsubsection {Exported Constants}

sTSD = \textit{\# String for path and filename for TSD file}\\
sSDF = \textit{\# String for path and filename for SDF file}\\
sIn = \textit{\# String for path and filename for Input file}\\
sOut = \textit{\# String for path and filename for Output file}

\subsubsection{Exported Access Programs}

\begin{center}
\begin{tabular}{p{2cm} p{4cm} p{4cm} p{2cm}}
\hline
\textbf{Name} & \textbf{In} & \textbf{Out} & \textbf{Exceptions} \\
\hline
main & - & - & - \\
\hline
\end{tabular}
\end{center}

\subsection{Semantics}

\subsubsection{State Variables}

cTSD: ContoursT\\
cSDF: ContoursT\\
$q: \mathbb{R}$\\
$\hat{q}: \mathbb{R}$\\
$J_\text{tol}: \mathbb{R}$\\
$J: \mathbb{R}$\\
$\hat{q}_\text{tol}: \mathbb{R}$\\
$B: \mathbb{R}$\\
$P_b: \mathbb{R}$\\
$\text{NFL}: \mathbb{R}$\\
$\text{LR}: \mathbb{R}$\\
$\text{is\_safePb}: \mathbb{B}$\\
$\text{is\_safeLR}: \mathbb{B}$

\subsubsection{Access Routine Semantics}

\noindent main():
\begin{itemize}
\item transition: Using the steps below, modify the state variables of this
  module, the state of the Input module, and the environment variable for the
  Output module.
\begin{itemize}
\item cTSD = LoadASTM.LoadTSD(sTSD)
\item cSDF = LoadASTM.LoadSDF(sSDF)
\item Input.load\_params(sIn)
\item $q$ = cTSD.eval(Input.SD, Input.$w$ * Input.TNT) \textit{\# From SRS IM3}
\item $\hat{q}$ = Calc.calc\_q\_hat($q$)
\item $J_\text{tol}$ = Calc.calc\_J\_tol()
\item $J$ = cSDF.evaly(Input.AR, $\hat{q}$) \textit{\# From SRS DD4}
\item $\hat{q}_\text{tol}$ = cSDF.eval(Input.AR, $J_\text{tol}$) \textit{\# From SRS DD8}
\item $B$ = Calc.calc\_B($J$)
\item $P_b$ = Calc.calc\_Pb($B$)
\item NFL = Calc.calc\_NFL($\hat{q}_\text{tol}$)
\item LR = Calc.calc\_LR(NFL)
\item is\_safePb = Calc.calc\_is\_safePb($P_b$)
\item is\_safeLR = Calc.calc\_is\_safeLR(LR, $q$)
\item Output.output(sOut, $\mbox{is\_safePb}$, $\mbox{is\_safeLR}$, $P_b$, $B$, LR, $q$, $J$, NFL, $\hat{q}$,
         $\hat{q}_{\text{tol}}$, $J_{\text{tol}}$)
\item print(``Main has been executed and the results have been written to '' + sOut)
\end{itemize}
\end{itemize}

\newpage

%%%%%%%%%%%%%%%%%%%%%%%%%%%%%%%%%%%%%%%%%%%

\section{MIS of Input Module} \label{Input}

The secrets of this module are the data structure for input parameters, how the
values are input and how the values are verified.  The load and verify secrets
are isolated to their own access programs.  This module follows the singleton
pattern; that is, there is only one instance of this module.

\subsection{Module}

Input

\subsection{Uses}

GlassTypeADT (Section~\ref{GlassTypeADT}), ThicknessADT
(Section~\ref{ThicknessADT}), Constants (Section~\ref{Constants}), Hardware
(Section~\ref{Hardware})

\subsection{Syntax}

\begin{tabular}{p{3cm} p{1cm} p{3cm} >{\raggedright\arraybackslash}p{7cm}}
\toprule
\textbf{Name} & \textbf{In} & \textbf{Out} & \textbf{Exceptions} \\
\midrule
load\_params & string & - &  FileError \\
verify\_params & - & - & ValueError \\
% From R1
$a$ & - & $\mathbb{R}$\\
$b$ & - & $\mathbb{R}$\\
$g$ & - & GlassTypeT\\
$P_{b_\text{tol}}$ & - & $\mathbb{R}$\\
$\text{SD}_x$ & - & $\mathbb{R}$\\
$\text{SD}_y$ & - & $\mathbb{R}$\\
$\text{SD}_z$ & - & $\mathbb{R}$\\
$t$ & - & ThicknessT\\
TNT & - & $\mathbb{R}$\\
$w$ & - & $\mathbb{R}$\\
% From R2
$m$ & - & $\mathbb{R}$\\
$k$ & - & $\mathbb{R}$\\
$E$ & - & $\mathbb{R}$\\
$t_d$ & - & $\mathbb{R}$\\
LDF & - & $\mathbb{R}$\\
LSF & - & $\mathbb{R}$\\
% From DD2
$h$ & - & $\mathbb{R}$\\
% From DD6
GTF & - & $\mathbb{R}$\\
% From DD10
SD & - & $\mathbb{R}$\\
AR & - & $\mathbb{R}$\\
\bottomrule
\end{tabular}

\subsection{Semantics}

\subsubsection{Environment Variables}

inputFile: sequence of string \#\textit{f[i] is the ith string in the text file f}

\subsubsection{State Variables}

\renewcommand{\arraystretch}{1.2}
\begin{longtable*}[l]{l} 
\# From SRS R1\\
$a$: $\mathbb{R}$ \\
$b$: $\mathbb{R}$ \\
$g$: GlassTypeT \\
$P_{b_\text{tol}}$: $\mathbb{R}$ \\
$\text{SD}_x$ : $\mathbb{R}$ \\
$\text{SD}_y$ : $\mathbb{R}$ \\
$\text{SD}_z$ : $\mathbb{R}$ \\
$t$: ThicknessT \\
TNT: $\mathbb{R}$ \\
$w$: $\mathbb{R}$ \\
~\\
\# From SRS R2\\
$m$: $\mathbb{R}$ \\
$k$: $\mathbb{R}$ \\
$E$: $\mathbb{R}$ \\
$t_d$: $\mathbb{R}$ \\
LDF: $\mathbb{R}$ \\
LSF: $\mathbb{R}$ \\
$h$: $\mathbb{R}$ \\
GTF: $\mathbb{R}$\\
SD: $\mathbb{R}$\\
AR: $\mathbb{R}$\\
\end{longtable*}

\subsubsection{Assumptions}

\begin{itemize}

\item load\_params will be called before the values of any state variables will be accessed.

\item The file contains the string equivalents of the numeric values for
each input parameter in order, each on a new line. The order is the same as in
the table in R1 of the SRS. Any comments in the input file should be denoted
with a '\#' symbol.

\end{itemize}

\subsubsection{Access Routine Semantics}

\noindent $a$:
\begin{itemize}
\item output: \textit{out} := $a$
\item exception: none
\end{itemize}

\noindent $b$:
\begin{itemize}
\item output: \textit{out} := $b$
\item exception: none
\end{itemize}

...
~\newline

\noindent GTF:
\begin{itemize}
\item output: \textit{out} := GTF
\item exception: none
\end{itemize}

\noindent load\_params($s$):
\begin{itemize}
\item transition: The filename $s$ is first associated with the file f.  {inputFile} is used to
  modify the state variables using the following procedural specification:
\begin{enumerate}
\item Read data sequentially from inputFile to populate the state variables from
  SRS R1.
\item Calculate the derived quantities (all other state variables, from SRS R2) as follows:
\begin{itemize}
\item $m, k, E, t_d, \mbox{LSF}$ as defined in Constants
  (Section~\ref{Constants})
\item $\text{LDF}=(\frac{t_d}{60})^{m/16}$ \textit{\# From SRS DD3}
%\item $w_\text{TNT} = w \mbox{TNT}$
\item $h = t.\mbox{toMinThick()}$ (Section~\ref{ThicknessADT})
\item $\mbox{GTF} = g.\mbox{GTF}()$ (Section~\ref{GlassTypeADT})
\item $\mbox{SD} = \sqrt{{\text{SD}_x}^2 + {\text{SD}_y}^2 + {\text{SD}_z}^2}$ \textit{\# From SRS DD10}
\item $\mbox{AR} = a/b$ \textit{\# From SRS DD11}
\end{itemize}
\item verify\_params()
\end{enumerate}

\item exception: exc := a file name $s$ cannot be found OR the format of
  inputFile is incorrect $\Rightarrow$  FileError
\end{itemize}

\noindent verify\_params():
\begin{itemize}
\item out: \textit{out} := none
\item exception: exc := \textit{\# From the SRS Table 2, R3}
\end{itemize}
\noindent \begin{longtable*}[l]{l l}
  $\neg (a > 0)$ & $\Rightarrow$ ValueError(``$a$ must be positive'')\\
  $\neg (\text{AR} \geq 1.0)$ & $\Rightarrow$ ValueError(``$a$ must be greater than or equal to $b$'')\\
  $\neg (d_{\text{min}} \leq a \leq d_{\text{max}})$ & $\Rightarrow$
  ValueError(``$a$ too
  large or small'')\\
  $\neg(\text{AR}< \text{AR}_{\text{max}})$ & $\Rightarrow$ ValueError(``allowable
  aspect
  ratio exceeded'')\\
  $\neg (b > 0)$ & $\Rightarrow$ ValueError(``$b$ must be positive'')\\
  $\neg(d_{\text{min}} \leq b \leq d_{\text{max}})$ & $\Rightarrow$
  ValueError(``$b$ too
  large or small'')\\
  $\neg(0 < P_{b_{\text{tol}}} < 1)$ & $\Rightarrow$
  ValueError(``$P_{b_{\text{tol}}}$
  must be between 0 and 1'')\\
  $\neg (w > 0)$ & $\Rightarrow$ ValueError(``charge weight ($w$) must be
  greater than zero'')\\
  $\neg (w_{\text{min}} \leq w \leq w_{\text{max}})$ & $\Rightarrow$ ValueError(``charge
  weight ($w$) is too small or too large'')\\
  $\neg (\mbox{TNT} > 0)$ & $\Rightarrow$ ValueError(``TNT must be positive'')\\
  $\neg (\mbox{SD} > 0)$ & $\Rightarrow$ ValueError(``stand off distance (SD)
  must be positive'')\\
  $\neg (\text{SD}_{\text{min}}<\text{SD}<\text{SD}_{\text{max}})$ &
  $\Rightarrow$ ValueError(``stand off distance (SD) is too small or too large'')\\

\end{longtable*}

\subsection{Considerations}

The value of each state variable can be accessed through its name (getter).  An
access program is available for each state variable.  There are no setters for
the state variables, since the values will be set and checked by load params and
will not be changed for the life of the program.

\newpage

%%%%%%%%%%%%%%%%%%%%%%%%%%%%%%%%%%%%%%%%%%%

\section{MIS of LoadASTM Module} \label{LoadASTM}

\subsection {Module}

LoadASTM

\subsection {Uses}

FunctADT (Section~\ref{FunctADT}), ContoursADT (Section~\ref{ContoursADT})

\subsection {Syntax}

\subsubsection {Exported Constants}

None

\subsubsection {Exported Access Programs}

\begin{tabular}{| l | l | l | l |}
\hline
\textbf{Routine name} & \textbf{In} & \textbf{Out} & \textbf{Exceptions}\\
\hline
LoadTSD & $s: \mbox{string}$ & ContoursT & FileError\\
\hline
LoadSDF & $s: \mbox{string}$ & ContoursT & FileError\\
\hline
\end{tabular}

\subsection {Semantics}

\subsubsection {Environment Variables}

infile: two dimensional sequence of text characters

\subsubsection {State Variables}

None

\subsubsection {State Invariant}

None

\subsubsection {Assumptions}

The input file will match the given specification.

\subsubsection {Access Routine Semantics}

\noindent LoadTSD($s$)
\begin{itemize}
\item output: Create $out$ following the following steps.  Read data from the
  file infile associated with the string $s$.  Use this data to create a ContoursT
  object.  For each value of $w$, create an object of FunctT and add it to the
  ContoursT object.  Each of the FunctT objects will consist of $q$ versus SD
  data.  The first row of the TSD file contains the values of $w$.  The
  subsequent columns are grouped in pairs.  Each pair corresponds to a column of
  SD data and a column of $q$ data.  There is a pair of columns in this pattern
  for each value of $w$.\\

\item exception: exc := a file name $s$ cannot be found OR the format of
  inputFile is incorrect $\Rightarrow$  FileError
\end{itemize}

\noindent LoadSDF($s$)
\begin{itemize}
\item output: Create $out$ following the following steps.  Read data from the
  file infile associated with the string $s$.  Use this data to create a ContoursT
  object.  For each value of $J$, create an object of Funct T and add it to the
  ContoursT object.  Each of the FunctT objects will consist of $q^*$ versus AR
  data.  The first row of the SDF file contains the values of $J$.  The
  subsequent columns are grouped in pairs.  Each pair corresponds to a column of
  AR data and a column of $q^*$ data.  There
  is a pair of columns in this pattern for each value of $J$.\\
\item exception: exc := a file name $s$ cannot be found OR the format of
  inputFile is incorrect $\Rightarrow$  FileError
\end{itemize}

\newpage

%%%%%%%%%%%%%%%%%%%%%%%%%%%%%%%%%%%%%%%%%%%

\section{MIS of Calc Module} \label{Calc}

\subsection {Module}

Calc

\subsection {Uses}

Input (Section~\ref{Input}), ContoursADT (Section~\ref{ContoursADT}), Constants
(Section~\ref{Constants})

\subsection {Syntax}

\subsubsection {Exported Constants}

None

\subsubsection {Exported Access Programs}

\begin{tabular}{| l | p{7cm} | l | l |}
\hline
\textbf{Routine name} & \textbf{In} & \textbf{Out} & \textbf{Exceptions}\\
\hline
calc\_q\_hat & $q: \mathbb{R}$ & $\mathbb{R}$ & -\\
\hline
calc\_J\_tol &  & $\mathbb{R}$ & -\\
\hline
calc\_Pb & $B: \mathbb{R}$ & $\mathbb{R}$ & InvalidOutput\\
\hline
calc\_B & $J: \mathbb{R}$ & $\mathbb{R}$ & -\\
\hline
calc\_NFL & $\hat{q}_\text{tol}: \mathbb{R}$ & $\mathbb{R}$ & -\\
\hline
calc\_LR & $\text{NFL}: \mathbb{R}$ & $\mathbb{R}$ & -\\
\hline
calc\_is\_safePb & $\mathit{P_b}: \mathbb{R}$ & $\mathbb{B}$ & -\\
\hline
calc\_is\_safeLR & $\text{LR}: \mathbb{R}, q : \mathbb{R}$ & $\mathbb{B}$ & -\\
\hline
\end{tabular}

\subsection {Semantics}

\subsubsection {State Variables}

None

\subsubsection {State Invariant}

None

\subsubsection {Assumptions}

None

\subsubsection {Access Routine Semantics}

\noindent calc\_q\_hat($q$) \textit{\# From SRS DD7 Dimensionless Load ($\hat{q}$)}
\begin{itemize}
\item output:  \textit{out} := $\frac{q(ab)^2}{Eh^4\text{GTF}}$
\item exception: None
\end{itemize}

\noindent calc\_J\_tol() \textit{\# From SRS DD9 Tolerable Stress Distribution Factor ($J_\text{tol}$)}
\begin{itemize}
\item output:  \textit{out} := $\ln[\ln( \frac{1}{1-P_{b_{\text{tol}}}} )
				\frac{(a \times b)^{m-1}}
					{k(E h^2)^m \text{LDF} }]$
\item exception: None
\end{itemize}

\noindent calc\_Pb($B$) \textit{\# From SRS IM1 Probability of Glass Breakage}
\begin{itemize}
\item output:  \textit{out} := $1-e^{-B}$
\item exception: $\neg(0 < (1-e^{-B}) < 1) \Rightarrow
  \mbox{InvalidOutput}$ \textit{\# From SRS Table 3, Output Variables}
\end{itemize}

\noindent calc\_B($J$) \textit{\# From SRS DD1 Risk of Failure (B)}
\begin{itemize}
\item output:  \textit{out} := $\frac{k}{(a \times b)^{m-1}}(E h^2)^m 
					\times \text{LDF} \times e^J$
\item exception: None
\end{itemize}

\noindent calc\_NFL($\hat{q}_\text{tol}$) \textit{\# From SRS DD5 Non-Factored Load}
\begin{itemize}
\item output:  \textit{out} := $\frac{\hat{q}_{\text{tol}}Eh^4}{(ab)^2}$
\item exception: None
\end{itemize}

\noindent calc\_LR($\text{NFL}$) \textit{\# From SRS IM2 Calculation of Capacity (LR)}
\begin{itemize}
\item output:  \textit{out} := $\text{NFL} \times \text{GTF} \times \text{LSF}$
\item exception: None
\end{itemize}

\noindent calc\_is\_safePb($P_b$) \textit{\# From SRS T1 Safety Req-Pb)}
\begin{itemize}
\item output:  \textit{out} := $P_b < P_{b_{\text{tol}}}$
\item exception: None
\end{itemize}

\noindent calc\_is\_safeLR($\text{LR}, q$) \textit{\# From SRS T2 Safety Req-LR)}
\begin{itemize}
\item output:  \textit{out} := $\text{LR} > q$
\item exception: None
\end{itemize}

\newpage

%%%%%%%%%%%%%%%%%%%%%%%%%%%%%%%%%%%%%%%%%%%

\section {MIS of GlassType ADT Module} \label{GlassTypeADT}

From DD6 (Glass Type Factor (GTF)) %if keep this, should automate
                                %cross-reference

\subsection{Template Module}

GlassTypeADT

\subsection {Uses}

None

\subsection {Syntax}

\subsubsection {Exported Constants}

None

\subsubsection {Exported Types}

GlassTypeT = ?

\subsubsection {Exported Access Programs}

\begin{tabular}{| l | l | l | p{5cm} |}
\hline
\textbf{Routine name} & \textbf{In} & \textbf{Out} & \textbf{Exceptions}\\
\hline
new GlassTypeT & $\mathbb{S}$ & GlassTypeT & ValueError\\
\hline
GTF & ~ & $\mathbb{R}$ & ~\\
\hline
toString & ~ & $\mathbb{S}$ & ~\\
\hline
\end{tabular}

\subsection {Semantics}

\subsubsection {State Variables}

$g$: \{AN, FT, HS\}

\subsubsection {State Invariant}

None

\subsubsection {Assumptions}

None

\subsubsection {Access Routine Semantics}

\noindent new GlassTypeT($s$):
\begin{itemize}
\item transition: $g := (s = \mbox{``AN''} \Rightarrow \mbox{AN} | s =
  \mbox{``FT''} \Rightarrow \mbox{FT} | s = \mbox{``HS''} \Rightarrow \mbox{HS})$
\item output: $out := \mbox{self}$
\item exception: $(\neg (s \in \{ \mbox{``AN''}, \mbox{``FT''}, \mbox{``HS''} \}) \Rightarrow \mbox{ValueError})$
\end{itemize}

\noindent GTF():
\begin{itemize}
\item output: $out := ( g = \mbox{AN} \Rightarrow 1.0 | g =
  \mbox{FT} \Rightarrow 4.0 | g = \mbox{HS} \Rightarrow 2.0)$
\item exception: None
\end{itemize}

\noindent toString():
\begin{itemize}
\item output: $out := (g = \mbox{AN} \Rightarrow \mbox{``AN''} | g =
  \mbox{FT} \Rightarrow \mbox{``FT''} | g = \mbox{HS} \Rightarrow \mbox{``HS''})$
\item exception: None
\end{itemize}

\newpage

%%%%%%%%%%%%%%%%%%%%%%%%%%%%%%%%%%%%%%%%%%%

\section {MIS of Thickness ADT Module} \label{ThicknessADT}

From DD2 (Minimum Thickness from Nominal Thickness) %if keep this, should automate
                                %cross-reference

\subsection{Template Module}

ThicknessADT

\subsection {Uses}

None

\subsection {Syntax}

\subsubsection {Exported Constants}

None

\subsubsection {Exported Types}

ThicknessT = ?

\subsubsection {Exported Access Programs}

\begin{tabular}{| l | l | l | p{5cm} |}
\hline
\textbf{Routine name} & \textbf{In} & \textbf{Out} & \textbf{Exceptions}\\
\hline
new ThicknessT & $\mathbb{R}$ & ThicknessT & ValueError\\
\hline
toMinThick & ~ & $\mathbb{R}$ & ~\\
\hline
toFloat & ~ & $\mathbb{R}$ & ~\\
\hline
\end{tabular}

\subsection {Semantics}

\subsubsection {State Variables}

$t: T$ where $T = \{2.5, 2.7, 3.0, 4.0, 5.0, 6.0, 8.0, 10.0, 12.0, 16.0, 19.0, 22.0 \}$

\subsubsection {State Invariant}

None

\subsubsection {Assumptions}

None

\subsubsection {Access Routine Semantics}

\noindent new ThicknessT($x$):
\begin{itemize}
\item transition: $t := x$
\item output: $out := \mbox{self}$
\item exception: $(\neg (x \in T) \Rightarrow \mbox{ValueError})$
\end{itemize}

\noindent toMinThick():
\begin{itemize}
\item output: $\begin{array}{rcrc}
out := ( t  = 2.5 \Rightarrow 2.16 & | & t = 2.7 \Rightarrow 2.59 & |\\
t = 3.0 \Rightarrow 2.92 & | & t = 4.0 \Rightarrow 3.78 & |\\
t = 5.0 \Rightarrow 4.57 & | & t = 6.0 \Rightarrow 5.56 & |\\
t = 8.0 \Rightarrow 7.42 & | & t = 10.0 \Rightarrow 9.02 & |\\
t = 12.0 \Rightarrow 11.91 & | & t = 16.0 \Rightarrow 15.09 & |\\
t = 19.0 \Rightarrow 18.26 & | & t = 22.0 \Rightarrow 21.44 & )\\
\end{array}$
\item exception: None
\end{itemize}

\noindent toFloat():
\begin{itemize}
\item output: $out := t$
\item exception: None
\end{itemize}

\newpage

%%%%%%%%%%%%%%%%%%%%%%%%%%%%%%%%%%%%%%%%%%%

\section{MIS of FunctADT Module} \label{FunctADT}

\subsection{Template Module}

FunctADT

\subsection {Uses}

SeqServices (Section~\ref{SeqServices})

\subsection {Syntax}

\subsubsection {Exported Constants}

MAX\_ORDER = 2

\subsubsection {Exported Types}

FunctT = ?

\subsubsection {Exported Access Programs}

\begin{tabular}{| l | l | l | p{5cm} |}
\hline
\textbf{Routine name} & \textbf{In} & \textbf{Out} & \textbf{Exceptions}\\
\hline
new FunctT & $X_\text{in}: \mathbb{R}^n$, $Y_\text{in}: \mathbb{R}^n$, $i: \mathbb{N}$ &
                                                                           FunctT
                                                   &
                                                     IndepVarNotAscending,\newline
                                                     SeqSizeMismatch, \newline
                                                     InvalidInterpOrder,
                                                     \newline TooFewDataPts\\
\hline
minD & ~ & $\mathbb{R}$ & ~\\
\hline
maxD & ~ & $\mathbb{R}$ & ~\\
\hline
order & ~ & $\mathbb{N}$ & ~\\
\hline
eval & $x: \mathbb{R}$ & $\mathbb{R}$ & OutOfDomain\\
\hline

\end{tabular}

\subsection {Semantics}

\subsubsection {State Variables}

X: $\mathbb{R}^n$\\
Y: $\mathbb{R}^n$\\
minX: $\mathbb{R}$\\
maxX: $\mathbb{R}$\\
o: $\mathbb{N}$

\subsubsection {State Invariant}

None

\subsubsection {Assumptions}

None

\subsubsection {Access Routine Semantics}

\noindent new FunctT($X_\text{in}, Y_\text{in}, i$):
\begin{itemize}
\item transition: $X, Y, \mbox{minX}, \mbox{maxX}, o := X_\text{in}, Y_\text{in}, X_\text{in}[0], X_\text{in}[|X|-1], i$

\item output: $out := \mbox{self}$
\item exception: $(\neg \mbox{isAscending}(X_\text{in}) \Rightarrow
  \mbox{IndepVarNotAscending} | |X_\text{in}| \neq |Y_\text{in}| \Rightarrow \mbox{SeqSizeMismatch}
  | i \notin [1..\mbox{MAX\_ORDER}] \Rightarrow \mbox{InvalidInterpOrder} |
  |X_\text{in}| < 3 \Rightarrow \mbox{TooFewDataPts})$
\end{itemize}

\noindent minD():
\begin{itemize}
\item output: $out := \mbox{minX}$
\item exception: None
\end{itemize}

\noindent maxD():
\begin{itemize}
\item output: $out := \mbox{maxX}$
\item exception: None
\end{itemize}

\noindent order():
\begin{itemize}
\item output: $out := \mbox{o}$
\item exception: None
\end{itemize}

\noindent eval($x$):
\begin{itemize}
\item output: $out :=$ $$(o=1 \Rightarrow \mbox{interpLin}(X_i, Y_i,
X_{i+1}, Y_{i+1}, x) | o = 2 \Rightarrow \mbox{interpQuad}(X_{i-1}, Y_{i-1}, X_i, Y_i, X_{i+1},
Y_{i+1}, x) )$$  where $i = \mbox{index}(X, x)$
\item exception: $(\neg(\mbox{minX} \leq x \leq \mbox{maxX}) \Rightarrow
  \mbox{OutOfDomain}) | \neg(1 \leq \mbox{index}(X, x) \leq |X| - 2) \Rightarrow
  \mbox{OutOfDomain})$ \textit{\# first check if within domain, then make sure
    not too close to edge, so quadratic interpolation is defined}
\end{itemize}

\subsection{Considerations}

For simplicity the function evaluation is not defined within one step of the
boundaries.  By considering the special cases it would be possible to get right 
to the edge.

\newpage

%%%%%%%%%%%%%%%%%%%%%%%%%%%%%%%%%%%%%%%%%%%

\section{MIS of ContoursADT Module} \label{ContoursADT}

\subsection {Template Module}

ContoursADT

\subsection {Uses}

FunctADT (Section~\ref{FunctADT}) for FunctT

\subsection {Syntax}

\subsubsection {Exported Constants}

MAX\_ORDER = 2

\subsubsection {Exported Types}

ContoursT = ?

\subsubsection {Exported Access Programs}

\begin{tabular}{| l | l | l | l |}
\hline
\textbf{Routine name} & \textbf{In} & \textbf{Out} & \textbf{Exceptions}\\
\hline
new ContoursT & $i: \mathbb{R}$ & ~ & InvalidInterpOrder\\
\hline
add & s: FunctT, $z: \mathbb{R}$ & ~ & IndepVarNotAscending\\
\hline
getC & $i: \mathbb{N}$ & ~ & InvalidIndex\\
\hline
eval & $x: \mathbb{R}, z: \mathbb{R}$ & ~ & OutOfDomain\\
\hline
evaly & $x: \mathbb{R}, y: \mathbb{R}$ & ~ & OutOfDomain\\
\hline
slice & $x: \mathbb{R}, \text{flip}: \mathbb{B}$ & FunctT & ~\\
\hline

\end{tabular}

\subsection {Semantics}

\subsubsection {State Variables}

$S$: sequence of FunctT\\
$Z$: sequence of $\mathbb{R}$\\
o: $\mathbb{N}$

\subsubsection {State Invariant}

None

\subsubsection {Assumptions}

None

\subsubsection {Access Routine Semantics}

new ContoursT(i):
\begin{itemize}
\item transition: $S, Z, o := < >, <>, i$
\item exception: $(i \notin [1..\mbox{MAX\_ORDER}] \Rightarrow \mbox{InvalidInterpOrder})$

\end{itemize}

\noindent add($s, z$):
\begin{itemize}
\item transition: $S, Z := S || <s>, Z || <z>$
\item exception: $exc := (|Z| > 0 \wedge z
  \leq Z_{|Z|-1} \Rightarrow \mbox{IndepVarNotAscending})$
\end{itemize}

\noindent getC($i$):
\begin{itemize}
\item output: $out := S[i]$
\item exception: $exc := (\neg(0 \leq i < |S|) \Rightarrow \mbox{InvalidIndex})$
\end{itemize}

\noindent eval($x, z$):
\begin{itemize}
\item output: $out := \mbox{self.slice}(x, \text{False}).\mbox{eval}(z)$
\item exception: none \textit{\# appropriate exceptions are generated by slice
    and eval}
\end{itemize}

\noindent evaly($x, y$):
\begin{itemize}
\item output: $out := \mbox{self.slice}(x, \text{True}).\mbox{eval}(y)$
\item exception: none \textit{\# appropriate exceptions are generated by slice
    and eval}
\end{itemize}

\noindent slice($x, \text{flip}$):
\begin{itemize}
\item output and exception:
\begin{verbatim}
Zdef = []
F = []
for i in [0 .. |S|-1]:
   try y = S[i].eval(x):
     Zdef.append(Z[i])
     F.append(y)
   except OutOfDomain:
     pass
if |Zdef| > 0:
   if flip:
     return FunctT(F, Zdef, o)
   else:
     return FunctT(Zdef, F, o)
else:
   raise OutOfDomain
\end{verbatim}
%\item exception: None
\end{itemize}

\newpage

%%%%%%%%%%%%%%%%%%%%%%%%%%%%%%%%%%%%%%%%%%%

\section{MIS of SeqServices Module} \label{SeqServices}

\subsection {Module}

SeqServices

\subsection {Uses}

None

\subsection {Syntax}

\subsubsection {Exported Constants}

None

\subsubsection {Exported Access Programs}

\begin{tabular}{| l | l | l | l |}
\hline
\textbf{Routine name} & \textbf{In} & \textbf{Out} & \textbf{Exceptions}\\
\hline
isAscending & $X: \mathbb{R}^n$ & $\mathbb{B}$ & ~\\
\hline
isInBounds & $X: \mathbb{R}^n, x: \mathbb{R}$ & $\mathbb{B}$ & ~\\
\hline
interpLin & $x_1: \mathbb{R}, y_1: \mathbb{R}, x_2: \mathbb{R}, y_2: \mathbb{R},
           x: \mathbb{R}$ & $\mathbb{R}$ & ~\\
\hline
interpQuad & $ x_0: \mathbb{R}, y_0: \mathbb{R},
           x_1: \mathbb{R}, y_1: \mathbb{R}, x_2: \mathbb{R}, y_2: \mathbb{R},
           x: \mathbb{R}$ & $\mathbb{R}$ & ~\\
\hline
index & $X: \mathbb{R}^n, x: \mathbb{R}$ & $\mathbb{N}$ & ~\\

\hline

\end{tabular}

\subsection {Semantics}

\subsubsection {State Variables}

None

\subsubsection {State Invariant}

None

\subsubsection {Assumptions}

None, unless noted with a particular access program

\subsubsection {Access Routine Semantics}

\noindent isAscending($X$)
\begin{itemize}
\item output: $out := \neg \exists(i | i \in [0..|X|-2] : X_{i+1} < X_i)$
\item exception: none
\end{itemize}

\noindent isInBounds($X, x$) \textit{\# assuming isAscending is True}
\begin{itemize}
\item output: $out := X_0 \leq x \leq X_{|X|-1}$
\item exception: none
\end{itemize}

\noindent interpLin($x_1, y_1, x_2, y_2, x$) \textit{\# assuming isAscending is True}
\begin{itemize}
\item output: $out := \frac{(y_2 - y_1)}{(x_2-x_1)} (x - x_1) + y_1$
\item exception: none
\end{itemize}

\noindent interpQuad($x_0, y_0, x_1, y_1, x_2, y_2, x$) \textit{\# assuming isAscending is True}
\begin{itemize}
\item output: $out := y_1 + \frac{y_2 - y_0}{x_2-x_0} (x - x_1) + \frac{y_2 - 2 y_1 + y_0}{2
  (x_2-x_1)^2} (x - x_1)^2$
\item exception: none
\end{itemize}

\noindent index($X, x$) \textit{\# assuming isAscending is True and isInBounds is True}
\begin{itemize}
\item output: $out := i \mbox{ such that } X_i \leq x < X_{i+1}$
\item exception: none
\end{itemize}

\newpage

%%%%%%%%%%%%%%%%%%%%%%%%%%%%%%%%%%%%%%%%%%%

\section{MIS of Output Module} \label{Output}

\subsection {Module}

Output

\subsection {Uses}

Input (Section~\ref{Input}), ThicknessADT (Section~\ref{ThicknessADT}),
GlassTypeADT (Section~\ref{GlassTypeADT}), Hardware (Section~\ref{Hardware})

\subsection {Syntax}

\subsubsection {Exported Constants}

None

\subsubsection {Exported Access Programs}

\begin{tabular}{| l | p{10cm} | l | l |}
\hline
\textbf{Routine name} & \textbf{In} & \textbf{Out} & \textbf{Exceptions}\\
\hline
output & $s: \mathbb{S}$, $\mbox{is\_safePb}: \mathbb{B}$,
         $\mbox{is\_safeLR}: \mathbb{B}$, $P_b: \mathbb{R}$, $B: \mathbb{R}$, LR$: \mathbb{R}$, $q:
         \mathbb{R}$,  $J: \mathbb{R}$, NFL: $\mathbb{R}$, $\hat{q}: \mathbb{R}$,
         $\hat{q}_{\text{tol}}: \mathbb{R}$, $J_{\text{tol}}: \mathbb{R}$ & - & -\\
\hline
\end{tabular}

\subsection {Semantics}

\subsubsection {Environment Variables}

outfile: two dimensional sequence of text characters

\subsubsection {State Variables}

None

\subsubsection {State Invariant}

None

\subsubsection {Assumptions}

None

\subsubsection {Access Routine Semantics}

\noindent output($s$, $\mbox{is\_safePb}$, $\mbox{is\_safeLR}$, $P_b$, $B$, 
LR, $q$, $J$, NFL, $\hat{q}$, $\hat{q}_{\text{tol}}$, $J_{\text{tol}}$)

\begin{itemize}
\item transition: write data to the file outfile associated with the string s.
  The data to output follows:
\begin{itemize}
\item From SRS R4: values from SRS R1 ($a$, $b$, $g$, $P_{b_\text{tol}}$, $\mbox{SD}_x$,
  $\mbox{SD}_y$, $\mbox{SD}_z$, $t$, TNT, $w$), values from SRS R2 ($m$, $k$, $E$,
  $t_d$, LDF, LSF, $h$, GTF, SD, AR)
\item From SRS R5: ($\mbox{is\_safePB} \wedge \mbox{is\_safeLR}
  \Rightarrow$ ``For the given input parameters, the glass is considered safe''
 $ |$ True $\Rightarrow$ ``For the given input parameters, the glass is NOT
 considered safe'') 
\item From SRS R6 ($P_b$, $B$, LR, $q$, $J$, NFL, $\hat{q}$, $\hat{q}_{\text{tol}}$, $J_{\text{tol}}$)
\end{itemize}
\item exception: None
\end{itemize}

\newpage

%%%%%%%%%%%%%%%%%%%%%%%%%%%%%%%%%%%%%%%%%%%

\section{MIS of Constants Module} \label{Constants}

\subsection{Module}

Constants

\subsection {Uses}

N/A

\subsection {Syntax}

\subsubsection {Exported Constants}

\renewcommand{\arraystretch}{1.2}
\begin{longtable*}[l]{l} 
\# From Table 8 in SRS\\
$m$ := 7\\
$k$ := $2.86\times 10^{-53}$\\
$E$ := $7.17\times 10^{7}$\\
${t_{d}}$ := $3$\\
LSF := $1$\\
${d_{\text{max}}}$ := $5.0$\\
${d_{\text{min}}}$ := $0.1$\\
${\text{AR}_{\text{max}}}$ := $5.0$\\
${w_{\text{max}}}$ := $910.0$\\
${w_{\text{min}}}$ := $4.5$\\
${\text{SD}_{\text{min}}}$ := $6.0$\\
${\text{SD}_{\text{max}}}$ := $130.0$\\
\end{longtable*}

\subsubsection {Exported Types}

None

\subsubsection {Exported Access Programs}

None

\subsection {Semantics}

\subsubsection {State Variables}

None

\subsubsection {State Invariant}

None

\newpage

%%%%%%%%%%%%%%%%%%%%%%%%%%%%%%%%%%%%%%%%%%%

\section{MIS of Hardware Module} \label{Hardware}

This module hides the underlying hardware for I/O (to the screen, or file, or
other device).  In general it will be provided by the selected programming
language and operating system.

\subsection{Module}

Hardware

\subsection {Uses}

None

\end{document}
