\documentclass[12pt, titlepage]{article}

\usepackage{amsmath, mathtools}

\usepackage[round]{natbib}
\usepackage{amsfonts}
\usepackage{amssymb}
\usepackage{graphicx}
\usepackage{colortbl}
\usepackage{xr}
\usepackage{hyperref}
\usepackage{longtable}
\usepackage{xfrac}
\usepackage{tabularx}
\usepackage{float}
\usepackage{siunitx}
\usepackage{booktabs}
\usepackage{multirow}
\usepackage[section]{placeins}
\usepackage{caption}
\usepackage{fullpage}

\hypersetup{
bookmarks=true,     % show bookmarks bar?
colorlinks=true,       % false: boxed links; true: colored links
linkcolor=red,          % color of internal links (change box color with linkbordercolor)
citecolor=blue,        % color of links to bibliography
filecolor=magenta,  % color of file links
urlcolor=cyan          % color of external links
}

\usepackage{array}

%% Comments
\newif\ifcomments\commentstrue

\ifcomments
\newcommand{\authornote}[3]{\textcolor{#1}{[#3 ---#2]}}
\newcommand{\todo}[1]{\textcolor{red}{[TODO: #1]}}
\else
\newcommand{\authornote}[3]{}
\newcommand{\todo}[1]{}
\fi

\newcommand{\wss}[1]{\authornote{blue}{SS}{#1}}

\newcommand{\progname}[1]{Glass-BR}

\externaldocument[SRS-]{../../SRS/glassbr_srs}
\externaldocument[MG-]{../MG/glassbr_mg}

\begin{document}

\title{Module Interface Specification for \progname} 
\author{Spencer Smith}
\date{\today}

\maketitle

\tableofcontents

\newpage

\section{Introduction}

The following document details the Module Interface Specifications for the
implemented modules in a program \progname{}. It is intended to ease navigation
through the program for design and maintenance purposes.  Complementary
documents include the \href{../SRS/glassbr_srs.pdf}{System Requirement
  Specifications} (SRS) and \href{../MG/glassbr_mg.pdf}{Module Guide} (MG).  The
full documentation and implementation can be found at
\url{https://github.com/smiths/caseStudies/tree/master/CaseStudies/glass}.

\section{Notation}

The structure of the MIS for modules comes from \citet{HoffmanAndStrooper1995},
with the addition that template modules have been adapted from
\cite{GhezziEtAl2003}.  The mathematical notation comes from Chapter 3 of
\citet{HoffmanAndStrooper1995}.  For instance, the symbol := is used for a
multiple assignment statement and conditional rules follow the form $(c_1
\Rightarrow r_1 | c_2 \Rightarrow r_2 | ... | c_n \Rightarrow r_n )$.

The following table summarizes the primitive data types used by \progname. 

\begin{center}
\renewcommand{\arraystretch}{1.2}
\noindent 
\begin{tabular}{l l p{7.5cm}} 
\toprule 
\textbf{Data Type} & \textbf{Notation} & \textbf{Description}\\ 
\midrule
character & char & a single symbol or digit\\
integer & $\mathbb{Z}$ & a number without a fractional component in (-$\infty$, $\infty$) \\
natural number & $\mathbb{N}$ & a number without a fractional component in [1, $\infty$) \\
real & $\mathbb{R}$ & any number in (-$\infty$, $\infty$)\\
string & $\mathbb{S}$ & all finite sequences of symbols from the ASCII character set\\
\bottomrule
\end{tabular} 
\end{center}

\noindent
The specification of \progname \ uses some derived data types: sequences, strings, and
tuples. Sequences are lists filled with elements of the same data type. Strings
are sequences of characters. Tuples contain a list of values, potentially of
different types. In addition, \progname \ uses functions, which
are defined by the data types of their inputs and outputs. Local functions are
described by giving their type signature followed by their specification.

\section{Module Hierarchy} 

To view the Module Hierarchy, see section 3 of the \href{../MG/glassbr_mg.pdf}{MG}.

\bibliographystyle {plainnat}
\bibliography {../../../refs/References}

\newpage

%%%%%%%%%%%%%%%%%%%%%%%%%%%%%%%%%%%%%%%%%%%

\section{MIS of Input Module} \label{Input}

The secrets of this module are the data structure for input parameters, how the
values are input and how the values are verified.  The load and verify secrets
are isolated to their own access programs.  This module follows the singelton
pattern; that is, there is only one instance of this module.

\subsection{Module}

Input

\subsection{Uses}

Constants (Section~\ref{Constants}), GlassTypeADT (Section~\ref{GlassTypeADT}), ThicknessADT
(Section~\ref{ThicknessADT})

\subsection{Syntax}

\begin{tabular}{p{3cm} p{1cm} p{3cm} >{\raggedright\arraybackslash}p{7cm}}
\toprule
\textbf{Name} & \textbf{In} & \textbf{Out} & \textbf{Exceptions} \\
\midrule
load\_params & string & - &  FileError \\
verify\_params & - & - & ValueError \\
% From R1
$a$ & - & $\mathbb{R}$\\
$b$ & - & $\mathbb{R}$\\
$g$ & - & GlassTypeT\\
$P_{b_\text{tol}}$ & - & $\mathbb{R}$\\
$\text{SD}_x$ & - & $\mathbb{R}$\\
$\text{SD}_y$ & - & $\mathbb{R}$\\
$\text{SD}_z$ & - & $\mathbb{R}$\\
$t$ & - & ThicknessT\\
TNT & - & $\mathbb{R}$\\
$w$ & - & $\mathbb{R}$\\
% From R2
$m$ & - & $\mathbb{R}$\\
$k$ & - & $\mathbb{R}$\\
$E$ & - & $\mathbb{R}$\\
$t_d$ & - & $\mathbb{R}$\\
LDF & - & $\mathbb{R}$\\
LSF & - & $\mathbb{R}$\\
% From DD2
$h$ & - & $\mathbb{R}$\\
% From DD6
GTF & - & $\mathbb{R}$\\
% From DD10
SD & - & $\mathbb{R}$\\
\bottomrule
\end{tabular}

\subsection{Semantics}

\subsubsection{Environment Variables}

inputFile: sequence of string \#\textit{f[i] is the ith string in the text file f}

\subsubsection{State Variables}

\renewcommand{\arraystretch}{1.2}
\begin{longtable*}[l]{l} 
\# From R1\\
$a$: $\mathbb{R}$ \\
$b$: $\mathbb{R}$ \\
$g$: GlassTypeT \\
$P_{b_\text{tol}}$: $\mathbb{R}$ \\
$\text{SD}_x$ : $\mathbb{R}$ \\
$\text{SD}_y$ : $\mathbb{R}$ \\
$\text{SD}_z$ : $\mathbb{R}$ \\
$t$: ThicknessT \\
TNT: $\mathbb{R}$ \\
$w$: $\mathbb{R}$ \\
~\\
\# From R2\\
$m$: $\mathbb{R}$ \\
$k$: $\mathbb{R}$ \\
$E$: $\mathbb{R}$ \\
$t_d$: $\mathbb{R}$ \\
LDF: $\mathbb{R}$ \\
LSF: $\mathbb{R}$ \\
$h$: $\mathbb{R}$ \\
GTF: $\mathbb{R}$\\
SD: $\mathbb{R}$\\
\end{longtable*}

\subsubsection{Assumptions}

\begin{itemize}

\item load\_params will be called before the values of any state variables will be accessed.

\item The file contains the string equivalents of the numeric values for
each input parameter in order, each on a new line. The order is the same as in
the table in R1 of the SRS. Any comments in the input file should be denoted
with a '\#' symbol.

\end{itemize}

\subsubsection{Access Routine Semantics}

\noindent Param.$a$:
\begin{itemize}
\item output: \textit{out} := $a$
\item exception: none
\end{itemize}

\noindent Param.$b$:
\begin{itemize}
\item output: \textit{out} := $b$
\item exception: none
\end{itemize}

...
~\newline

\noindent Param.GTF:
\begin{itemize}
\item output: \textit{out} := GTF
\item exception: none
\end{itemize}

\noindent load\_params($s$):
\begin{itemize}
\item transition: The filename $s$ is first associated with the file f.  {inputFile} is used to
  modify the state variables using the following procedural specification:
\begin{enumerate}
\item Read data sequentially from inputFile to populate the state variables from
  R1 ($a$ to $w$).
\item Calculate the derived quantities (all other state variables, from R2) as follows:
\begin{itemize}
\item $m, k, E, t_d, \mbox{LDF}, \mbox{LSF}$ as defined in Constants (Section~\ref{Constants})
\item $h = t.\mbox{toMinThick()}$ (Section~\ref{ThicknessADT})
\item $\mbox{GTF} = g.\mbox{GTF}()$ (Section~\ref{GlassTypeADT})
\item $\mbox{SD} = \sqrt{{\text{SD}_x}^2 + {\text{SD}_y}^2 + {\text{SD}_z}^2}$
\end{itemize}
\item verify\_params()
\end{enumerate}

\item exception: exc := a file name $s$ cannot be found OR the format of
  inputFile is incorrect $\Rightarrow$  FileError
\end{itemize}

\noindent verify\_params():
\begin{itemize}
\item out: \textit{out} := none
\item exception: exc := 
\end{itemize}
\noindent \begin{longtable*}[l]{l l}
  $\neg (a > 0)$ & $\Rightarrow$ ValueError(``dimension $a$ must be positive'')\\
  $\neg (a/b \geq 0)$ & $\Rightarrow$ ValueError(``$a$ must be equal to, or
  greater than, $b$'')\\
  $\neg (d_{\text{min}} \leq a \leq d_{\text{max}})$ & $\Rightarrow$
  ValueError(``$a$ too
  large or small'')\\
  $\neg(a/b< \text{AR}_{\text{max}})$ & $\Rightarrow$ ValueError(``Allowable
  aspect
  ratio exceeded'')\\
  $\neg (b > 0)$ & $\Rightarrow$ ValueError(``dimension $b$ must be positive'')\\
  $\neg(d_{\text{min}} \leq b \leq d_{\text{max}})$ & $\Rightarrow$
  ValueError(``$b$ too
  large or small'')\\
  $\neg(0 < P_{b_{\text{tol}}} < 1)$ & $\Rightarrow$
  ValueError(``$P_{b_{\text{tol}}}$
  must lie between 0 and 1'')\\
  $\neg (w \geq 0)$ & $\Rightarrow$ ValueError(``charge weight $w$ must be
  greater
  or equal to zero'')\\
  $\neg (w_{\text{min}}<w<w_{\text{max}})$ & $\Rightarrow$ ValueError(``charge
  weight $w$ is too small or too large'')\\
  $\neg (\mbox{TNT} > 0)$ & $\Rightarrow$ ValueError(``TNT must be positive'')\\
  $\neg (\mbox{SD} > 0)$ & $\Rightarrow$ ValueError(``stand off distance (SD)
  must be positive'')\\
  $\neg (\text{SD}_{\text{min}}<\text{SD}<\text{SD}_{\text{max}})$ &
  $\Rightarrow$ ValueError(``stand off distance (SD) is too small or too large'')\\

\end{longtable*}

\subsection{Considerations}

The value of each state variable can be accessed through its name (getter).  An
access program is available for each state variable.  There are no setters for
the state variables, since the values will be set and checked by load params and
not changed for the life of the program.

\newpage

%%%%%%%%%%%%%%%%%%%%%%%%%%%%%%%%%%%%%%%%%%%

\section{MIS of LoadASTM Module} \label{LoadASTM}

\subsection* {Module}

LoadASTM

\subsection* {Uses}

FunctADT, ContoursADT

\subsection* {Syntax}

\subsubsection* {Exported Constants}

None

\subsubsection* {Exported Access Programs}

\begin{tabular}{| l | l | l | l |}
\hline
\textbf{Routine name} & \textbf{In} & \textbf{Out} & \textbf{Exceptions}\\
\hline
LoadTSD & $s: \mbox{string}$ & ContoursT & FileError\\
\hline
LoadSDF & $s: \mbox{string}$ & ContoursT & FileError\\
\hline
\end{tabular}

\subsection* {Semantics}

\subsubsection* {Environment Variables}

infile: two dimensional sequence of text characters

\subsubsection* {State Variables}

None

\subsubsection* {State Invariant}

None

\subsubsection* {Assumptions}

The input file will match the given specification.

\subsubsection* {Access Routine Semantics}

\noindent LoadTSD($s$)
\begin{itemize}
\item output: Create $out$ following the following steps.  read data from the
  file infile associated with the string s.  Use this data to create a ContoursT
  object. For each value of $w$ create an object of FunctT and add it to the
  ContoursT object.  Each of the FunctT objects will consist of $q$ versus SD
  data.  The first row of the TSD file contains the values of $w$.  The
  subsequent columns are grouped in pairs.  Each pair corresponds to a column of
  SD data and a column of $q$ data.  There
  is a pair of columns in this pattern for each value of $w$.\\
\item exception: exc := a file name $s$ cannot be found OR the format of
  inputFile is incorrect $\Rightarrow$  FileError
\end{itemize}

\noindent LoadSDF($s$)
\begin{itemize}
\item output: Create $out$ following the following steps.  read data from the
  file infile associated with the string s.  Use this data to create a ContoursT
  object.  For each value of $J$ create an object of Funct T and add it to the
  ContoursT object.  Each of the FunctT objects will consist of $q^*$ versus AR
  data.  The first row of the SDF file contains the values of $J$.  The
  subsequent columns are grouped in pairs.  Each pair corresponds to a column of
  AR data and a column of $q^*$ data.  There
  is a pair of columns in this pattern for each value of $J$.\\
\item exception: exc := a file name $s$ cannot be found OR the format of
  inputFile is incorrect $\Rightarrow$  FileError
\end{itemize}

\newpage

%%%%%%%%%%%%%%%%%%%%%%%%%%%%%%%%%%%%%%%%%%%

\section{MIS of Output Module} \label{Output}

\subsection* {Module}

Output

\subsection* {Uses}

Input, ThicknessADT, GlassTypeADT

\subsection* {Syntax}

\subsubsection* {Exported Constants}

None

\subsubsection* {Exported Access Programs}

\begin{tabular}{| l | p{10cm} | l | l |}
\hline
\textbf{Routine name} & \textbf{In} & \textbf{Out} & \textbf{Exceptions}\\
\hline
output & $s: \mathbb{S}$, $P_b: \mathbb{R}$, $\mbox{is}_\text{safe1}:
         \mathbb{B}$, $\mbox{is}_\text{safe2}: \mathbb{B}$, LR$: \mathbb{R}$, $q:
         \mathbb{R}$,  $h: \mathbb{R}$, LDF: $\mathbb{R}$, $J: \mathbb{R}$, NFL:
         $\mathbb{R}$, GTF: $\mathbb{R}$, $\hat{q}: \mathbb{R}$,
         $\hat{q}_{\text{tol}}: \mathbb{R}$, $J_{\text{tol}}: \mathbb{R}$, AR$:
                                    \mathbb{R}$ & - & -\\
\hline
\end{tabular}

\subsection* {Semantics}

\subsubsection* {Environment Variables}

outfile: two dimensional sequence of text characters

\subsubsection* {State Variables}

None

\subsubsection* {State Invariant}

None

\subsubsection* {Assumptions}

None

\subsubsection* {Access Routine Semantics}

\noindent output($s$, $P_b$, $\mbox{is}_\text{safe1}$, $\mbox{is}_\text{safe2}$, LR, $q$,  $h$, LDF, $J$, NFL, GTF, $\hat{q}$,
         $\hat{q}_{\text{tol}}$, $J_{\text{tol}}$, AR)
\begin{itemize}
\item transition: write data to the file outfile associated with the string s.
  The data to output follows:
\begin{itemize}
\item From R4: values from R1 ($a$, $b$, $g$, $P_{b_\text{tol}}$, $\mbox{SD}_x$,
  $\mbox{SD}_y$, $\mbox{SD}_z$, $t$, TNT, $w$), values from R2 ($m$, $k$, $E$,
  $t_d$, LDF, LSF, $h$, GTF, SD)
\item From R5: ($\mbox{is}_\text{safe1} \wedge \mbox{is}_\text{safe2}
  \Rightarrow$ ``For the given input parameters, the glass is considered safe''
 $ |$ True $\Rightarrow$ ``For the given input parameters, the glass is NOT
 considered safe'') 
\item From R6 ($s$, $P_b$, $\mbox{is}_\text{safe1}$, $\mbox{is}_\text{safe2}$, LR, $q$,  $h$, LDF, $J$, NFL, GTF, $\hat{q}$,
         $\hat{q}_{\text{tol}}$, $J_{\text{tol}}$, AR)
\end{itemize}
\item exception: None
\end{itemize}

\newpage

%%%%%%%%%%%%%%%%%%%%%%%%%%%%%%%%%%%%%%%%%%%

\section{MIS of Calc Module} \label{Calc}

\subsection* {Module}

Calc

\subsection* {Uses}

Input, Constants, ContoursADT

\subsection* {Syntax}

\subsubsection* {Exported Constants}

None

\subsubsection* {Exported Access Programs}

\begin{tabular}{| l | p{7cm} | l | l |}
\hline
\textbf{Routine name} & \textbf{In} & \textbf{Out} & \textbf{Exceptions}\\
\hline
calc\_q\_hat & $q: \mathbb{R}$ & $\mathbb{R}$ & -\\
calc\_j\_tol &  & $\mathbb{R}$ & -\\
calc\_pb & $J: \mathbb{R}$ & $\mathbb{R}$ & -\\
calc\_nfl & $\hat{q}_\text{tol}: \mathbb{R}$ & $\mathbb{R}$ & -\\
calc\_lr & $\mathit{nfl}: \mathbb{R}$ & $\mathbb{R}$ & -\\
calc\_is\_safePb & $\mathit{pb}: \mathbb{R}$ & $\mathbb{R}$ & -\\
calc\_is\_safeLr & $: \mathbb{R}$ & $\mathbb{R}$ & -\\
\hline
\end{tabular}

\subsection* {Semantics}

\subsubsection* {State Variables}

None

\subsubsection* {State Invariant}

None

\subsubsection* {Assumptions}

None

\subsubsection* {Access Routine Semantics}

\noindent calc\_q\_hat($q$) \textit{\# From SRS DD7 Dimensionless Load ($\hat{q}$)}
\begin{itemize}
\item output:  \textit{out} := $\frac{q(ab)^2}{Eh^4\text{GTF}}$
\item exception: None
\end{itemize}

\noindent calc\_j\_tol() \textit{\# From SRS DD9 Tolerable Stress Distribution Factor ($J_\text{tol}$)}
\begin{itemize}
\item output:  \textit{out} := $=\ln[\ln( \frac{1}{1-P_{b_{\text{tol}}}} )
				\frac{(\frac{a}{1000}\times\frac{b}{1000})^{m-1}}
					{k((E \times 1000)(\frac{h}{1000})^2)^m \text{LDF} }]$
\item exception: None
\end{itemize}

\noindent calc\_pb($J$) \textit{\# From SRS IM1 Probability of Glass Breakage}
\begin{itemize}
\item output:  \textit{out} := $\frac{q(ab)^2}{Eh^4\text{GTF}}$
\item exception: None
\end{itemize}

\noindent calc\_nfl($\hat{q}_\text{tol}$) \textit{\# From SRS DD7 Dimensionless Load ($\hat{q}$)}
\begin{itemize}
\item output:  \textit{out} := $\frac{q(ab)^2}{Eh^4\text{GTF}}$
\item exception: None
\end{itemize}

\noindent calc\_lr($\mathit{nfl}$) \textit{\# From SRS DD7 Dimensionless Load ($\hat{q}$)}
\begin{itemize}
\item output:  \textit{out} := $\frac{q(ab)^2}{Eh^4\text{GTF}}$
\item exception: None
\end{itemize}

\noindent calc\_is\_safePb($pb$) \textit{\# From SRS DD7 Dimensionless Load ($\hat{q}$)}
\begin{itemize}
\item output:  \textit{out} := $\frac{q(ab)^2}{Eh^4\text{GTF}}$
\item exception: None
\end{itemize}

\noindent calc\_is\_safeLr($lr, q$) \textit{\# From SRS DD7 Dimensionless Load ($\hat{q}$)}
\begin{itemize}
\item output:  \textit{out} := $\frac{q(ab)^2}{Eh^4\text{GTF}}$
\item exception: None
\end{itemize}

\newpage

%%%%%%%%%%%%%%%%%%%%%%%%%%%%%%%%%%%%%%%%%%%

\section{MIS of Control Module} \label{Main}

\subsection{Module}

main

\subsection{Uses}

Input (Section~\ref{Input}), Output (Section~\ref{Output})

\subsection{Syntax}

\subsubsection{Exported Access Programs}

\begin{center}
\begin{tabular}{p{2cm} p{4cm} p{4cm} p{2cm}}
\hline
\textbf{Name} & \textbf{In} & \textbf{Out} & \textbf{Exceptions} \\
\hline
main & - & - & - \\
\hline
\end{tabular}
\end{center}

\subsection{Semantics}

\subsubsection{State Variables}

None

\subsubsection{Access Routine Semantics}

\noindent main():
\begin{itemize}
\item transition: Modify the state of Param module and the environment variables
  for the Plot and Output modules by following these steps\\
\end{itemize}

\noindent Get (filenameIn: string) and (filenameOut: string) from user\\

\noindent load\_params(filenameIn)\\

\newpage

%%%%%%%%%%%%%%%%%%%%%%%%%%%%%%%%%%%%%%%%%%%

\section{MIS Constants Module} \label{Constants}

\subsection{Module}

Constants

\subsection {Uses}

N/A

\subsection {Syntax}

\subsubsection {Exported Constants}

\renewcommand{\arraystretch}{1.2}
\begin{longtable*}[l]{l} 
\# From Table 8 in SRS\\
$m$ := 7\\
$k$ := $\left(2.86\right)10^{-53}$\\
$E$ := $\left(7.17\right)10^{7}$\\
${t_{d}}$ := $3$\\
LDF := $\left(\frac{{t_{d}}}{60}\right)^{\frac{m}{16}}$\\
LSF := $1$\\
${d_{\text{max}}}$ := $5.0$\\
${d_{\text{min}}}$ := $0.1$\\
${\text{AR}_{\text{max}}}$ := $5.0$\\
${w_{\text{max}}}$ := $910.0$\\
${w_{\text{min}}}$ := $4.5$\\
${\text{SD}_{\text{min}}}$ := $6.0$\\
${\text{SD}_{\text{max}}}$ := $130.0$\\
\end{longtable*}

\subsubsection {Exported Types}

None

\subsubsection {Exported Access Programs}

None

\subsection {Semantics}

\subsubsection {State Variables}

None

\subsubsection {State Invariant}

None

\newpage


%%%%%%%%%%%%%%%%%%%%%%%%%%%%%%%%%%%%%%%%%%%

\section {GlassType ADT Module} \label{GlassTypeADT}

From DD6 (Glass Type Factor (GTF)) %if keep this, should automate
                                %cross-reference

\subsection{Template Module}

GlassTypeADT

\subsection {Uses}

None

\subsection {Syntax}

\subsubsection {Exported Constants}

None

\subsubsection {Exported Types}

GlassTypeT = ?

\subsubsection {Exported Access Programs}

\begin{tabular}{| l | l | l | p{5cm} |}
\hline
\textbf{Routine name} & \textbf{In} & \textbf{Out} & \textbf{Exceptions}\\
\hline
new GlassTypeT & $\mathbb{S}$ & GlassTypeT & ValueError\\
\hline
GTF & ~ & $\mathbb{R}$ & ~\\
\hline
toString & ~ & $\mathbb{S}$ & ~\\
\hline
\end{tabular}

\subsection {Semantics}

\subsubsection {State Variables}

$g$: \{AN, FT, HS\}

\subsubsection {State Invariant}

None

\subsubsection {Assumptions}

None

\subsubsection {Access Routine Semantics}

\noindent new GlassTypeT($s$):
\begin{itemize}
\item transition: $g := (s = \mbox{``AN''} \Rightarrow \mbox{AN} | s =
  \mbox{``FT''} \Rightarrow \mbox{FT} | s = \mbox{``HS''} \Rightarrow \mbox{HS})$
\item output: $out := \mbox{self}$
\item exception: $(\neg (s \in \{ \mbox{``AN''}, \mbox{``FT''}, \mbox{``HS''} \}) \Rightarrow \mbox{ValueError})$
\end{itemize}

\noindent GTF():
\begin{itemize}
\item output: $out := ( g = \mbox{AN} \Rightarrow 1.0 | g =
  \mbox{FT} \Rightarrow 4.0 | g = \mbox{HS} \Rightarrow 2.0)$
\item exception: None
\end{itemize}

\noindent toString():
\begin{itemize}
\item output: $out := (g = \mbox{AN} \Rightarrow \mbox{``AN''} | g =
  \mbox{FT} \Rightarrow \mbox{``FT''} | g = \mbox{HS} \Rightarrow \mbox{``HS''})$
\item exception: None
\end{itemize}

\newpage

%%%%%%%%%%%%%%%%%%%%%%%%%%%%%%%%%%%%%%%%%%%

\section {Thickness ADT Module} \label{ThicknessADT}

From DD2 (Minimum Thickness from Nominal Thickness) %if keep this, should automate
                                %cross-reference

\subsection{Template Module}

ThicknessADT

\subsection {Uses}

None

\subsection {Syntax}

\subsubsection {Exported Constants}

None

\subsubsection {Exported Types}

ThicknessT = ?

\subsubsection {Exported Access Programs}

\begin{tabular}{| l | l | l | p{5cm} |}
\hline
\textbf{Routine name} & \textbf{In} & \textbf{Out} & \textbf{Exceptions}\\
\hline
new ThicknessT & $\mathbb{R}$ & ThicknessT & ValueError\\
\hline
toMinThick & ~ & $\mathbb{R}$ & ~\\
\hline
toFloat & ~ & $\mathbb{R}$ & ~\\
\hline
\end{tabular}

\subsection {Semantics}

\subsubsection {State Variables}

$t: T$ where $T = \{2.5, 2.7, 3.0, 4.0, 5.0, 6.0, 8.0, 10.0, 12.0, 16.0, 19.0, 22.0 \}$

\subsubsection {State Invariant}

None

\subsubsection {Assumptions}

None

\subsubsection {Access Routine Semantics}

\noindent new ThicknessT($x$):
\begin{itemize}
\item transition: $t := x$
\item output: $out := \mbox{self}$
\item exception: $(\neg (x \in T) \Rightarrow \mbox{ValueError})$
\end{itemize}

\noindent toMinThick():
\begin{itemize}
\item output: $\begin{array}{rcrc}
out := ( t  = 2.5 \Rightarrow 2.16 & | & t = 2.7 \Rightarrow 2.59 & |\\
t = 3.0 \Rightarrow 2.92 & | & t = 4.0 \Rightarrow 3.78 & |\\
t = 5.0 \Rightarrow 4.57 & | & t = 6.0 \Rightarrow 5.56 & |\\
t = 8.0 \Rightarrow 7.42 & | & t = 10.0 \Rightarrow 9.02 & |\\
t = 12.0 \Rightarrow 11.91 & | & t = 16.0 \Rightarrow 15.09 & |\\
t = 19.0 \Rightarrow 18.26 & | & t = 22.0 \Rightarrow 21.44 & )\\
\end{array}$
\item exception: None
\end{itemize}

\noindent toFloat():
\begin{itemize}
\item output: $out := t$
\item exception: None
\end{itemize}

\newpage

%%%%%%%%%%%%%%%%%%%%%%%%%%%%%%%%%%%%%%%%%%%

\section{MIS of FunctADT Module} \label{LoadASTM}

\newpage

%%%%%%%%%%%%%%%%%%%%%%%%%%%%%%%%%%%%%%%%%%%

\section{MIS of ContoursADT Module} \label{LoadASTM}

\newpage

%%%%%%%%%%%%%%%%%%%%%%%%%%%%%%%%%%%%%%%%%%%

\section{MIS of SeqServices Module} \label{LoadASTM}

\newpage

%%%%%%%%%%%%%%%%%%%%%%%%%%%%%%%%%%%%%%%%%%%

\end{document}
