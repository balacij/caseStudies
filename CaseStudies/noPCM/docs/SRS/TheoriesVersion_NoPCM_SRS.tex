\documentclass[12pt]{article}
\usepackage{fontspec}
\usepackage{fullpage}
\usepackage{hyperref}
\hypersetup{bookmarks=true,colorlinks=true,linkcolor=red,citecolor=blue,filecolor=magenta,urlcolor=cyan}
\usepackage{amsmath}
\usepackage{amssymb}
\usepackage{mathtools}
\usepackage{unicode-math}
\usepackage{tabu}
\usepackage{longtable}
\usepackage{booktabs}
\usepackage{caption}
\usepackage{graphics}
\usepackage{svg}
\usepackage{enumitem}
\usepackage{filecontents}
\usepackage[backend=bibtex]{biblatex}
\usepackage{url}

\usepackage{color}

\newif\ifcomments\commentstrue

\ifcomments
\newcommand{\authornote}[3]{\textcolor{#1}{[#3 ---#2]}}
\newcommand{\todo}[1]{\textcolor{red}{[TODO: #1]}}
\else
\newcommand{\authornote}[3]{}
\newcommand{\todo}[1]{}
\fi

\newcommand{\wss}[1]{\authornote{blue}{SS}{#1}} 
\newcommand{\jc}[1]{\authornote{magenta}{JC}{#1}} %For explanation of the template
\newcommand{\jb}[1]{\authornote{cyan}{JB}{#1}}

\setmathfont{Latin Modern Math}
\newcommand{\gt}{\ensuremath >}
\newcommand{\lt}{\ensuremath <}
\global\tabulinesep=1mm
\newlist{symbDescription}{description}{1}
\setlist[symbDescription]{noitemsep, topsep=0pt, parsep=0pt, partopsep=0pt}
\bibliography{bibfile}
\title{Software Requirements Specification for Solar Water Heating Systems}
\author{Thulasi Jegatheesan}

\begin{document}

Recently we have determined that the exiting definitions for theoretical models,
data definitions, general definitions and instance models are not entirely
consistent.  Moreover, the terminology, especially the terminology that combines
``theory'' and ``model'' is confusing.

Our new terminology will be based on the concept of different ``levels'' of
theories.  We currently have what we are calling context theories, background
theories, theories and final theories.  Context theories are the starting point
for a given scientific model.  These theories are not defined, but their
implicit presence is necessary for completeness.  Typical examples include
theories for arithmetic operations, differentiation, integration, vector
calculus, etc.  Following the context theories we have background theories.
Background theories are defined, but not derived.  Typically they will be the
general forms of the conservation equations, like conservation of thermal energy
or momentum, and constitutive equations.  Assumptions are often invoked by The level of detail used to define
background theories can vary, since we do not always need the full theory; we
sometimes just need to know that the theory exists.  The background theories are
refined into other theories by making assumptions, like plane stress, or linear
elasticity, or isothermal material properties, or laminar flow, etc.  These
theories are combined and refined until the point where we have final theories.
All theories are part of the documentation, but the final theories are the ones
that will be transformed into code.

We also have data definitions.  A data definition is a label for part of a
theory. Assumptions will be maintained from the previous terminology.

Below is an attempt to clarify the new concept of different levels of theories
by re-writing the relevant parts of the NoPCM model using the new conceptual
model of theories.

\section{Assumptions}
\label{Sec:Assumps}

\begin{itemize}
\item[Thermal-Energy-Only:\phantomsection\label{assumpTEO}]{The only form of energy that is relevant for this problem is thermal energy. All other forms of energy, such as mechanical energy, are assumed to be negligible. (RefBy: \hyperref[BT:consThermE]{BT:consThermE}.)}
\item[Heat-Transfer-Coeffs-Constant:\phantomsection\label{assumpHTCC}]{All heat transfer coefficients are constant over time. (RefBy: \hyperref[TM:nwtnCooling]{TM:nwtnCooling}.)}
\item[Constant-Water-Temp-Across-Tank:\phantomsection\label{assumpCWTAT}]{The water in the tank is fully mixed, so the temperature of the water is the same throughout the entire tank. (RefBy: \hyperref[GD:rocTempSimp]{GD:rocTempSimp}.)}
\item[Density-Water-Constant-over-Volume:\phantomsection\label{assumpDWCoW}]{The density of water has no spatial variation; that is, it is constant over their entire volume. (RefBy: \hyperref[GD:rocTempSimp]{GD:rocTempSimp}.)}
\item[Specific-Heat-Energy-Constant-over-Volume:\phantomsection\label{assumpSHECoW}]{The specific heat capacity of water has no spatial variation; that is, it is constant over its entire volume. (RefBy: \hyperref[GD:rocTempSimp]{GD:rocTempSimp}.)}
\item[Newton-Law-Convective-Cooling-Coil-Water:\phantomsection\label{assumpLCCCW}]{Newton's law of convective cooling applies between the heating coil and the water. (RefBy: \hyperref[GD:htFluxWaterFromCoil]{GD:htFluxWaterFromCoil}.)}
\item[Temp-Heating-Coil-Constant-over-Time:\phantomsection\label{assumpTHCCoT}]{The temperature of the heating coil is constant over time. (RefBy: \hyperref[likeChgTCVOD]{LC:Temperature-Coil-Variable-Over-Day} and \hyperref[GD:htFluxWaterFromCoil]{GD:htFluxWaterFromCoil}.)}
\item[Temp-Heating-Coil-Constant-over-Length:\phantomsection\label{assumpTHCCoL}]{The temperature of the heating coil does not vary along its length. (RefBy: \hyperref[likeChgTCVOL]{LC:Temperature-Coil-Variable-Over-Length}.)}
\item[Charging-Tank-No-Temp-Discharge:\phantomsection\label{assumpCTNTD}]{The model only accounts for charging the tank, not discharging. The temperature of the water can only increase, or remain constant; it cannot decrease. This implies that the initial temperature is less than (or equal to) the temperature of the heating coil. (RefBy: \hyperref[likeChgDT]{LC:Discharging-Tank}.)}
\item[Water-Always-Liquid:\phantomsection\label{assumpWAL}]{The operating temperature range of the system is such that the material (water in this case) is always in liquid state. That is, the temperature will not drop below the melting point temperature of water, or rise above its boiling point temperature. (RefBy: \hyperref[unlikeChgWFS]{UC:Water-Fixed-States}, \hyperref[BT:sensHtE]{BT:sensHtE}, \hyperref[IM:heatEInWtr]{IM:heatEInWtr}, and \hyperref[IM:eBalanceOnWtr]{IM:eBalanceOnWtr}.)}
\item[Perfect-Insulation-Tank:\phantomsection\label{assumpPIT}]{The tank is perfectly insulated so that there is no heat loss from the tank. (RefBy: \hyperref[likeChgTLH]{LC:Tank-Lose-Heat} and \hyperref[IM:eBalanceOnWtr]{IM:eBalanceOnWtr}.)}
\item[No-Internal-Heat-Generation-By-Water:\phantomsection\label{assumpNIHGBW}]{No internal heat is generated by the water; therefore, the volumetric heat generation per unit volume is zero. (RefBy: \hyperref[unlikeChgNIHG]{UC:No-Internal-Heat-Generation} and \hyperref[IM:eBalanceOnWtr]{IM:eBalanceOnWtr}.)}
\item[Atmospheric-Pressure-Tank:\phantomsection\label{assumpAPT}]{The pressure in the tank is atmospheric, so the melting point temperature and boiling point temperature of water are 0${{}^{\circ}\text{C}}$ and 100${{}^{\circ}\text{C}}$, respectively. (RefBy: \hyperref[IM:heatEInWtr]{IM:heatEInWtr}.)}
\item[Volume-Coil-Negligible:\phantomsection\label{assumpVCN}]{When considering the volume of water in the tank, the volume of the heating coil is assumed to be negligible. (RefBy: \hyperref[DD:waterVolume.nopcm]{DD:waterVolume\_nopcm}.)}
\end{itemize}

\section{Context Theories}

Some theories do not have to be explicitly invoked.  They are part of the
context for the other theories, without having to be explicitly stated or
defined.  The context theories for this problem are as follows: arithmetic
operations, differentiation, integration, and vector calculus.

\section{Background Theories (BT)}
\label{Sec:TMs}

\vspace{\baselineskip}
\noindent
\begin{minipage}{\textwidth}
\begin{tabular}{>{\raggedright}p{0.13\textwidth}>{\raggedright\arraybackslash}p{0.82\textwidth}}
\toprule \textbf{Refname} & \textbf{BT:consThermE}
\phantomsection 
\label{BT:consThermE}
\\ \midrule \\
Label & Conservation of thermal energy
        
\\ \midrule \\
Equation & \begin{displaymath}
           -∇\cdot{}\symbf{q}+g=ρ C \frac{\,\partial{}T}{\,\partial{}t}
           \end{displaymath}
\\ \midrule \\
Description & \begin{symbDescription}
              \item{$∇$ is the gradient (Unitless)}
              \item{$\symbf{q}$ is the thermal flux vector ($\frac{\text{W}}{\text{m}^{2}}$)}
              \item{$g$ is the volumetric heat generation per unit volume ($\frac{\text{W}}{\text{m}^{3}}$)}
              \item{$ρ$ is the density ($\frac{\text{kg}}{\text{m}^{3}}$)}
              \item{$C$ is the specific heat capacity ($\frac{\text{J}}{\text{kg}{}^{\circ}\text{C}}$)}
              \item{$t$ is the time (${\text{s}}$)}
              \item{$T$ is the temperature (${{}^{\circ}\text{C}}$)}
              \end{symbDescription}
\\ \midrule \\
Notes & The above equation gives the law of conservation of energy for transient heat transfer in a given material.
        
        For this equation to apply, other forms of energy, such as mechanical
        energy, are assumed to be negligible in the system
        (\hyperref[assumpTEO]{A:Thermal-Energy-Only}).  \wss{Should we
        explicitly say that the above equation relies on vector calculus
        ``theories'', or should we leave that implicit?}
        
\\ \midrule \\
Source & \hyperref{http://www.efunda.com/formulae/heat_transfer/conduction/overview_cond.cfm}{}{}{Fourier Law of Heat Conduction and Heat Equation}
         
\\ \midrule \\
RefBy & \hyperref[GD:rocTempSimp]{GD:rocTempSimp}
        
\\ \bottomrule
\end{tabular}
\end{minipage}
\vspace{\baselineskip}

\noindent
\begin{minipage}{\textwidth}
\begin{tabular}{>{\raggedright}p{0.13\textwidth}>{\raggedright\arraybackslash}p{0.82\textwidth}}
\toprule \textbf{Refname} & \textbf{BT:sensHtE}
\phantomsection 
\label{BT:sensHtE}
\\ \midrule \\
Label & Sensible heat energy (no state change)
        
\\ \midrule \\
Equation & \begin{displaymath}
           E={C^{\text{L}}} m ΔT
           \end{displaymath}
\\ \midrule \\
Description & \begin{symbDescription}
              \item{$E$ is the sensible heat (${\text{J}}$)}
              \item{${C^{\text{L}}}$ is the specific heat capacity of a liquid ($\frac{\text{J}}{\text{kg}{}^{\circ}\text{C}}$)}
              \item{$m$ is the mass (${\text{kg}}$)}
              \item{$ΔT$ is the change in temperature (${{}^{\circ}\text{C}}$)}
              \end{symbDescription}
\\ \midrule \\
Notes & $E$ occurs as long as the material does not reach a temperature where a phase change occurs, as assumed in \hyperref[assumpWAL]{A:Water-Always-Liquid}.
        
\\ \midrule \\
Source & \hyperref{http://en.wikipedia.org/wiki/Sensible_heat}{}{}{Definition of Sensible Heat}
         
\\ \midrule \\
RefBy & \hyperref[IM:heatEInWtr]{IM:heatEInWtr}
        
\\ \bottomrule
\end{tabular}
\end{minipage}
\vspace{\baselineskip}
\noindent
\begin{minipage}{\textwidth}
\begin{tabular}{>{\raggedright}p{0.13\textwidth}>{\raggedright\arraybackslash}p{0.82\textwidth}}
\toprule \textbf{Refname} & \textbf{TM:nwtnCooling}
\phantomsection 
\label{TM:nwtnCooling}
\\ \midrule \\
Label & Newton's law of cooling
        
\\ \midrule \\
Equation & \begin{displaymath}
           q\left(t\right)=h ΔT\left(t\right)
           \end{displaymath}
\\ \midrule \\
Description & \begin{symbDescription}
              \item{$q$ is the heat flux ($\frac{\text{W}}{\text{m}^{2}}$)}
              \item{$t$ is the time (${\text{s}}$)}
              \item{$h$ is the convective heat transfer coefficient ($\frac{\text{W}}{\text{m}^{2}{}^{\circ}\text{C}}$)}
              \item{$ΔT$ is the change in temperature (${{}^{\circ}\text{C}}$)}
              \end{symbDescription}
\\ \midrule \\
Notes & Newton's law of cooling describes convective cooling from a surface. The law is stated as: the rate of heat loss from a body is proportional to the difference in temperatures between the body and its surroundings.
        
        $h$ is assumed to be independent of $T$ (from \hyperref[assumpHTCC]{A:Heat-Transfer-Coeffs-Constant}).
        
        $ΔT\left(t\right)=T\left(t\right)-{T_{\text{env}}}\left(t\right)$ is the time-dependant thermal gradient between the environment and the object.
        
\\ \midrule \\
Source & \cite[(pg. 8)]{incroperaEtAl2007}
         
\\ \midrule \\
RefBy & \hyperref[GD:htFluxWaterFromCoil]{GD:htFluxWaterFromCoil}
        
\\ \bottomrule
\end{tabular}
\end{minipage}

\section{General Definitions}
\label{Sec:GDs}
This section collects the laws and equations that will be used to build the instance models.

\vspace{\baselineskip}
\noindent
\begin{minipage}{\textwidth}
\begin{tabular}{>{\raggedright}p{0.13\textwidth}>{\raggedright\arraybackslash}p{0.82\textwidth}}
\toprule \textbf{Refname} & \textbf{GD:rocTempSimp}
\phantomsection 
\label{GD:rocTempSimp}
\\ \midrule \\
Label & Simplified rate of change of temperature
        
\\ \midrule \\
Equation & \begin{displaymath}
           m C \frac{\,dT}{\,dt}={q_{\text{in}}} {A_{\text{in}}}-{q_{\text{out}}} {A_{\text{out}}}+g V
           \end{displaymath}
\\ \midrule \\
Description & \begin{symbDescription}
              \item{$m$ is the mass (${\text{kg}}$)}
              \item{$C$ is the specific heat capacity ($\frac{\text{J}}{\text{kg}{}^{\circ}\text{C}}$)}
              \item{$t$ is the time (${\text{s}}$)}
              \item{$T$ is the temperature (${{}^{\circ}\text{C}}$)}
              \item{${q_{\text{in}}}$ is the heat flux input ($\frac{\text{W}}{\text{m}^{2}}$)}
              \item{${A_{\text{in}}}$ is the surface area over which heat is transferred in (${\text{m}^{2}}$)}
              \item{${q_{\text{out}}}$ is the heat flux output ($\frac{\text{W}}{\text{m}^{2}}$)}
              \item{${A_{\text{out}}}$ is the surface area over which heat is transferred out (${\text{m}^{2}}$)}
              \item{$g$ is the volumetric heat generation per unit volume ($\frac{\text{W}}{\text{m}^{3}}$)}
              \item{$V$ is the volume (${\text{m}^{3}}$)}
              \end{symbDescription}
\\ \midrule \\
Source & --
         
\\ \midrule \\
RefBy & \hyperref[GD:rocTempSimp]{GD:rocTempSimp} and \hyperref[IM:eBalanceOnWtr]{IM:eBalanceOnWtr}
        
\\ \bottomrule
\end{tabular}
\end{minipage}
\paragraph{Detailed derivation of simplified rate of change of temperature:}
\label{GD:rocTempSimpDeriv}
Integrating \hyperref[BT:consThermE]{BT:consThermE} over a volume ($V$), we have:

\begin{displaymath}
-\int_{V}{∇\cdot{}\symbf{q}}\,dV+\int_{V}{g}\,dV=\int_{V}{ρ C \frac{\,\partial{}T}{\,\partial{}t}}\,dV
\end{displaymath}
Applying Gauss's Divergence Theorem to the first term over the surface $S$ of the volume, with $\symbf{q}$ as the thermal flux vector for the surface and $\symbf{\hat{n}}$ as a unit outward normal vector for a surface:

\begin{displaymath}
-\int_{S}{\symbf{q}\cdot{}\symbf{\hat{n}}}\,dS+\int_{V}{g}\,dV=\int_{V}{ρ C \frac{\,\partial{}T}{\,\partial{}t}}\,dV
\end{displaymath}
We consider an arbitrary volume. The volumetric heat generation per unit volume is assumed constant. Then Equation (1) can be written as:

\begin{displaymath}
{q_{\text{in}}} {A_{\text{in}}}-{q_{\text{out}}} {A_{\text{out}}}+g V=\int_{V}{ρ C \frac{\,\partial{}T}{\,\partial{}t}}\,dV
\end{displaymath}
Where ${q_{\text{in}}}$, ${q_{\text{out}}}$, ${A_{\text{in}}}$, and ${A_{\text{out}}}$ are explained in \hyperref[GD:rocTempSimp]{GD:rocTempSimp}. Assuming $ρ$, $C$, and $T$ are constant over the volume, which is true in our case by \hyperref[assumpCWTAT]{A:Constant-Water-Temp-Across-Tank}, \hyperref[assumpDWCoW]{A:Density-Water-Constant-over-Volume}, and \hyperref[assumpSHECoW]{A:Specific-Heat-Energy-Constant-over-Volume}, we have:

\begin{displaymath}
ρ C V \frac{\,dT}{\,dt}={q_{\text{in}}} {A_{\text{in}}}-{q_{\text{out}}} {A_{\text{out}}}+g V
\end{displaymath}
Using the fact that $ρ$=$m$/$V$, Equation (2) can be written as:

\begin{displaymath}
m C \frac{\,dT}{\,dt}={q_{\text{in}}} {A_{\text{in}}}-{q_{\text{out}}} {A_{\text{out}}}+g V
\end{displaymath}
\vspace{\baselineskip}
\noindent
\begin{minipage}{\textwidth}
\begin{tabular}{>{\raggedright}p{0.13\textwidth}>{\raggedright\arraybackslash}p{0.82\textwidth}}
\toprule \textbf{Refname} & \textbf{GD:htFluxWaterFromCoil}
\phantomsection 
\label{GD:htFluxWaterFromCoil}
\\ \midrule \\
Label & Heat flux into the water from the coil
        
\\ \midrule \\
Units & $\frac{\text{W}}{\text{m}^{2}}$
        
\\ \midrule \\
Equation & \begin{displaymath}
           {q_{\text{C}}}={h_{\text{C}}} \left({T_{\text{C}}}-{T_{\text{W}}}\left(t\right)\right)
           \end{displaymath}
\\ \midrule \\
Description & \begin{symbDescription}
              \item{${q_{\text{C}}}$ is the heat flux into the water from the coil ($\frac{\text{W}}{\text{m}^{2}}$)}
              \item{${h_{\text{C}}}$ is the convective heat transfer coefficient between coil and water ($\frac{\text{W}}{\text{m}^{2}{}^{\circ}\text{C}}$)}
              \item{${T_{\text{C}}}$ is the temperature of the heating coil (${{}^{\circ}\text{C}}$)}
              \item{${T_{\text{W}}}$ is the temperature of the water (${{}^{\circ}\text{C}}$)}
              \item{$t$ is the time (${\text{s}}$)}
              \end{symbDescription}
\\ \midrule \\
Notes & ${q_{\text{C}}}$ is found by assuming that Newton's law of cooling applies (\hyperref[assumpLCCCW]{A:Newton-Law-Convective-Cooling-Coil-Water}). This law (defined in \hyperref[TM:nwtnCooling]{TM:nwtnCooling}) is used on the surface of the heating coil.
        
        \hyperref[assumpTHCCoT]{A:Temp-Heating-Coil-Constant-over-Time}
        
\\ \midrule \\
Source & \cite{koothoor2013}
         
\\ \midrule \\
RefBy & \hyperref[IM:eBalanceOnWtr]{IM:eBalanceOnWtr}
        
\\ \bottomrule
\end{tabular}
\end{minipage}

\section{Data Definitions}
\label{Sec:DDs}
This section collects and defines all the data needed to build the instance models.

\vspace{\baselineskip}
\noindent
\begin{minipage}{\textwidth}
\begin{tabular}{>{\raggedright}p{0.13\textwidth}>{\raggedright\arraybackslash}p{0.82\textwidth}}
\toprule \textbf{Refname} & \textbf{DD:waterMass}
\phantomsection 
\label{DD:waterMass}
\\ \midrule \\
Label & Mass of water
        
\\ \midrule \\
Symbol & ${m_{\text{W}}}$
         
\\ \midrule \\
Units & ${\text{kg}}$
        
\\ \midrule \\
Equation & \begin{displaymath}
           {m_{\text{W}}}={V_{\text{W}}} {ρ_{\text{W}}}
           \end{displaymath}
\\ \midrule \\
Description & \begin{symbDescription}
              \item{${m_{\text{W}}}$ is the mass of water (${\text{kg}}$)}
              \item{${V_{\text{W}}}$ is the volume of water (${\text{m}^{3}}$)}
              \item{${ρ_{\text{W}}}$ is the density of water ($\frac{\text{kg}}{\text{m}^{3}}$)}
              \end{symbDescription}
\\ \midrule \\
Source & --
         
\\ \midrule \\
RefBy & \hyperref[findMass]{FR:Find-Mass}
        
\\ \bottomrule
\end{tabular}
\end{minipage}

\vspace{\baselineskip}
\noindent
\begin{minipage}{\textwidth}
\begin{tabular}{>{\raggedright}p{0.13\textwidth}>{\raggedright\arraybackslash}p{0.82\textwidth}}
\toprule \textbf{Refname} & \textbf{DD:waterVolume.nopcm}
\phantomsection 
\label{DD:waterVolume.nopcm}
\\ \midrule \\
Label & Volume of water
        
\\ \midrule \\
Symbol & ${V_{\text{W}}}$
         
\\ \midrule \\
Units & ${\text{m}^{3}}$
        
\\ \midrule \\
Equation & \begin{displaymath}
           {V_{\text{W}}}={V_{\text{tank}}}
           \end{displaymath}
\\ \midrule \\
Description & \begin{symbDescription}
              \item{${V_{\text{W}}}$ is the volume of water (${\text{m}^{3}}$)}
              \item{${V_{\text{tank}}}$ is the volume of the cylindrical tank (${\text{m}^{3}}$)}
              \end{symbDescription}
\\ \midrule \\
Notes & Based on \hyperref[assumpVCN]{A:Volume-Coil-Negligible}. ${V_{\text{tank}}}$ is defined in \hyperref[DD:tankVolume]{DD:tankVolume}.
        
\\ \midrule \\
Source & --
         
\\ \midrule \\
RefBy & \hyperref[findMass]{FR:Find-Mass}
        
\\ \bottomrule
\end{tabular}
\end{minipage}

\vspace{\baselineskip}
\noindent
\begin{minipage}{\textwidth}
\begin{tabular}{>{\raggedright}p{0.13\textwidth}>{\raggedright\arraybackslash}p{0.82\textwidth}}
\toprule \textbf{Refname} & \textbf{DD:tankVolume}
\phantomsection 
\label{DD:tankVolume}
\\ \midrule \\
Label & Volume of the cylindrical tank
        
\\ \midrule \\
Symbol & ${V_{\text{tank}}}$
         
\\ \midrule \\
Units & ${\text{m}^{3}}$
        
\\ \midrule \\
Equation & \begin{displaymath}
           {V_{\text{tank}}}=π \left(\frac{D}{2}\right)^{2} L
           \end{displaymath}
\\ \midrule \\
Description & \begin{symbDescription}
              \item{${V_{\text{tank}}}$ is the volume of the cylindrical tank (${\text{m}^{3}}$)}
              \item{$π$ is the ratio of circumference to diameter for any circle (Unitless)}
              \item{$D$ is the diameter of tank (${\text{m}}$)}
              \item{$L$ is the length of tank (${\text{m}}$)}
              \end{symbDescription}
\\ \midrule \\
Source & --
         
\\ \midrule \\
RefBy & \hyperref[DD:waterVolume.nopcm]{DD:waterVolume\_nopcm} and \hyperref[findMass]{FR:Find-Mass}
        
\\ \bottomrule
\end{tabular}
\end{minipage}

\vspace{\baselineskip}
\noindent
\begin{minipage}{\textwidth}
\begin{tabular}{>{\raggedright}p{0.13\textwidth}>{\raggedright\arraybackslash}p{0.82\textwidth}}
\toprule \textbf{Refname} & \textbf{DD:balanceDecayRate}
\phantomsection 
\label{DD:balanceDecayRate}
\\ \midrule \\
Label & ODE parameter for water related to decay time
        
\\ \midrule \\
Symbol & ${τ_{\text{W}}}$
         
\\ \midrule \\
Units & ${\text{s}}$
        
\\ \midrule \\
Equation & \begin{displaymath}
           {τ_{\text{W}}}=\frac{{m_{\text{W}}} {C_{\text{W}}}}{{h_{\text{C}}} {A_{\text{C}}}}
           \end{displaymath}
\\ \midrule \\
Description & \begin{symbDescription}
              \item{${τ_{\text{W}}}$ is the ODE parameter for water related to decay time (${\text{s}}$)}
              \item{${m_{\text{W}}}$ is the mass of water (${\text{kg}}$)}
              \item{${C_{\text{W}}}$ is the specific heat capacity of water ($\frac{\text{J}}{\text{kg}{}^{\circ}\text{C}}$)}
              \item{${h_{\text{C}}}$ is the convective heat transfer coefficient between coil and water ($\frac{\text{W}}{\text{m}^{2}{}^{\circ}\text{C}}$)}
              \item{${A_{\text{C}}}$ is the heating coil surface area (${\text{m}^{2}}$)}
              \end{symbDescription}
\\ \midrule \\
Source & \cite{koothoor2013}
         
\\ \midrule \\
RefBy & \hyperref[outputInputDerivVals]{FR:Output-Input-Derived-Values} and \hyperref[IM:eBalanceOnWtr]{IM:eBalanceOnWtr}
        
\\ \bottomrule
\end{tabular}
\end{minipage}

\section{Instance Models}
\label{Sec:IMs}
This section transforms the problem defined in the \hyperref[Sec:ProbDesc]{problem description} into one which is expressed in mathematical terms. It uses concrete symbols defined in the \hyperref[Sec:DDs]{data definitions} to replace the abstract symbols in the models identified in \hyperref[Sec:TMs]{theoretical models} and \hyperref[Sec:GDs]{general definitions}.

The goal \hyperref[waterTempGS]{GS:Predict-Water-Temperature} is met by \hyperref[IM:eBalanceOnWtr]{IM:eBalanceOnWtr} and the goal \hyperref[waterEnergyGS]{GS:Predict-Water-Energy} is met by \hyperref[IM:heatEInWtr]{IM:heatEInWtr}.

\vspace{\baselineskip}
\noindent
\begin{minipage}{\textwidth}
\begin{tabular}{>{\raggedright}p{0.13\textwidth}>{\raggedright\arraybackslash}p{0.82\textwidth}}
\toprule \textbf{Refname} & \textbf{IM:eBalanceOnWtr}
\phantomsection 
\label{IM:eBalanceOnWtr}
\\ \midrule \\
Label & Energy balance on water to find the temperature of the water
        
\\ \midrule \\
Input & ${T_{\text{C}}}$, ${T_{\text{init}}}$, ${t_{\text{final}}}$, ${A_{\text{C}}}$, ${h_{\text{C}}}$, ${C_{\text{W}}}$, ${m_{\text{W}}}$
        
\\ \midrule \\
Output & ${T_{\text{W}}}$
         
\\ \midrule \\
Input Constraints & \begin{displaymath}
                    {T_{\text{C}}}\geq{}{T_{\text{init}}}
                    \end{displaymath}
\\ \midrule \\
Output Constraints & 
\\ \midrule \\
Equation & \begin{displaymath}
           \frac{\,d{T_{\text{W}}}}{\,dt}=\frac{1}{{τ_{\text{W}}}} \left({T_{\text{C}}}-{T_{\text{W}}}\left(t\right)\right)
           \end{displaymath}
\\ \midrule \\
Description & \begin{symbDescription}
              \item{$t$ is the time (${\text{s}}$)}
              \item{${T_{\text{W}}}$ is the temperature of the water (${{}^{\circ}\text{C}}$)}
              \item{${τ_{\text{W}}}$ is the ODE parameter for water related to decay time (${\text{s}}$)}
              \item{${T_{\text{C}}}$ is the temperature of the heating coil (${{}^{\circ}\text{C}}$)}
              \end{symbDescription}
\\ \midrule \\
Notes & ${τ_{\text{W}}}$ is calculated from \hyperref[DD:balanceDecayRate]{DD:balanceDecayRate}.
        
        The above equation applies as long as the water is in liquid form, $0\lt{}{T_{\text{W}}}\lt{}100$ (${{}^{\circ}\text{C}}$) where $0$ (${{}^{\circ}\text{C}}$) and $100$ (${{}^{\circ}\text{C}}$) are the melting and boiling point temperatures of water, respectively (\hyperref[assumpWAL]{A:Water-Always-Liquid}).
        
\\ \midrule \\
Source & \cite[(with PCM removed)]{koothoor2013}
         
\\ \midrule \\
RefBy & \hyperref[unlikeChgNIHG]{UC:No-Internal-Heat-Generation}, \hyperref[findMass]{FR:Find-Mass}, and \hyperref[calcTempWtrOverTime]{FR:Calculate-Temperature-Water-Over-Time}
        
\\ \bottomrule
\end{tabular}
\end{minipage}
\paragraph{Detailed derivation of the energy balance on water:}
\label{IM:eBalanceOnWtrDeriv}
To find the rate of change of ${T_{\text{W}}}$, we look at the energy balance on water. The volume being considered is the volume of water in the tank ${V_{\text{W}}}$, which has mass ${m_{\text{W}}}$ and specific heat capacity, ${C_{\text{W}}}$. Heat transfer occurs in the water from the heating coil as ${q_{\text{C}}}$ (\hyperref[GD:htFluxWaterFromCoil]{GD:htFluxWaterFromCoil}), over area ${A_{\text{C}}}$. No heat transfer occurs to the outside of the tank, since it has been assumed to be perfectly insulated (\hyperref[assumpPIT]{A:Perfect-Insulation-Tank}). Since the assumption is made that no internal heat is generated (\hyperref[assumpNIHGBW]{A:No-Internal-Heat-Generation-By-Water}), $g=0$. Therefore, the equation for \hyperref[GD:rocTempSimp]{GD:rocTempSimp} can be written as:

\begin{displaymath}
{m_{\text{W}}} {C_{\text{W}}} \frac{\,d{T_{\text{W}}}}{\,dt}={q_{\text{C}}} {A_{\text{C}}}
\end{displaymath}
Using \hyperref[GD:htFluxWaterFromCoil]{GD:htFluxWaterFromCoil} for ${q_{\text{C}}}$, this can be written as:

\begin{displaymath}
{m_{\text{W}}} {C_{\text{W}}} \frac{\,d{T_{\text{W}}}}{\,dt}={h_{\text{C}}} {A_{\text{C}}} \left({T_{\text{C}}}-{T_{\text{W}}}\right)
\end{displaymath}
Dividing Equation (2) by ${m_{\text{W}}} {C_{\text{W}}}$, we obtain:

\begin{displaymath}
\frac{\,d{T_{\text{W}}}}{\,dt}=\frac{{h_{\text{C}}} {A_{\text{C}}}}{{m_{\text{W}}} {C_{\text{W}}}} \left({T_{\text{C}}}-{T_{\text{W}}}\right)
\end{displaymath}
By substituting ${τ_{\text{W}}}$ (from \hyperref[DD:balanceDecayRate]{DD:balanceDecayRate}), this can be written as:

\begin{displaymath}
\frac{\,d{T_{\text{W}}}}{\,dt}=\frac{1}{{τ_{\text{W}}}} \left({T_{\text{C}}}-{T_{\text{W}}}\right)
\end{displaymath}
\vspace{\baselineskip}
\noindent
\begin{minipage}{\textwidth}
\begin{tabular}{>{\raggedright}p{0.13\textwidth}>{\raggedright\arraybackslash}p{0.82\textwidth}}
\toprule \textbf{Refname} & \textbf{IM:heatEInWtr}
\phantomsection 
\label{IM:heatEInWtr}
\\ \midrule \\
Label & Heat energy in the water
        
\\ \midrule \\
Input & ${T_{\text{init}}}$, ${m_{\text{W}}}$, ${C_{\text{W}}}$, ${m_{\text{W}}}$
        
\\ \midrule \\
Output & ${E_{\text{W}}}$
         
\\ \midrule \\
Input Constraints & 
\\ \midrule \\
Output Constraints & 
\\ \midrule \\
Equation & \begin{displaymath}
           {E_{\text{W}}}\left(t\right)={C_{\text{W}}} {m_{\text{W}}} \left({T_{\text{W}}}\left(t\right)-{T_{\text{init}}}\right)
           \end{displaymath}
\\ \midrule \\
Description & \begin{symbDescription}
              \item{${E_{\text{W}}}$ is the change in heat energy in the water (${\text{J}}$)}
              \item{$t$ is the time (${\text{s}}$)}
              \item{${C_{\text{W}}}$ is the specific heat capacity of water ($\frac{\text{J}}{\text{kg}{}^{\circ}\text{C}}$)}
              \item{${m_{\text{W}}}$ is the mass of water (${\text{kg}}$)}
              \item{${T_{\text{W}}}$ is the temperature of the water (${{}^{\circ}\text{C}}$)}
              \item{${T_{\text{init}}}$ is the initial temperature (${{}^{\circ}\text{C}}$)}
              \end{symbDescription}
\\ \midrule \\
Notes & The above equation is derived using \hyperref[BT:sensHtE]{BT:sensHtE}.
        
        The change in temperature is the difference between the temperature at time $t$ (${\text{s}}$), ${T_{\text{W}}}$ and the initial temperature, ${T_{\text{init}}}$ (${{}^{\circ}\text{C}}$).
        
        This equation applies as long as $0\lt{}{T_{\text{W}}}\lt{}100$${{}^{\circ}\text{C}}$ (\hyperref[assumpWAL]{A:Water-Always-Liquid}, \hyperref[assumpAPT]{A:Atmospheric-Pressure-Tank}).
        
\\ \midrule \\
Source & \cite{koothoor2013}
         
\\ \midrule \\
RefBy & \hyperref[calcChgHeatEnergyWtrOverTime]{FR:Calculate-Change-Heat\_Energy-Water-Over-Time}
        
\\ \bottomrule
\end{tabular}
\end{minipage}


\end{document}
